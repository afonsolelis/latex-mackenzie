\documentclass[12pt,a4paper]{article}
\usepackage[utf8]{inputenc}
\usepackage[brazil]{babel}
\usepackage{graphicx}
\usepackage{hyperref}
\usepackage{abnt-alf}
\usepackage[top=3cm,bottom=2cm,left=3cm,right=2cm]{geometry}
\usepackage{indentfirst}

\begin{document}

% CAPA
\pagestyle{empty}
\begin{center}
\large  \textbf{UNIVERSIDADE PRESBITERIANA MACKENZIE}
\large  \textbf{PROGRAMA DE PÓS-GRADUAÇÃO EM}\\
\large  \textbf{ENGENHARIA ELÉTRICA E COMPUTAÇÃO}\\
\vskip 2.0cm
\textbf{\large Nome completo}\\
\vskip 4.0cm
\setlength{\baselineskip}{1.5\baselineskip}
\textbf{\large TÍTULO DO PROJETO DE PESQUISA}\\
\vskip 4.5cm
\end{center}
\hfill{\vbox{\hsize=8.5cm\noindent\strut
Projeto de Pesquisa apresentado ao Programa\break
de Pós-Graduação em Engenharia Elétrica e\break
Computação da Universidade Presbiteriana\break
Mackenzie como parte dos requisitos para a\break
aprovação na disciplina de Metodologia do\break
Trabalho Científico.}\\
\strut}
\vskip 3.0cm
\textbf{\normalsize Orientador: }\\
\vskip 2.0cm
\begin{center}
São Paulo\\
ANO\\
\end{center}

% RESUMO
\newpage
\thispagestyle{plain}
\pagenumbering{roman}
\begin{center}
\large
\textbf{RESUMO}
\end{center}
\renewcommand{\baselinestretch}{0.6666666}
O resumo deve apresentar, em no máximo 250 palavras, o tema da pesquisa, os objetivos, o recorte teórico-metodológico e os resultados almejados. A proposta do projeto de pesquisa é estabelecer uma primeira delimitação do trabalho que será realizado ao longo do Mestrado. Neste sentido, a apresentação do projeto de pesquisa é realizada por meio de um texto que possui a estrutura de divisões deste documento, a formatação de um mínimo de dezessete (17) a vinte (20) laudas de conteúdo em papel A4 branco com espaçamento entre linhas de 1,5.
\\[0.5cm]
\begin{flushleft}
{\bf Palavras-chave:} {\it apresentação, separada por vírgulas, de três a seis unitermos significativos para o trabalho.}
\end{flushleft}

% SUMÁRIO
\newpage
\thispagestyle{empty}
\tableofcontents

% DESENVOLVIMENTO
\newpage
\pagestyle{plain}
\pagenumbering{arabic}
\renewcommand{\baselinestretch}{1.5}
\normalsize
\section{INTRODUÇÃO}
Ao longo da introdução deve-se desenvolver o tema de forma a construir um cenário que possibilite compreender a delimitação do estudo e o contexto onde este se encontra inserido.

Neste item também é necessário apresentar o objetivo geral da pesquisa, ou seja, a meta proposta para a investigação. De forma complementar, os objetivos específicos, também presentes na introdução, constituem os elementos de trabalho que permitem alcançar o objetivo geral. Como os objetivos traduzem ações que serão executadas ao longo da pesquisa, a apresentação destes no texto requer a utilização de verbos no infinitivo.

Ainda, deve-se colocar a hipótese: uma afirmação provisória que será confirmada ou refutada pelo pesquisador ao longo da investigação.

Finalmente, é preciso descrever as partes do projeto de pesquisa.

\section{JUSTIFICATIVA}
Na justificativa deve-se colocar, de maneira clara e objetiva, os elementos teórico-práticos que demonstram a relevância para a realização da pesquisa, bem como as possíveis contribuições resultantes do trabalho proposto.

\section{REFERENCIAL TEÓRICO}
A fundamentação teórica estabelece os contornos da problemática da pesquisa por meio da descrição/definição de conceitos essenciais ao tema estudado, da apresentação de um panorama histórico ou mesmo de uma compilação de aspectos tratados em outras pesquisas e do delineamento das lacunas encontradas na literatura.

A bibliografia selecionada para a construção do quadro teórico deverá considerar a relevância e atualidade em relação ao tema em questão. Ao longo da apresentação da revisão da literatura deve-se demonstrar o entendimento e articulação entre os conceitos e definições que fazem parte da área e da especificidade da pesquisa que será desenvolvida. Neste sentido, esta parte do projeto também comporta, conforme o caso, subitens temáticos que possibilitem a melhor compreensão do contexto da investigação.

De maneira complementar, a apresentação de conceitos, definições, etc., deve ser feita por meio de paráfrase, sendo a referência utilizada incluída no formato ``(AUTOR, Ano)''. A não apresentação da referência ou cópia literal de elementos textuais sem a indicação da fonte bibliográfica pertinente constitui plágio e é passível de punições, inclusive de reprovação na disciplina.

Constam da {\it bibliografia básica} os elementos bibliográficos referenciados ao longo do projeto de pesquisa e outras bibliografias que ainda serão consultadas e estudas e ao longo do desenvolvimento do trabalho. A formatação da bibliografia básica deverá obedecer às regras da ABNT (NBR 6023/2015) e ser disposta em ordem alfabética (conforme exemplo apresentado no final deste documento).

Abaixo encontram-se exemplos de citação:

Citação entre parênteses após paráfrase \cite{andrade99}.

Citação em que o autor \citeonline{gil91} é referenciado durante a explicação.

\section{METODOLOGIA}
Descrição detalhada e sequencial dos procedimentos científicos (métodos e técnicas) estabelecidos para atingir os objetivos propostos inicialmente para a pesquisa.

\section{CRONOGRAMA}
O cronograma do projeto de pesquisa deverá apresentar as etapas do trabalho e a previsão do tempo necessário para a realização de cada uma destas fases em um período de 12 meses. Inclusive, é necessário atentar para o fato de que existem partes da atividade científica que podem ser realizadas simultaneamente, enquanto outras possuem uma estrutura hierárquica de dependência.

\def\refname{REFERÊNCIAS BIBLIOGRÁFICAS}
\bibliography{biblproj}
\addcontentsline{toc}{section}{REFERÊNCIAS BIBLIOGRÁFICAS}
\bibliographystyle{abnt-alf}

\end{document}
