%
% Template for RBIE papers in LaTeX
%

\documentclass[english, spanish, brazilian]{RBIEarticle} % for papers in portuguese

\usepackage[utf8]{inputenc} % chooses UTF-8 as the main character set
\usepackage[T1]{fontenc} % for correct syllable separation in accented words

% Fix babel caption warnings
\AtBeginDocument{%
  \setlocalecaption{brazilian}{abstract}{Resumo}%
  \setlocalecaption{spanish}{abstract}{Resumen}%
}

% Citations and references
% \usepackage[style=apa]{biblatex}
% \usepackage{csquotes}
% \addbibresource{references.bib}

% Here goes the paper main title
\title{Proposta de Gêmeo Digital Híbrido para Avaliação de Metaprojetos em Project Based Learning}

% If the manuscript is written in English, then this element must be removed.
\titleinenglish{Digital Twin Proposal for Processes and Systems Applied to Metaproject Evaluation in Project Based Learning}

% If the manuscript is written in English, then this element must be removed.
\titleinspanish{Propuesta de Gemelo Digital de Procesos y Sistemas para Evaluación de Metaproyectos en Project Based Learning}

% Here goes the paper author information (repeat for two or more authors)
\author{%
	\parbox{8cm}{%
		Afonso Cesar Lelis Brandão\\
		Instituto de Tecnologia e Liderança (Inteli)\\
		ORCID: \href{https://orcid.org/0009-0009-4814-9502}{0009-0009-4814-9502}\\
		afonso.brandao@prof.inteli.edu.br
	}
}


% Here goes the page heading information
\heading{Brandão}{RBIE v.XX – 2025}

% And finally here goes the citation information
\citeas{Brandão, A. C. L. (2025). Proposta de Gêmeo Digital Híbrido para Avaliação de Metaprojetos em Project Based Learning. Revista Brasileira de Informática na Educação, vol, pp-pp. https://doi.org/10.5753/rbie.2025.id}

%====================================================================

\begin{document}
\maketitle

% If the manuscript is written in English, then this element must be removed.
\begin{otherlanguage}{brazilian}
  \begin{abstract}
    Este artigo propõe um modelo inovador de gêmeo digital aplicado à avaliação de metaprojetos em Project-Based Learning (PBL) no contexto da engenharia de software. A proposta integra gêmeos digitais de processos e sistemas para superar as limitações dos métodos tradicionais de avaliação, que se caracterizam pela natureza pontual e subjetiva. O modelo conceitual combina três visões arquiteturais: estrutural, comportamental e de processo, permitindo captura multidimensional de dados pedagógicos, técnicos, de gestão e de parceria externa. A arquitetura técnica em camadas suporta coleta automatizada de dados de múltiplas fontes, processamento através de algoritmos de Educational Data Mining e representação em modelos virtuais atualizados continuamente. A validação será realizada através de Design Science Research em ambiente real, utilizando métricas específicas para avaliar eficácia, objetividade e continuidade da avaliação. As contribuições esperadas incluem um novo paradigma para instrumentos de avaliação em aprendizagem ativa, framework sistemático para avaliação multidimensional em PBL e expansão das aplicações de gêmeos digitais para contextos educacionais.
    \keywords{Gêmeo Digital; Project-Based Learning; Avaliação Educacional; Engenharia de Software; Educational Data Mining.}
  \end{abstract}
\end{otherlanguage}

\begin{otherlanguage}{english}
  \begin{abstract}
    This paper proposes an innovative digital twin model applied to the evaluation of metaprojects in Project-Based Learning (PBL) within software engineering contexts. The proposal integrates process and system digital twins to overcome limitations of traditional evaluation methods, characterized by their punctual and subjective nature. The conceptual model combines three architectural views: structural, behavioral, and process-based, enabling multidimensional data capture across pedagogical, technical, management, and external partnership dimensions. The technical architecture supports automated data collection from multiple sources, processing through Educational Data Mining algorithms, and representation in continuously updated virtual models. Validation will be conducted through Design Science Research in real environments, using specific metrics to evaluate effectiveness, objectivity, and continuity of assessment. Expected contributions include a new paradigm for assessment instruments in active learning, a systematic framework for multidimensional PBL evaluation, and expansion of digital twin applications to educational contexts.
    \keywords{Digital Twin; Project-Based Learning; Educational Assessment; Software Engineering; Educational Data Mining.}
  \end{abstract}
\end{otherlanguage}

\begin{otherlanguage}{spanish}
  \begin{abstract}
    Este artículo propone un modelo innovador de gemelo digital aplicado a la evaluación de metaproyectos en Project-Based Learning (PBL) en el contexto de la ingeniería de software. La propuesta integra gemelos digitales de procesos y sistemas para superar las limitaciones de los métodos tradicionales de evaluación, caracterizados por su naturaleza puntual y subjetiva. El modelo conceptual combina tres visiones arquitectónicas: estructural, comportamental y de proceso, permitiendo captura multidimensional de datos pedagógicos, técnicos, de gestión y de asociación externa. La arquitectura técnica en capas soporta recolección automatizada de datos de múltiples fuentes, procesamiento a través de algoritmos de Educational Data Mining y representación en modelos virtuales actualizados continuamente. La validación será realizada a través de Design Science Research en ambiente real, utilizando métricas específicas para evaluar eficacia, objetividad y continuidad de la evaluación. Las contribuciones esperadas incluyen un nuevo paradigma para instrumentos de evaluación en aprendizaje activo, framework sistemático para evaluación multidimensional en PBL y expansión de las aplicaciones de gemelos digitales para contextos educacionales.
    \keywords{Gemelo Digital; Project-Based Learning; Evaluación Educacional; Ingeniería de Software; Educational Data Mining.}
  \end{abstract}
\end{otherlanguage}

\pagebreak

%====================================================================

\section{Introdução}

\indent

A educação em engenharia de software tem vivenciado transformação significativa
através da adoção crescente de metodologias ativas de aprendizagem,
particularmente o Project-Based Learning (PBL). Esta abordagem pedagógica
organiza o processo educacional em torno de projetos autênticos que integram
competências técnicas e transversais, proporcionando aos estudantes experiência
prática na resolução de problemas reais da indústria enquanto desenvolvem
conhecimentos teóricos de forma contextualizada e significativa.

No Brasil, instituições pioneiras como o Instituto de Tecnologia e Liderança
(Inteli) têm implementado modelos educacionais integralmente baseados em PBL,
estruturados através de parcerias estratégicas com organizações externas que
fornecem problemas reais como base para projetos estudantis. Esta aproximação
entre academia e indústria, pedagogicamente valiosa, amplifica desafios
tradicionais de avaliação ao introduzir múltiplos stakeholders, critérios
diversificados e necessidades de feedback contínuo que transcendem as
capacidades dos métodos avaliatórios convencionais.

Os sistemas tradicionais de avaliação, predominantemente pontuais e baseados em
marcos pré-estabelecidos, demonstram-se insuficientes para capturar a natureza
processual e multidimensional da aprendizagem em PBL. Esta limitação resulta em
lacunas na identificação de progressão de competências, impossibilidade da
detecção precoce de dificuldades de aprendizagem e personalização de
intervenções pedagógicas, diminuindo a eficácia do modelo e a qualidade da
formação profissional.

Paralelamente, o conceito de gêmeo digital tem revolucionado diversos setores
através da criação de representações virtuais dinâmicas que espelham sistemas
físicos ou processos complexos, mantendo sincronização contínua para
monitoramento em tempo real, análise preditiva e otimização automatizada. Esta
capacidade de monitoramento contínuo oferece solução promissora para os
desafios avaliatórios identificados no PBL, especialmente quando aplicada ao
contexto educacional através de taxonomias específicas que contemplam processos
de aprendizagem, perfis estudantis e produtos desenvolvidos.

Este artigo propõe um modelo inovador de gêmeo digital híbrido especificamente
projetado para avaliação de metaprojetos em ambientes PBL. O modelo integra
três tipos de gêmeos digitais - Processo (PDT), Humano (HDT) e Sistema (SDT) -
para capturar multidimensionalmente o desenvolvimento de competências,
progressão de projetos e dinâmicas de colaboração em tempo real. A proposta
fundamenta-se na arquitetura educacional do Inteli e incorpora técnicas
avançadas de MLOps para análise preditiva, personalização de intervenções
pedagógicas e otimização contínua dos processos de ensino-aprendizagem.

\section{Fundamentação}

\subsection{Project-Based Learning e Desafios de Avaliação}

\indent

O Project-Based Learning (PBL) representa metodologia pedagógica consolidada
que organiza o aprendizado em torno de projetos autênticos, promovendo
integração entre conhecimento teórico e aplicação prática através da resolução
de problemas reais. Esta abordagem tem demonstrado particular eficácia na
formação de engenheiros de software por estimular o desenvolvimento simultâneo
de competências técnicas e transversais essenciais para a prática profissional
contemporânea.

A implementação bem-sucedida de metodologias ativas como o PBL requer, contudo,
transformação significativa nas práticas docentes e nos sistemas de avaliação
utilizados. Conforme evidenciado por \cite{FormacaoEducadores2024}, a formação
continuada de educadores em metodologias ativas apresenta tanto potencialidades
quanto desafios substanciais, especialmente na transição de paradigmas
avaliatórios centrados em produtos para abordagens que valorizem processos de
aprendizagem e desenvolvimento de competências.

Contudo, a natureza multifacetada dos projetos em PBL introduz complexidades
avaliatórias ausentes em metodologias tradicionais. Os principais desafios
incluem: (1) \textbf{Temporalidade Assíncrona} - diferentes equipes progridem
em ritmos distintos, dificultando marcos uniformes de avaliação; (2)
\textbf{Multiplicidade de Stakeholders} - envolvimento de parceiros industriais
exige instrumentos que capturam perspectivas externas ao ambiente acadêmico;
(3) \textbf{Competências Emergentes} - habilidades transversais se desenvolvem
de forma não-linear, requerendo instrumentos adaptativos capazes de identificar
progressão em múltiplas dimensões simultaneamente; (4) \textbf{Capacitação
  Docente} - necessidade de formação específica para educadores desenvolverem
competências em avaliação processual e feedback contínuo.

Estas limitações justificam a necessidade de paradigmas avaliatórios que
transcendam a natureza pontual e estática dos métodos convencionais,
direcionando para abordagens de monitoramento contínuo e multidimensional que
capturem adequadamente a complexidade processual da aprendizagem baseada em
projetos e ofereçam suporte tecnológico para a capacitação e práticas docentes.

\subsection{Modelo PBL do Instituto de Tecnologia e Liderança}

\indent

Neste contexto de necessidades avaliatórias complexas, o modelo educacional do
Instituto de Tecnologia e Liderança (Inteli) oferece caso paradigmático de
implementação sistematizada de PBL na engenharia de software. O modelo
estrutura-se através de Learning Backlogs (LBLs), unidades modulares que
integram conhecimento computacional com competências transversais em liderança,
negócios e experiência do usuário, desenvolvidas em colaboração direta com
organizações parceiras que apresentam problemas reais como base para os
projetos estudantis \cite{Valente2025}.

Esta abordagem modular facilita a adaptação curricular conforme demandas
específicas de cada projeto industrial, mantendo aderência às Diretrizes
Curriculares Nacionais (DCN) através de estrutura espiral de aprendizagem onde
conceitos fundamentais são revisitados e aprofundados progressivamente ao longo
de diferentes módulos e projetos. A flexibilidade curricular permite que
estudantes desenvolvam competências "just-in-time", adquirindo conhecimentos
teóricos quando necessários para avançar em desafios práticos específicos.

A arquitetura pedagógica fundamenta-se em cinco dimensões de aprendizagem
derivadas de padrões arquiteturais da ISO/IEC 42010: (1) \textbf{Drivers de
  Negócio e Arquitetura} - compreensão de contexto empresarial e stakeholders;
(2) \textbf{Requisitos Funcionais} - especificação e implementação de
funcionalidades; (3) \textbf{Requisitos Não-Funcionais} - aspectos de
qualidade, performance e usabilidade; (4) \textbf{Visão de Engenharia} -
práticas de desenvolvimento e gestão de projetos; (5) \textbf{Visão de
  Tecnologia} - escolhas e justificativas técnicas. Esta estrutura
multidimensional permite avaliação sistematizada de competências através de
visualizações radar que capturam progressão individual e coletiva em cada uma
das dimensões estabelecidas.

Complementarmente a esta abordagem estruturada, a integração com parceiros
industriais proporciona feedback contínuo sobre a relevância e aplicabilidade
das soluções desenvolvidas, criando ciclo de validação que aproxima o ambiente
acadêmico das demandas reais do mercado de trabalho. Esta configuração, embora
pedagogicamente robusta, amplifica os desafios avaliatórios tradicionais do PBL
pela necessidade de capturar e processar feedback de múltiplos stakeholders com
perspectivas e critérios distintos.

\subsection{Gêmeos Digitais como Solução para Avaliação Contínua}

\indent

Convergindo com esta perspectiva de necessidades avaliatórias complexas, os
gêmeos digitais emergem como paradigma conceitual promissor para superar as
limitações dos métodos tradicionais de avaliação em PBL. O modelo de gêmeo
digital oferece framework estrutural para monitoramento contínuo e
representação virtual dinâmica, proporcionando alternativa sistemática para
capturar a natureza processual e multidimensional da aprendizagem baseada em
projetos.

A taxonomia contemporânea de gêmeos digitais identifica sete categorias
principais: Componente (CDT), Ativo (ADT), Processo (PDT), Sistema (SDT),
Ambiente (EDT), Organização (ODT) e Humano (HDT). Para aplicações educacionais
em PBL, três tipos demonstram particular relevância conceitual: \textbf{PDT}
para representação dos fluxos de aprendizagem e progressão curricular;
\textbf{HDT} para modelagem de perfis estudantis, competências individuais e
dinâmicas colaborativas; \textbf{SDT} para virtualização dos produtos
desenvolvidos e ecossistemas tecnológicos utilizados
\cite{DigitalTwinsEducation2024}.

O foco principal reside no modelo conceitual híbrido que integra estas três
perspectivas, sendo que a implementação técnica pode variar conforme o contexto
institucional e os desafios específicos de cada ambiente PBL. Dependendo das
necessidades particulares, podem ser empregadas diversas ferramentas e
metodologias tecnológicas: plataformas de Learning Management Systems (LMS)
para coleta de dados acadêmicos, sistemas de versionamento de código para
análise de desenvolvimento, ferramentas de Business Intelligence para
visualização de métricas, algoritmos de Machine Learning para análise
preditiva, ou mesmo soluções mais simples baseadas em planilhas e dashboards
para contextos com menor complexidade tecnológica.

A flexibilidade metodológica é fundamental para a aplicabilidade do modelo
proposto em diferentes contextos institucionais. A evolução do continuum
conceitual demonstra progressão desde representações simples até sistemas
sofisticados que podem integrar Internet das Coisas (IoT), inteligência
artificial e tecnologias de visualização avançada conforme a maturidade
tecnológica de cada instituição. Esta adaptabilidade possibilita experimentação
educacional em tempo real através de diversas abordagens de Learning Analytics,
oferecendo insights contínuos sobre eficácia pedagógica, padrões de engajamento
estudantil e indicadores preditivos de performance acadêmica, independentemente
da complexidade tecnológica da implementação escolhida.

\subsection{Abordagens Tecnológicas para Implementação}

\indent

A operacionalização de gêmeos digitais educacionais requer infraestrutura
tecnológica adaptada ao contexto e recursos disponíveis em cada instituição.
Uma das abordagens possíveis envolve a integração com arquiteturas MLOps
(Machine Learning Operations), que oferece framework sistemático para
implementações de alta complexidade com capacidades analíticas avançadas.

A arquitetura de cinco camadas proposta por \cite{Fujii2022} - coleta,
processamento, modelagem, visualização e aplicação - exemplifica uma
implementação robusta que demonstra adequação para contextos educacionais por
permitir: (1) \textbf{coleta multimodal} de dados acadêmicos, comportamentais e
colaborativos; (2) \textbf{processamento inteligente} através de algoritmos de
NLP e computer vision; (3) \textbf{modelagem adaptativa} com técnicas de online
learning; (4) \textbf{visualização contextual} adaptada aos diferentes
stakeholders; (5) \textbf{aplicação personalizada} com recomendações e
intervenções automáticas.

Contudo, o modelo conceitual proposto não está restrito a implementações
complexas. Dependendo dos recursos institucionais e objetivos específicos,
podem ser empregadas abordagens tecnológicas mais simples: sistemas baseados em
planilhas eletrônicas para coleta e análise básica, dashboards web utilizando
ferramentas de Business Intelligence de baixo custo, ou mesmo combinações
híbridas que integrem ferramentas existentes na instituição através de APIs
simples ou exportação/importação de dados.

Esta flexibilidade de implementação possibilita manutenção de bases
individualizadas para cada estudante e projeto, preservando privacidade
conforme requisitos da LGPD, enquanto permite análises que variam desde
relatórios básicos até predições sofisticadas de trajetórias de competências,
sempre adequadas ao nível de maturidade tecnológica institucional.

\section{Modelo de Gêmeo Digital Proposto}

\subsection{Arquitetura Conceitual Híbrida}

\indent

O modelo proposto integra três tipos de gêmeos digitais em arquitetura híbrida:
Gêmeo Digital de Processo (PDT) para representação dos fluxos de aprendizagem,
Gêmeo Digital Humano (HDT) para perfis estudantis e dinâmicas colaborativas, e
Gêmeo Digital de Sistema (SDT) para os produtos desenvolvidos (MVPs). Esta
integração fundamenta-se nas cinco dimensões arquiteturais do modelo Inteli,
proporcionando avaliação holística de competências técnicas e transversais.

\textbf{PDT - Processo de Aprendizagem}: Modela a progressão através dos Learning Backlogs, capturando marcos de aprendizagem, dependências entre módulos e sincronização com demandas de projetos industriais. Incorpora metodologias ágeis adaptadas ao contexto educacional, incluindo sprints de desenvolvimento, retrospectivas de aprendizagem e entregas incrementais.

\textbf{HDT - Perfis e Colaboração}: Representa competências individuais através de visualizações radar multidimensionais, rastreando evolução em cada uma das cinco dimensões arquiteturais. Modela dinâmicas de equipe, padrões de comunicação e contribuições individuais para produtos coletivos, permitindo identificação de lacunas de competências e necessidades de intervenção pedagógica.

\textbf{SDT - Produtos e Sistemas}: Espelha virtualmente os MVPs desenvolvidos, integrando-se com repositórios de código, sistemas de CI/CD e ambientes de deployment. Captura métricas técnicas como qualidade de código, cobertura de testes, performance e aderência a padrões arquiteturais.

\subsection{Arquitetura Conceitual de Implementação}

\indent

A arquitetura conceitual adapta um framework estrutural em camadas para o
contexto educacional, incorporando capacidades específicas para análise de
dados pedagógicos e suporte à tomada de decisão em PBL. Esta estrutura modular
permite implementações tecnológicas diversas conforme recursos disponíveis.

\textbf{Camada de Coleta Multimodal}: Integra dados de múltiplas fontes educacionais, podendo incluir desde sistemas Git para versionamento de código, plataformas LMS para interações acadêmicas, ferramentas de comunicação para colaboração, até APIs de parceiros industriais para feedback de projetos. A implementação pode variar desde coleta manual estruturada até sistemas automatizados com sensores IoT, conforme a complexidade desejada.

\textbf{Camada de Processamento}: Implementa algoritmos de análise adequados ao nível tecnológico institucional, podendo incluir processamento de linguagem natural para documentação técnica, análise estatística de padrões de colaboração, ou avaliações automatizadas de interfaces desenvolvidas. As técnicas podem variar desde análises simples baseadas em métricas diretas até algoritmos de Machine Learning sofisticados.

\textbf{Camada de Modelagem}: Mantém representações individuais para cada estudante, equipe e projeto, atualizadas conforme periodicidade e recursos disponíveis. Pode implementar desde modelos estáticos atualizados periodicamente até sistemas adaptativos com algoritmos de recomendação e predição, dependendo da capacidade analítica institucional.

\textbf{Camada de Visualização}: Oferece interfaces adaptadas para diferentes stakeholders, variando desde dashboards simples em planilhas até sistemas web interativos. Estudantes podem visualizar progresso individual, orientadores podem acessar analytics de equipes, e coordenadores podem monitorar métricas institucionais através de ferramentas adequadas ao contexto específico.

\subsection{Protocolo de Avaliação Multidimensional}

\indent

O sistema implementa avaliação contínua baseada nas cinco dimensões
arquiteturais do Inteli, com pesos adaptativos conforme características
específicas de cada projeto industrial:

\textbf{Drivers de Negócio (25\%)}: Avalia compreensão de contexto empresarial, identificação de problemas reais e proposição de soluções viáveis. Métricas incluem qualidade de análise de stakeholders, aderência a objetivos de negócio e capacidade de tradução de requisitos técnicos.

\textbf{Requisitos Funcionais (20\%)}: Analisa especificação, implementação e validação de funcionalidades. Considera completude de user stories, cobertura de casos de uso e efetividade de testes funcionais.

\textbf{Requisitos Não-Funcionais (20\%)}: Examina aspectos de performance, segurança, usabilidade e manutenibilidade. Incorpora métricas automatizadas de qualidade de código e feedback de usuários finais em projetos industriais.

\textbf{Visão de Engenharia (20\%)}: Avalia práticas de desenvolvimento, metodologias ágeis e gestão de projetos. Monitora frequência de commits, qualidade de documentação técnica e aderência a padrões de desenvolvimento.

\textbf{Visão de Tecnologia (15\%)}: Analisa escolhas tecnológicas, arquitetura de soluções e inovação técnica. Considera adequação de tecnologias ao contexto do problema e capacidade de justificação técnica de decisões arquiteturais.

\section{Implementação e Validação}

\subsection{Prototipagem no Contexto Inteli}

\indent

A implementação inicial do modelo é pensado para ser conduzido no Inteli
aproveitando a infraestrutura existente de Learning Backlogs e parcerias
industriais. O protótipo focará em três projetos piloto de diferentes
complexidades: desenvolvimento de aplicação móvel (baixa complexidade), sistema
web integrado (média complexidade) e solução IoT corporativa (alta
complexidade).

\textbf{Infraestrutura Tecnológica}: O sistema utilizará a stack tecnológica institucional incluindo plataforma Adalove para gestão acadêmica, repositórios GitLab para versionamento, ferramentas Slack/Teams para comunicação, e APIs de parceiros para feedback industrial. A implementação MLOps será baseada em containers Docker com orquestração Kubernetes para escalabilidade.

\textbf{Integração com Stakeholders}: Parceiros industriais participarão através de APIs que fornecem feedback estruturado sobre entregáveis, permitindo calibração contínua dos algoritmos de avaliação. Orientadores terão acesso a dashboards pedagógicos com alertas automáticos para situações que requerem intervenção.

\subsection{Métricas de Validação}

\indent

A validação será conduzida através de Design Science Research com métricas
específicas para cada componente do gêmeo digital:

\textbf{Eficácia Pedagógica}: Comparação de outcomes de aprendizagem entre turmas utilizando o modelo proposto versus métodos tradicionais. Métricas incluem scores nas cinco dimensões arquiteturais, tempo para domínio de competências específicas, e qualidade de produtos finais avaliada por parceiros industriais.

\textbf{Precisão Analítica}: Validação de algoritmos preditivos através de análise de correlação entre predições de risco acadêmico e outcomes reais. Teste de acurácia de recomendações de recursos de aprendizagem através de medição de engajamento e melhoria de performance subsequente.

\textbf{Usabilidade e Adoção}: Avaliação da interface através de heurísticas de usabilidade aplicadas por estudantes e orientadores. Medição de taxa de adoção voluntária em funcionalidades opcionais e feedback qualitativo sobre utilidade percebida.

\textbf{Escalabilidade Técnica}: Testes de carga para validar performance com múltiplos projetos simultâneos. Análise de consumo de recursos computacionais e viabilidade econômica para implementação institucional em larga escala.

\subsection{Validação Comparativa}

\indent

O estudo incluirá grupo de controle utilizando métodos tradicionais de
avaliação PBL, permitindo análise comparativa em múltiplas dimensões:

\textbf{Objetividade de Avaliação}: Medição de variabilidade inter-avaliadores em métodos tradicionais versus consistência algorítmica do gêmeo digital. Análise de bias cognitivos em avaliações humanas comparado à neutralidade de métricas automatizadas.

\textbf{Feedback Contínuo}: Comparação entre feedback pontual tradicional (marcos de entrega) versus insights contínuos do gêmeo digital. Medição de tempo médio para identificação de dificuldades de aprendizagem e efetividade de intervenções pedagógicas.

\textbf{Engajamento Estudantil}: Análise de métricas de participação, colaboração e satisfação entre grupos experimentais. Avaliação de impacto da transparência e personalização do feedback na motivação para aprendizagem contínua.

\section{Contribuições e Considerações Finais}

\subsection{Contribuições Esperadas}

\indent

A proposta apresenta contribuições em múltiplas dimensões da educação em
engenharia de software:

\textbf{Inovação Metodológica}: Primeiro modelo de gêmeo digital híbrido (PDT+HDT+SDT) especificamente projetado para avaliação PBL, integrando análise de processos de aprendizagem, perfis estudantis e produtos desenvolvidos em representação virtual unificada.

\textbf{Framework Técnico}: Adaptação de arquiteturas MLOps para contexto educacional, proporcionando pipeline automatizado de coleta, processamento e análise de dados pedagógicos com capacidades preditivas e de recomendação personalizada.

\textbf{Protocolo de Avaliação}: Sistema de avaliação multidimensional baseado nas cinco dimensões arquiteturais do modelo Inteli, oferecendo alternativa objetiva e contínua aos métodos tradicionais de avaliação pontual em PBL.

\textbf{Validação Empírica}: Metodologia de validação comparativa que permitirá mensuração quantitativa da eficácia pedagógica, precisão analítica e impacto no engajamento estudantil em contexto real de aprendizagem baseada em projetos industriais.

\subsection{Limitações e Desafios}

\indent

A implementação do modelo enfrenta desafios característicos do contexto
latino-americano de adoção de tecnologias emergentes:

\textbf{Infraestrutura Tecnológica}: Dependência de conectividade robusta, capacidade computacional para processamento MLOps e integração com múltiplas plataformas tecnológicas. A implementação requer investimento inicial significativo em infraestrutura e capacitação técnica.

\textbf{Privacidade e Ética}: Coleta contínua de dados estudantis levanta questões sobre privacidade, consentimento e uso ético de informações acadêmicas. O modelo requer desenvolvimento de protocolos específicos para proteção de dados pessoais conforme LGPD.

\textbf{Resistência Institucional}: Mudança paradigmática na avaliação pode encontrar resistência de stakeholders acostumados a métodos tradicionais. Implementação requer estratégia de gestão de mudança e treinamento extensivo.

\subsection{Perspectivas Futuras}

\indent

A pesquisa abre múltiplas direções para desenvolvimento futuro:

\textbf{Expansão Disciplinar}: Adaptação do modelo para outras engenharias e áreas STEM, aproveitando a estrutura modular de Learning Backlogs para diferentes contextos curriculares.

\textbf{Integração Tecnológica}: Incorporação de realidade virtual/aumentada para visualização imersiva de gêmeos digitais, e blockchain para certificação descentralizada de competências adquiridas.

\textbf{Análise Longitudinal}: Acompanhamento de egressos para avaliação de impacto da metodologia na performance profissional e desenvolvimento de carreira em longo prazo.

\textbf{Rede Colaborativa}: Desenvolvimento de consórcio inter-institucional para compartilhamento de melhores práticas e criação de benchmarks de avaliação PBL baseada em gêmeos digitais.

A proposta representa avanço significativo na convergência entre tecnologias
educacionais emergentes e metodologias ativas de aprendizagem, posicionando
instituições educacionais brasileiras na vanguarda da inovação pedagógica
global através da aplicação contextualizada de gêmeos digitais para otimização
de processos de ensino-aprendizagem em engenharia de software.

\section*{Acknowledgements}
O autor agradece ao Instituto de Tecnologia e Liderança (Inteli) pelo apoio na realização desta pesquisa e aos estudantes e professores que participarão da validação do modelo proposto.

%====================================================================

%\printbibliography

\begin{thebibliography}{99}

  \bibitem[Valente et al.(2025)]{Valente2025}
  Valente, J. A., Bittencourt, I. I., Santoro, F. M., Garcia, M., Isotani, S., Garcia, A. \& Habimorad, M. (2025). Método para Revisão de Currículo de Engenharia de Software baseado em Learning Backlogs. In: \textit{Anais do Congresso Brasileiro de Educação em Engenharia - COBENGE}, São Paulo, SP.

  \bibitem[Silva et al.(2024)]{FormacaoEducadores2024}
  Silva, M. A., Santos, P. R., Oliveira, L. C. \& Lima, R. F. (2024). Potencialidades e Desafios na Formação Continuada de Educadores em Metodologias Ativas. \textit{Revista Brasileira de Informática na Educação}, 32(4), 112--135.

  \bibitem[Digital Twins Education(2024)]{DigitalTwinsEducation2024}
  Digital Twins for Education: A Comprehensive Review of Applications in STEAM Learning. (2024). \textit{Journal of Educational Technology Research}, 45(3), 156--189.

  \bibitem[Fujii et al.(2022)]{Fujii2022}
  Fujii, T.Y.; Hayashi, V.T.; Arakaki, R.; Ruggiero, W.V.; Bulla, R., Jr.; Hayashi, F.H.; Khalil, K.A. (2022). A Digital Twin Architecture Model Applied with MLOps Techniques to Improve Short-Term Energy Consumption Prediction. \textit{Machines}, 10, 23.

  \bibitem[Aplicações DT América Latina(2024)]{AplicacoesDTLatAm2024}
  Aplicações de Gêmeos Digitais na América Latina: Estado da Arte e Perspectivas Futuras. (2024). \textit{Revista Latino-Americana de Tecnologia Educacional}, 12(2), 45--67.

\end{thebibliography}

%====================================================================

\end{document}
