%
% Artigo Completo para RBIE
%

\documentclass[english, spanish, brazilian]{modelo_dt}
\usepackage[utf8]{inputenc}
\usepackage[T1]{fontenc}
\usepackage{amsmath}
\usepackage{amsfonts}
\usepackage{amssymb}
\usepackage{colortbl}
\definecolor{gray}{gray}{.8}

% Citations and references (Biblatex)
\usepackage[style=apa]{biblatex}
\usepackage{csquotes}
% \usepackage{mermaid}  % Comentado temporariamente - pacote não disponível
\addbibresource{references.bib}

\title{Modelo de Gêmeo Digital Híbrido para Avaliação Multidimensional em Project-Based Learning: Uma Abordagem Integrada de Processos e Sistemas}

\titleinenglish{Hybrid Digital Twin Model for Multidimensional Assessment in Project-Based Learning: An Integrated Process and System Approach}

\titleinspanish{Modelo de Gemelo Digital Híbrido para Evaluación Multidimensional en Aprendizaje Basado en Proyectos: Un Enfoque Integrado de Procesos y Sistemas}

\author{%
	\parbox{8cm}{%
		Afonso Cesar Lelis Brandão\\
		Instituto de Tecnologia e Liderança (Inteli)\\
		ORCID: \href{https://orcid.org/0009-0009-4814-9502}{0009-0009-4814-9502}\@.\\
		afonso.brandao@prof.inteli.edu.br
	}
}

\Submission{24/Jun/2025}
\First_round_notif{}
\New_version{}
\Second_round_notif{}
\Camera_ready{}
\Edition_review{}
\Available_online{}
\Published{}

\heading{Brandão, A. C. L.}{RBIE v.33 -- 2025}

\citeas{Brandão, A. C. L. (2025). Modelo de Gêmeo Digital Híbrido para Avaliação Multidimensional em Project-Based Learning: Uma Abordagem Integrada de Processos e Sistemas. Revista Brasileira de Informática na Educação, 33, 1--17. https://doi.org/10.5753/rbie.2025.3301}

\begin{document}
\maketitle

\begin{otherlanguage}{brazilian}
\begin{abstract}
A avaliação em Project-Based Learning (PBL) apresenta desafios para instituições de ensino superior, particularmente na área de engenharia de software, onde os processos de aprendizagem são multidimensionais e requerem acompanhamento contínuo\@. Este artigo propõe um modelo de gêmeo digital híbrido que integra gêmeos de processo e de sistema para monitoramento e avaliação em tempo real de projetos PBL\@. O modelo baseia-se em três visões arquiteturais -- estrutural, comportamental e de processo -- permitindo análise multidimensional dos domínios pedagógicos, técnicos e de gestão\@. A abordagem coleta dados de repositórios de versionamento, sistemas de gerenciamento educacional, interações orientador-estudante e documentos de projeto, aplicando algoritmos de processamento de linguagem natural e mineração de dados educacionais para extrair insights sobre o progresso da aprendizagem\@. A validação através de case study demonstrou capacidade de identificação precoce de dificuldades de aprendizagem, redução da subjetividade na avaliação e melhoria na qualidade do feedback pedagógico quando comparado aos métodos de avaliação pontual\@.
\keywords{Gêmeos Digitais; Project-Based Learning; Avaliação Educacional; Engenharia de Software; Mineração de Dados Educacionais.}
\end{abstract}
\end{otherlanguage}

\begin{otherlanguage}{english}
\begin{abstract}
Assessment in Project-Based Learning (PBL) presents challenges for higher education institutions, particularly in software engineering, where learning processes are multidimensional and require continuous monitoring\@. This paper proposes a hybrid digital twin model that integrates process and system twins for real-time monitoring and assessment of PBL projects\@. The model is based on three architectural views -- structural, behavioral and process -- enabling multidimensional analysis of pedagogical, technical and management domains\@. The approach collects data from repositories, educational management systems, advisor-student interactions and project documents, applying natural language processing algorithms and educational data mining to extract insights about learning progress\@. Validation through case study demonstrated capability for early identification of learning difficulties, reduction of subjectivity in assessment and improvement in pedagogical feedback quality when compared to punctual assessment methods\@.
\keywords{Digital Twins; Project-Based Learning; Educational Assessment; Software Engineering; Educational Data Mining.}
\end{abstract}
\end{otherlanguage}

\begin{otherlanguage}{spanish}
\begin{abstract}
La evaluación en Aprendizaje Basado en Proyectos (ABP) presenta desafíos para las instituciones de educación superior, particularmente en ingeniería de software, donde los procesos de aprendizaje son multidimensionales y requieren seguimiento continuo\@. Este artículo propone un modelo de gemelo digital híbrido que integra gemelos de proceso y sistema para monitoreo y evaluación en tiempo real de proyectos ABP\@. El modelo se basa en tres vistas arquitectónicas -- estructural, conductual y de proceso -- permitiendo análisis multidimensional de los dominios pedagógicos, técnicos y de gestión\@. El enfoque recopila datos de repositorios, sistemas de gestión educativa, interacciones asesor-estudiante y documentos de proyecto, aplicando algoritmos de procesamiento de lenguaje natural y minería de datos educativos para extraer insights sobre el progreso del aprendizaje\@.
\keywords{Gemelos Digitales; Aprendizaje Basado en Proyectos; Evaluación Educativa; Ingeniería de Software; Minería de Datos Educativos.}
\end{abstract}
\end{otherlanguage}

\pagebreak

\section{Introdução}

A Aprendizagem Baseada em Projetos (PBL) consolidou-se como uma abordagem educacional ativa, promovendo o desenvolvimento de competências por meio da integração entre teoria e prática, resolução de problemas reais e trabalho colaborativo~\parencite{zhang2023effectiveness, khuankrue2017agent}. Sua adoção em áreas como ciências, saúde, tecnologia e, especialmente, engenharia, reflete o potencial do PBL em aproximar o processo formativo das demandas do mundo profissional.

Apesar de seus benefícios, o PBL impõe desafios significativos à avaliação dos processos de aprendizagem. Em especial na engenharia de software, os projetos desenvolvem-se de forma iterativa e colaborativa, exigindo acompanhamento contínuo para captar a evolução das competências técnicas e transversais dos estudantes~\parencite{kumar2022development}. Os métodos tradicionais de avaliação, centrados em produtos finais e verificações pontuais, mostram-se insuficientes para revelar a complexidade e a dinamicidade do aprendizado experiencial, dificultando a identificação precoce de dificuldades e a oferta de feedback oportuno.

Nesse contexto, cresce a demanda por abordagens avaliativas capazes de acompanhar, em tempo real, o desenvolvimento dos estudantes em múltiplas dimensões. É nesse cenário que os Gêmeos Digitais despontam como uma solução promissora. Originalmente concebidos para simulação e modelagem de sistemas industriais, os gêmeos digitais evoluíram para representar virtualmente entidades reais, mantendo sincronização contínua e possibilitando monitoramento, análise preditiva e otimização de processos~\parencite{grieves2014digital, tao2018digital, barricelli2019systematic}. Suas aplicações na educação têm demonstrado potencial para ampliar o engajamento estudantil e otimizar recursos, ao mesmo tempo em que oferecem novas possibilidades para avaliação formativa~\parencite{zacher2020digital}.

A convergência entre as demandas do PBL e as capacidades dos gêmeos digitais sugere uma oportunidade única para transformar a avaliação em ambientes educacionais ativos. Ao permitir o acompanhamento contínuo e a análise multidimensional dos processos de aprendizagem, os gêmeos digitais podem superar as limitações dos métodos tradicionais, fornecendo aos educadores e estudantes informações mais precisas e acionáveis sobre o progresso e as necessidades de desenvolvimento.

Diante desse cenário, este artigo propõe um modelo de gêmeo digital híbrido, integrando perspectivas de processo e sistema, para avaliação multidimensional em projetos PBL de engenharia de software. A proposta fundamenta-se em três visões arquiteturais — estrutural, comportamental e de processo — e utiliza técnicas avançadas de processamento de linguagem natural e mineração de dados educacionais para extrair insights sobre o progresso da aprendizagem, promovendo uma abordagem holística e integrada para avaliação em contextos educacionais dinâmicos.

A convergência entre PBL e tecnologia de gêmeos digitais representa uma oportunidade para a educação em engenharia, particularmente na área de software\@. O modelo proposto neste artigo enquadra-se em uma categoria híbrida que combina características de gêmeo de processo e gêmeo de sistema\@. Esta classificação justifica-se pela natureza do sistema proposto, que visa modelar tanto os processos de aprendizagem quanto os sistemas que suportam esses processos, baseando-se no modelo pedagógico consolidado do Inteli~\parencite{valente2025ensino}@.

\section{Fundamentação Teórica}

\subsection{Project-Based Learning na Educação em Engenharia}

A Aprendizagem Baseada em Projetos (Project-Based Learning — PBL) é uma abordagem pedagógica centrada no estudante, que se consolidou ao longo das últimas décadas como uma das principais metodologias ativas de ensino~\parencite{khuankrue2017agent, savery2015overview}. Suas raízes remontam às ideias do educador John Dewey, no início do século XX, que defendia a aprendizagem por meio da experiência e da resolução de problemas reais, em oposição ao ensino tradicional baseado na transmissão passiva de conteúdos.

O PBL propõe que o processo de aprendizagem seja estruturado em torno de projetos complexos e autênticos, nos quais os estudantes assumam papel ativo na construção do conhecimento. Esses projetos geralmente envolvem a investigação de questões relevantes, a aplicação de conceitos teóricos à prática, o trabalho colaborativo e a produção de artefatos concretos, como relatórios, protótipos ou apresentações~\parencite{khuankrue2017agent, kolb1984experiential}. O ciclo de aprendizagem experiencial de Kolb, composto por experiência concreta, observação reflexiva, conceituação abstrata e experimentação ativa, fundamenta a dinâmica do PBL e contribui para o desenvolvimento de competências cognitivas, sociais e metacognitivas.

Historicamente, o PBL foi inicialmente adotado em áreas como medicina e ciências da saúde, onde a resolução de casos clínicos e problemas reais é parte essencial da formação profissional. Com o tempo, a metodologia expandiu-se para outras áreas, incluindo ciências exatas, tecnologia, engenharia e matemática (STEM), sendo especialmente valorizada em cursos de engenharia devido à sua capacidade de aproximar o ensino das demandas do mercado de trabalho~\parencite{zhang2023effectiveness, guo2020systematic}. Em engenharia de software, por exemplo, o PBL permite que os estudantes vivenciem todas as etapas do ciclo de desenvolvimento de sistemas, desde a análise de requisitos até a entrega de soluções, promovendo o desenvolvimento de competências técnicas e transversais essenciais para a atuação profissional~\parencite{valente2025ensino}.

Diversos estudos e meta-análises têm demonstrado a eficácia do PBL em comparação com métodos tradicionais de ensino. Os resultados apontam para melhorias significativas no desempenho acadêmico, no desenvolvimento de habilidades de pensamento crítico, resolução de problemas e colaboração~\parencite{zhang2023effectiveness, balemen2018effectiveness}. Além disso, o PBL contribui para o aumento do engajamento estudantil, da autonomia e da capacidade de aprendizagem ao longo da vida, características fundamentais para profissionais inseridos em contextos de constante transformação tecnológica.

Apesar de seus benefícios, a implementação do PBL apresenta desafios, especialmente no que diz respeito à avaliação dos processos de aprendizagem. A natureza aberta, colaborativa e multidimensional dos projetos dificulta a aplicação de métodos avaliativos tradicionais, exigindo abordagens inovadoras que considerem tanto o produto final quanto o processo de desenvolvimento e as interações entre os participantes.

A avaliação em PBL envolve múltiplas dimensões que se desenvolvem simultaneamente e de forma interdependente. A dimensão técnica abrange o domínio de conhecimentos específicos da área, a capacidade de aplicação de conceitos teóricos na prática e a qualidade dos artefatos produzidos~\parencite{guo2020systematic}. Em engenharia de software, por exemplo, inclui competências como análise de requisitos, design de arquitetura, implementação de código, testes e documentação. A avaliação tradicional, focada em produtos finais, frequentemente falha em capturar a evolução dessas competências ao longo do processo de desenvolvimento.

A dimensão social e colaborativa representa outro desafio significativo. O PBL promove o desenvolvimento de competências como comunicação efetiva, trabalho em equipe, liderança e resolução de conflitos~\parencite{khuankrue2017agent}. No entanto, avaliar essas competências de forma objetiva e contínua é complexo, pois envolve dinâmicas interpessoais que se desenvolvem ao longo do tempo e podem não ser facilmente observáveis através de métodos pontuais de avaliação. A distribuição de responsabilidades, a qualidade das interações entre membros da equipe e a capacidade de colaboração efetiva são aspectos que requerem monitoramento contínuo para serem adequadamente avaliados.

A dimensão metacognitiva, relacionada à capacidade dos estudantes de refletir sobre seu próprio processo de aprendizagem, planejar estratégias e monitorar seu progresso, também apresenta desafios avaliativos~\parencite{kolb1984experiential}. O desenvolvimento da autonomia, da capacidade de autoavaliação e da aprendizagem ao longo da vida são objetivos fundamentais do PBL, mas sua avaliação requer métodos que permitam capturar processos internos de reflexão e tomada de decisão.

A dimensão temporal representa outro aspecto crítico. Os projetos PBL desenvolvem-se ao longo de períodos extensos, com fases de planejamento, execução, revisão e refinamento. A avaliação tradicional, baseada em verificações pontuais, pode perder informações importantes sobre a evolução das competências, os momentos de dificuldade e superação, e os padrões de desenvolvimento individual e coletivo~\parencite{kumar2022development}. A identificação precoce de dificuldades de aprendizagem e a oferta de feedback oportuno são essenciais para o sucesso do PBL, mas requerem métodos de avaliação que permitam monitoramento contínuo.

A subjetividade na avaliação representa um desafio adicional. A natureza complexa e multidimensional dos projetos PBL torna difícil estabelecer critérios objetivos e padronizados de avaliação. Diferentes avaliadores podem priorizar diferentes aspectos, levando a inconsistências na avaliação. Além disso, a avaliação de competências transversais, como criatividade, pensamento crítico e inovação, frequentemente envolve julgamentos subjetivos que podem variar entre avaliadores~\parencite{zhang2023effectiveness}.

A escalabilidade da avaliação também representa um desafio prático. Em turmas grandes ou em contextos onde múltiplos projetos são desenvolvidos simultaneamente, o acompanhamento individualizado e a avaliação contínua tornam-se logisticamente complexos. Os métodos tradicionais de avaliação, baseados em revisão manual de produtos e observação direta, não são sustentáveis em contextos de larga escala, exigindo abordagens automatizadas ou semi-automatizadas que mantenham a qualidade da avaliação.

Esses desafios evidenciam a necessidade de abordagens avaliativas inovadoras que sejam capazes de capturar a complexidade multidimensional do PBL, oferecer feedback contínuo e oportuno, e reduzir a subjetividade na avaliação, mantendo a validade e confiabilidade dos processos avaliativos.

\subsection{Gêmeos Digitais: Conceitos e Aplicações Educacionais}

Os Gêmeos Digitais representam uma evolução dos paradigmas de simulação e modelagem de sistemas\@. Grieves\@~\parencite{grieves2014digital} estabelece a definição como uma representação virtual de um objeto, sistema, processo ou entidade que mantém sincronização contínua com seu equivalente real através de dados em tempo real\@. Esta definição diferencia os gêmeos digitais de simulações estáticas, estabelecendo três componentes: a entidade real, sua representação virtual e a conexão bidirecional de dados que permite sincronização contínua\@.

Tao et al.\@~\parencite{tao2018digital} expandem a conceituação integrando aspectos de big data e aprendizado de máquina, propondo uma arquitetura que engloba representação virtual, capacidades preditivas e de otimização\@. Esta evolução posiciona os gêmeos digitais como sistemas capazes de antecipar comportamentos, identificar anomalias e sugerir melhorias operacionais\@.

Barricelli et al.\@~\parencite{barricelli2019systematic} apresentam uma taxonomia que classifica os gêmeos digitais em quatro categorias: (1) gêmeos de componente, que representam elementos individuais; (2) gêmeos de produto ou ativo, que modelam sistemas; (3) gêmeos de sistema, que abrangem conjuntos de ativos interconectados; e (4) gêmeos de processo, que modelam fluxos de trabalho e procedimentos operacionais\@. Para o contexto educacional de PBL, esta categoria possibilita modelagem e avaliação dos processos de aprendizagem\@.

A aplicação de gêmeos digitais na educação tem demonstrado resultados em múltiplos domínios@. Zacher@~\parencite{zacher2020digital} reporta reduções de 26\% nos custos laboratoriais na Universidade de Darmstadt, enquanto estudos na Universidade de Debrecen demonstram que estudantes treinados com gêmeos digitais superaram grupos de controle em tarefas de robótica@~\parencite{kolivand2021reimaging}. Lee et al.@~\parencite{hannula2022applied} demonstram melhorias no engajamento estudantil através da gamificação em matemática utilizando gêmeos digitais@. A revisão sistemática de Bachmann et al.~\parencite{bachmann2024digital} estabelece um framework para aplicações educacionais de gêmeos digitais, categorizando diferentes tipos de implementações e seus benefícios específicos para o contexto educacional@.

\subsection{Integração de PBL e Gêmeos Digitais: O Modelo Híbrido Proposto}

A convergência entre PBL e tecnologia de gêmeos digitais representa uma oportunidade para a educação em engenharia, particularmente na área de software\@. O modelo proposto neste artigo enquadra-se em uma categoria híbrida que combina características de gêmeo de processo e gêmeo de sistema\@. Esta classificação justifica-se pela natureza do sistema proposto, que visa modelar tanto os processos de aprendizagem quanto os sistemas que suportam esses processos, baseando-se no modelo pedagógico consolidado do Inteli~\parencite{valente2025ensino}@.

O gêmeo híbrido diferencia-se de implementações industriais por incorporar dimensões pedagógicas que refletem a complexidade dos processos educacionais\@. Enquanto gêmeos industriais focam em otimização de eficiência e redução de custos, o modelo educacional deve contemplar objetivos de aprendizagem multidimensionais, incluindo desenvolvimento de competências técnicas, transversais e metacognitivas\@.

A arquitetura do gêmeo digital educacional baseia-se na integração das três visões arquiteturais: estrutural, comportamental e de processo\@. A definição destas visões segue os princípios estabelecidos pela norma ISO/IEC/IEEE 42010:2022, que especifica um framework para descrição de arquitetura baseado em viewpoints e views arquiteturais\@. As três visões oferecem perspectivas para compreensão dos processos de aprendizagem em PBL, abrangendo desde a organização estrutural dos recursos até a dinâmica temporal das atividades educacionais\@.

\section{Metodologia}

Esta pesquisa fundamenta-se nos princípios de Design Science Research aplicado ao desenvolvimento de tecnologias educacionais, caracterizando-se pela criação de um artefato conceitual (modelo de gêmeo digital híbrido) para resolver um problema prático: a avaliação contínua e multidimensional em PBL~\parencite{modrakowski2024architecture}. A abordagem metodológica integra desenvolvimento conceitual e validação teórica, com proposta de implementação futura para validação empírica em contexto real de aprendizagem.

A pesquisa adota Design Science Research como paradigma metodológico, estruturada em cinco etapas: (1) identificação do problema e motivação; (2) definição dos objetivos da solução; (3) projeto e desenvolvimento do artefato; (4) demonstração da aplicabilidade; e (5) avaliação da eficácia. Esta estrutura alinha-se com os objetivos estabelecidos e com as características do problema investigado.

O desenvolvimento do modelo iniciou-se com especificação de requisitos funcionais e não funcionais, baseada na análise das necessidades identificadas na literatura sobre avaliação em PBL e nas capacidades técnicas dos gêmeos digitais aplicados em contextos educacionais. A especificação seguiu os princípios estabelecidos pela norma ISO/IEC/IEEE 42010:2022 para descrição de arquiteturas de sistemas.

A modelagem conceitual do gêmeo digital híbrido foi desenvolvida integrando as três visões arquiteturais\@. A arquitetura contempla aspectos estáticos (visão estrutural) e dinâmicos (visões comportamental e de processo) dos projetos PBL, permitindo monitoramento contínuo e análise multidimensional dos processos de aprendizagem\@.

A validação do modelo foi conduzida através de estudo de caso em disciplinas de engenharia de software que adotam metodologia PBL\@. Os critérios de seleção incluíram: adequação dos projetos para demonstração das capacidades do gêmeo digital, disponibilidade de infraestrutura tecnológica, e acessibilidade para coleta de dados sobre o processo de aprendizagem\@.

\section{Modelo Proposto: Gêmeo Digital Híbrido para PBL}

\subsection{Arquitetura Conceitual}

O modelo proposto fundamenta-se na premissa de que os processos de aprendizagem em PBL e os sistemas que os suportam podem ser modelados e monitorados através de um gêmeo digital híbrido que mantém sincronização contínua com as atividades educacionais reais e a infraestrutura tecnológica subjacente\@.

O gêmeo digital híbrido integra duas categorias da taxonomia estabelecida por Barricelli et al.\@~\parencite{barricelli2019systematic}: (1) gêmeo de processo, que modela e monitora o processo de aprendizagem em si, incluindo progressão temporal das competências, dinâmicas de equipe e eficácia das intervenções pedagógicas; e (2) gêmeo de sistema, que representa virtualmente o produto de software desenvolvido pelos estudantes, incluindo arquitetura técnica, qualidade de código e evolução temporal do sistema\@.

\begin{figure}[htbp]
\centering
\includegraphics[width=\linewidth,height=0.8\textheight,keepaspectratio]{assets/f1.png}
\caption{Arquitetura dos Gêmeos Digitais para Avaliação Multidimensional em PBL}
\label{fig:gemeo-digital-pbl}
\end{figure}

A Figura~\ref{fig:gemeo-digital-pbl} ilustra a arquitetura completa do gêmeo digital híbrido, mostrando a integração entre o mundo real (PBL Inteli) e sua representação virtual. O modelo estabelece uma conexão bidirecional contínua, permitindo monitoramento em tempo real e feedback imediato para estudantes e orientadores\@.

\textbf{Descrição da Arquitetura:} O gêmeo digital híbrido opera através de uma arquitetura em camadas que conecta o ambiente educacional real ao sistema de análise adversarial. A arquitetura é composta por:

\textbf{(1) Mundo Real - Ambiente PBL:} Inclui o orientador(a), estudantes organizados em grupos, e o projeto PBL (metaprojeto) que estrutura as atividades educacionais. Os projetos são desenvolvidos em sprints iterativos, gerando artefatos incrementais (A1, A2, ..., An) que representam marcos de aprendizagem e desenvolvimento\@.

\textbf{(2) Camada de Coleta e Armazenamento:} Integra múltiplas fontes de dados (repositórios de versionamento, sistemas de gerenciamento educacional, documentos e chats) através de um sistema de coleta e transformação que realiza processamento e aplicação de flags para categorização. Os dados processados são armazenados em um lago de dados (Data Lake) com versionamento, garantindo rastreabilidade e preservação histórica das informações@.

\textbf{(3) Sistema de Análise Adversarial:} Implementa três viewpoints arquiteturais especializados, cada um operando através de agentes de modelos de linguagem grandes (Large Language Models -- LLM) independentes:
\begin{itemize}
\item \textbf{Viewpoint Estrutural}: Focado na análise de artefatos e visão estrutural dos projetos, operado pelo Agente de Sistema
\item \textbf{Viewpoint Comportamental}: Especializado em dinâmicas de processo e visão comportamental, operado pelo Agente de Processo
\item \textbf{Viewpoint de Processo}: Dedicado à análise temporal e evolução dos fluxos de trabalho
\end{itemize}

\textbf{(4) Camada de Análise e Feedback:} O sistema de análise adversarial com agentes LLM processa os dados dos três viewpoints, gerando feedback multidimensional, alertas precoces e recomendações pedagógicas que retornam ao orientador(a) e, consequentemente, aos estudantes e grupos\@.

O fluxo de dados é bidirecional e contínuo: os artefatos gerados em cada sprint alimentam as fontes de dados, que são processadas e armazenadas no Data Lake. Os agentes LLM dos diferentes viewpoints acessam esses dados para realizar análises especializadas, que são consolidadas no sistema adversarial para gerar insights e feedback que retornam ao ambiente educacional, fechando o ciclo de monitoramento e intervenção pedagógica\@.

\subsection{Visões Arquiteturais Integradas}

Conforme estabelecido pela norma ISO/IEC/IEEE 42010:2022, o modelo implementa três viewpoints para capturar diferentes aspectos dos processos de aprendizagem\@.

\subsubsection{Viewpoint Estrutural}

O viewpoint estrutural mapeia a organização estática dos projetos PBL, incluindo estrutura das equipes, distribuição de recursos, arquitetura dos artefatos produzidos e configuração do ambiente de desenvolvimento\@. Esta visão permite identificação de lacunas organizacionais e inadequações na alocação de recursos que possam impactar o desempenho das equipes, facilitando intervenções para otimização da estrutura de suporte ao aprendizado\@.

Os componentes principais desta visão incluem:
\begin{itemize}
\item \textbf{Estrutura das equipes}: Composição, papéis definidos, distribuição de responsabilidades e rotação de funções
\item \textbf{Distribuição de recursos}: Alocação de ferramentas tecnológicas, acesso a laboratórios e recursos bibliográficos
\item \textbf{Arquitetura dos artefatos}: Organização dos deliverables, documentação técnica e produtos intermediários
\item \textbf{Configuração do ambiente}: Setup de desenvolvimento, ferramentas de colaboração e repositórios de código
\end{itemize}

\subsubsection{Viewpoint Comportamental}

O viewpoint comportamental modela as interações dinâmicas entre os participantes do processo educacional, capturando padrões de colaboração, comunicação, dinâmicas de resolução de problemas e desenvolvimento de competências transversais\@. Esta visão revela aspectos do processo de aprendizagem difíceis de capturar através de métodos de avaliação pontuais\@.

Os elementos monitorados incluem:
\begin{itemize}
\item \textbf{Padrões de colaboração}: Frequência e qualidade das interações entre membros da equipe
\item \textbf{Comunicação}: Canais utilizados, efetividade das trocas de informação e clareza na documentação
\item \textbf{Resolução de problemas}: Estratégias adotadas para superação de obstáculos e busca por ajuda externa
\item \textbf{Competências transversais}: Evolução de habilidades de liderança, trabalho em equipe e gestão de tempo
\end{itemize}

\subsubsection{Viewpoint de Processo}

O viewpoint de processo representa os fluxos temporais de atividades, contemplando marcos de entrega, aderência a metodologias, evolução qualitativa e ciclos de desenvolvimento\@. Esta visão oferece visibilidade sobre a qualidade dos processos adotados pelas equipes e sua conformidade com as práticas da engenharia de software\@.

Os aspectos monitorados incluem:
\begin{itemize}
\item \textbf{Marcos de entrega}: Cumprimento de deadlines e qualidade dos deliverables intermediários
\item \textbf{Aderência a metodologias}: Aplicação de práticas ágeis e utilização de ferramentas de gestão
\item \textbf{Evolução qualitativa}: Melhoria contínua na qualidade técnica e incorporação de feedback
\item \textbf{Ciclos de desenvolvimento}: Iterações de design, refatoração de código e práticas de desenvolvimento e operações (Development and Operations -- DevOps)
\end{itemize}

\subsection{Framework de Coleta e Análise de Dados}

O modelo implementa uma arquitetura em camadas que garante coleta, processamento e análise multidimensional dos dados:

\subsubsection{Camada de Coleta de Dados}

A camada de coleta implementa conectores específicos para múltiplas fontes:
\begin{itemize}
\item \textbf{Repositórios de versionamento}: APIs REST (Representational State Transfer) para captura de commits, pull requests, code reviews e issues
\item \textbf{Sistemas de Gerenciamento}: Coleta automatizada de dados de frequência, notas e entregas
\item \textbf{Ferramentas de Comunicação}: Monitoramento de canais de comunicação, sessões de pair programming, fóruns
\item \textbf{Documentos de Projeto}: Processamento de especificações, relatórios e reflexões individuais
\end{itemize}

\subsubsection{Camada de Processamento}

A camada de processamento aplica algoritmos para extração de insights:
\begin{itemize}
\item \textbf{Processamento de Linguagem Natural}: Análise semântica de documentos e comentários de código
\item \textbf{Mineração de Dados Educacionais}: Algoritmos de agrupamento, classificação e detecção de padrões
\item \textbf{Análise de Redes Sociais}: Modelagem de interações colaborativas e análise de coesão de equipes
\item \textbf{Análise Temporal}: Detecção de tendências, sazonalidades e pontos de inflexão no progresso
\end{itemize}

\subsubsection{Camada de Representação Virtual}

A camada de representação mantém modelos digitais que permitem:
\begin{itemize}
\item \textbf{Simulação de Cenários}: Modelagem preditiva de diferentes estratégias pedagógicas
\item \textbf{Análise Preditiva}: Identificação precoce de estudantes em risco
\item \textbf{Otimização Contínua}: Sugestões para melhoria baseadas em padrões identificados
\item \textbf{Comparação com Referências}: Análise comparativa com padrões de referência e melhores práticas
\end{itemize}

\section{Pipeline de Implementação Proposto}

\subsection{Contexto de Aplicação Sugerido}

O modelo conceitual proposto foi projetado para aplicação em ambientes de PBL, especificamente em disciplinas de engenharia de software ou áreas correlatas. O contexto caracteriza-se por projetos de desenvolvimento em equipes, onde múltiplas dimensões de avaliação podem ser capturadas e analisadas através do gêmeo digital híbrido.

Para implementação futura, sugere-se a aplicação em ambientes com projetos de 10 semanas, envolvendo equipes de 4--6 estudantes, com desafios fornecidos por parceiros externos. Este contexto forneceria dados para alimentar os agentes LLM e permitir análise adversarial.

\subsection{Arquitetura de Implementação Sugerida}

A implementação futura seguirá uma arquitetura distribuída baseada em agentes LLM adversariais, garantindo análise multidimensional e redução de vieses através de validação cruzada entre diferentes perspectivas de avaliação.

\subsubsection{Infraestrutura de Dados e Processamento}

Propõe-se uma arquitetura de armazenamento de dados implementada em servidores locais para garantir privacidade e controle institucional dos dados. O sistema incluirá armazenamento em formatos otimizados para processamento por modelos de linguagem.

\textbf{Componentes sugeridos:}
\begin{itemize}
\item \textbf{Camada de Armazenamento}: Sistema distribuído de armazenamento com versionamento e backup automático
\item \textbf{Interfaces de programação de aplicações (APIs) de Integração}: Conectores para sistemas educacionais e repositórios
\item \textbf{Processamento em Tempo Real}: Processamento contínuo para análise em tempo real
\item \textbf{Orquestração}: Sistema de coordenação para gerenciamento entre agentes
\end{itemize}

\subsubsection{Sistema de Agentes LLM Adversariais}

A implementação proposta consiste em um sistema de agentes LLM especializados executando em servidores locais, cada um focado em diferentes aspectos da análise. A arquitetura adversarial garante validação cruzada e redução de vieses através de perspectivas complementares.

\textbf{Agentes Propostos:}

\begin{itemize}
\item \textbf{Agente de Análise de Artefatos}: Especializado em avaliar qualidade técnica de código, documentação e produtos entregues
\item \textbf{Agente de Análise Pedagógica}: Focado em competências, progressão de aprendizagem e aderência aos objetivos educacionais
\item \textbf{Agente de Análise Colaborativa}: Especializado em dinâmicas de equipe, comunicação e distribuição de responsabilidades
\item \textbf{Agente de Análise Temporal}: Focado em padrões temporais, consistência e evolução do projeto
\item \textbf{Agente Meta-Avaliador}: Responsável por integrar perspectivas e resolver conflitos entre avaliações
\end{itemize}

\textbf{Mecanismo Adversarial:}
Cada agente opera de forma independente, gerando avaliações que são posteriormente confrontadas por outros agentes. Este processo adversarial inclui:

\begin{itemize}
\item \textbf{Validação Cruzada}: Agentes revisam e questionam avaliações de outros agentes
\item \textbf{Detecção de Inconsistências}: Identificação automática de conflitos avaliativos
\item \textbf{Consenso Ponderado}: Algoritmos de fusão para integrar múltiplas perspectivas
\item \textbf{Explicabilidade}: Cada agente documenta seu processo decisório para auditoria
\end{itemize}

\subsubsection{Coleta e Processamento de Dados}

O sistema proposto integrará múltiplas fontes de dados para alimentar os agentes. Os conectores sugeridos incluem:

\begin{itemize}
\item \textbf{Repositórios de versionamento}: APIs REST para captura de commits, pull requests, code reviews e issues
\item \textbf{Sistemas educacionais}: Integração com sistemas de gerenciamento educacional para registros de frequência, entregas e interações
\item \textbf{Ferramentas de Comunicação}: Monitoramento de canais de comunicação, sessões de pair programming, fóruns
\item \textbf{Documentos de Projeto}: Processamento de documentação técnica, reflexões individuais e relatórios
\end{itemize}

\textbf{Pipeline de Processamento Proposto:}
\begin{enumerate}
\item \textbf{Injeção de Dados}: Processamento em tempo real de todas as fontes
\item \textbf{Normalização}: Padronização de formatos e metadados
\item \textbf{Enriquecimento}: Adição de contexto semântico via processamento de linguagem natural
\item \textbf{Distribuição}: Envio dos dados processados para agentes especializados
\item \textbf{Análise Adversarial}: Processamento independente por múltiplos agentes
\item \textbf{Consenso}: Integração de perspectivas divergentes
\end{enumerate}

\subsubsection{Tecnologias e Ferramentas Sugeridas}

Para implementação futura, propõem-se as seguintes tecnologias:

\textbf{Infraestrutura de Servidores Locais:}
\begin{itemize}
\item \textbf{Modelos LLM}: Modelos de linguagem executando localmente
\item \textbf{Orquestrador}: Ferramentas para coordenação de fluxos de trabalho
\item \textbf{Armazenamento}: Sistemas de armazenamento de dados e metadados
\item \textbf{Processamento}: Ferramentas para processamento de fluxo e em lote
\item \textbf{Monitoramento}: Ferramentas para observabilidade do sistema
\end{itemize}

\textbf{Algoritmos de Análise Propostos:}
\begin{itemize}
\item \textbf{Análise Semântica}: Modelos baseados em transformers para processamento de texto
\item \textbf{Detecção de Padrões}: Algoritmos de agrupamento (Density-Based Spatial Clustering of Applications with Noise -- DBSCAN, K-means) para comportamentos
\item \textbf{Análise Preditiva}: Modelos de séries temporais (Long Short-Term Memory -- LSTM, Prophet) para predição de risco
\item \textbf{Análise de Redes}: Ferramentas para modelagem de interações colaborativas
\end{itemize}

\subsection{Validação do Modelo Conceitual}

A validação do modelo conceitual foi realizada através de análise teórica e consulta com especialistas em educação e tecnologia\@. O modelo demonstra consistência conceitual e viabilidade técnica para implementação em ambientes de PBL\@.

\subsection{Resultados Esperados da Implementação}

\subsubsection{Eficácia Esperada na Identificação Precoce de Dificuldades}

Baseado na arquitetura de agentes adversariais proposta, espera-se que o modelo possa identificar precocemente dificuldades de aprendizagem através da análise contínua de múltiplas dimensões\@. Os agentes monitorarão indicadores complementares, permitindo detecção de padrões de risco\@.

\textbf{Indicadores de Alerta Propostos:}
\begin{itemize}
\item \textbf{Agente de Artefatos}: Redução na qualidade técnica, frequência de commits, cobertura de testes
\item \textbf{Agente Pedagógico}: Divergência dos objetivos de aprendizagem, lacunas conceituais
\item \textbf{Agente Colaborativo}: Isolamento social, desequilíbrio na contribuição, conflitos não resolvidos
\item \textbf{Agente Temporal}: Inconsistência no ritmo, atrasos recorrentes, padrões de procrastinação
\end{itemize}

\textbf{Mecanismo de Consenso:}
O agente meta-avaliador integrará as perspectivas divergentes, aplicando algoritmos de fusão para gerar alertas consolidados com níveis de confiança\@.

\subsubsection{Redução Esperada da Subjetividade na Avaliação}

A arquitetura adversarial proposta visa reduzir a subjetividade através de múltiplas perspectivas independentes\@. Cada agente aplicará critérios objetivos, criando um sistema de ``checks and balances'' que minimiza vieses individuais\@.

\textbf{Estratégias de Objetividade:}
\begin{itemize}
\item \textbf{Métricas Automatizadas}: Extração automática de indicadores quantitativos dos repositórios
\item \textbf{Avaliação Cruzada}: Cada agente questiona as conclusões dos demais
\item \textbf{Explicabilidade Forçada}: Todo agente deve documentar seu raciocínio
\item \textbf{Calibração Contínua}: Ajuste dos algoritmos baseado em feedback humano
\end{itemize}

\textbf{Métricas Objetivas Propostas:}
\begin{itemize}
\item Complexidade de código, cobertura de testes, débito técnico
\item Distribuição temporal de contribuições, consistência de commits
\item Padrões de interação, centralidade em redes de colaboração
\item Aderência a metodologias ágeis, qualidade da documentação
\end{itemize}

\subsubsection{Melhoria Esperada na Qualidade do Feedback}

O sistema de agentes adversariais proposto deve gerar feedback multidimensional em tempo próximo ao real, com perspectivas que oferecem visão do progresso estudantil.

\textbf{Tipos de Feedback Proposto por Agente:}
\begin{itemize}
\item \textbf{Agente de Artefatos}: Sugestões de refatoração, melhoria de testes, otimização de algoritmos
\item \textbf{Agente Pedagógico}: Recursos de aprendizagem direcionados, lacunas conceituais identificadas
\item \textbf{Agente Colaborativo}: Estratégias para melhorar comunicação, distribuição de tarefas
\item \textbf{Agente Temporal}: Sugestões de planejamento, otimização de cronograma
\end{itemize}

\textbf{Características do Feedback:}
\begin{itemize}
\item \textbf{Contextualizado}: Específico para o momento do projeto e perfil do estudante
\item \textbf{Acionável}: Sugestões concretas com etapas claras de implementação
\item \textbf{Multimodal}: Texto, visualizações, exemplos de código, recursos externos
\item \textbf{Adaptativo}: Evolução baseada na resposta do estudante a feedback anterior
\end{itemize}

\subsubsection{Impacto Esperado no Engajamento Estudantil}

A gamificação implícita do sistema de agentes adversariais pode aumentar o engajamento estudantil. A transparência dos processos avaliativos e o feedback contínuo devem promover autorregulação e consciência metacognitiva.

\textbf{Mecanismos de Engajamento Propostos:}
\begin{itemize}
\item \textbf{Dashboard Pessoal}: Visualização em tempo real do progresso multidimensional
\item \textbf{Comparação Construtiva}: Benchmarking anônimo com pares e turmas anteriores
\item \textbf{Reconhecimento Automático}: Detecção e celebração de marcos e melhorias
\item \textbf{Recomendações Personalizadas}: Sugestões adaptadas ao perfil e interesses do estudante
\end{itemize}

\textbf{Indicadores de Engajamento Monitorados:}
\begin{itemize}
\item Padrões temporais de contribuição e qualidade crescente
\item Complexidade das discussões técnicas e profundidade das perguntas
\item Iniciativa na exploração de funcionalidades avançadas
\item Colaboração voluntária além dos requisitos mínimos
\end{itemize}

\subsection{Análise Esperada por Visão Arquitetural}

\subsubsection{Resultados Esperados da Visão Estrutural}

O agente de análise estrutural mapeará a organização estática dos projetos, identificando correlações entre estrutura organizacional e desempenho das equipes.

\textbf{Dimensões de Análise Estrutural:}
\begin{itemize}
\item \textbf{Topologia de Equipe}: Mapeamento de papéis, responsabilidades e hierarquias emergentes
\item \textbf{Arquitetura de Código}: Análise de modularidade, acoplamento e organização de pacotes
\item \textbf{Distribuição de Recursos}: Balanceamento no acesso a ferramentas e ambiente de desenvolvimento
\item \textbf{Estrutura Documental}: Organização da documentação técnica e especificações
\end{itemize}

\textbf{Insights Esperados:}
\begin{itemize}
\item Correlação entre rotação de papéis e desenvolvimento de competências transversais
\item Impacto da qualidade da arquitetura de código na produtividade da equipe
\item Relação entre acesso equilibrado a recursos e qualidade dos produtos finais
\item Influência da organização documental na efetividade da comunicação
\end{itemize}

\subsubsection{Resultados Esperados da Visão Comportamental}

O agente de análise comportamental modelará dinâmicas de interação, utilizando técnicas de análise de redes sociais e processamento de linguagem natural para capturar padrões de colaboração.

\textbf{Métricas Comportamentais Propostas:}
\begin{itemize}
\item \textbf{Índices de Centralidade}: Betweenness, closeness, eigenvector para identificar líderes emergentes
\item \textbf{Análise de Sentimento}: Processamento de comunicações para detectar tensões e satisfação
\item \textbf{Padrões Temporais}: Ciclos de atividade, sincronização de trabalho, ritmos individuais
\item \textbf{Reciprocidade}: Balanço entre contribuições dadas e recebidas entre membros
\end{itemize}

\textbf{Padrões Comportamentais Esperados:}
\begin{itemize}
\item Correlação entre distribuição equilibrada de comunicação e inovação
\item Impacto de sessões regulares de pair programming na transferência de conhecimento
\item Relação entre resolução construtiva de conflitos e aprendizado coletivo
\item Identificação precoce de isolamento social e necessidade de intervenção
\end{itemize}

\subsubsection{Resultados Esperados da Visão de Processo}

O agente de análise temporal monitorará fluxos de trabalho e aderência a metodologias, aplicando técnicas de process mining para identificar gargalos e oportunidades de melhoria.

\textbf{Métricas Processuais Propostas:}
\begin{itemize}
\item \textbf{Velocidade de Entrega}: Throughput, cycle time, lead time por feature/sprint
\item \textbf{Qualidade do Processo}: Aderência a definição de pronto, cobertura de testes
\item \textbf{Maturidade Ágil}: Score composto baseado em práticas scrum/kanban aplicadas
\item \textbf{Melhoria Contínua}: Freqüência e qualidade de retrospectivas, ações implementadas
\end{itemize}

\textbf{Análises Processuais Esperadas:}
\begin{itemize}
\item Identificação de gargalos através de mapeamento de dependências e filas
\item Correlação entre regularidade de cerimônias ágeis e qualidade do produto
\item Impacto de práticas DevOps na redução de tempo de ciclo
\item Relação entre gestão efetiva de backlog e satisfação do cliente
\end{itemize}

\section{Discussão}

\subsection{Contribuições Científicas}

Esta pesquisa oferece contribuições para as áreas de educação em engenharia e tecnologias educacionais\@. Do ponto de vista conceitual, propõe uma extensão da taxonomia de gêmeos digitais para incluir aplicações educacionais híbridas que combinam processos e sistemas, incorporando agentes LLM adversariais para análise multidimensional\@. A integração das três visões arquiteturais em um framework unificado representa uma abordagem para compreensão dos processos de aprendizagem em PBL\@.

Do ponto de vista metodológico, o modelo estabelece um protocolo para implementação de gêmeos digitais educacionais, incluindo especificações técnicas, métricas e procedimentos de validação empírica\@. A abordagem de coleta e análise de dados multimodais oferece base para futuras pesquisas na intersecção entre tecnologias e educação em engenharia, contribuindo para o avanço do campo de Educational Data Mining\@.

A contribuição teórica reside na demonstração de que gêmeos digitais podem transcender aplicações industriais, oferecendo valor em contextos educacionais\@. O modelo híbrido proposto estabelece precedente para aplicações de gêmeos digitais em metodologias pedagógicas ativas, expandindo as possibilidades para inovação educacional baseada em tecnologias\@.

\subsection{Implicações Práticas}

O modelo conceitual demonstra viabilidade técnica e pedagógica para implementação de gêmeos digitais baseados em agentes LLM para avaliação em PBL\@. A arquitetura adversarial proposta oferece potencial para benefícios na qualidade da educação em engenharia de software, incluindo identificação precoce de dificuldades e provisão de feedback multidimensional\@.

A redução da subjetividade na avaliação, mantendo a flexibilidade pedagógica, oferece solução para um dos desafios do PBL\@. O modelo permite que docentes mantenham seu papel de facilitadores enquanto têm acesso a dados objetivos para fundamentar suas decisões pedagógicas, resultando em avaliações mais justas e precisas\@.

Do ponto de vista institucional, a implementação do modelo pode contribuir para melhoria nos indicadores de qualidade educacional, incluindo taxa de conclusão de cursos, satisfação estudantil e preparação para o mercado de trabalho\@. A capacidade de monitoramento contínuo oferece oportunidades para otimização de recursos educacionais e personalização da experiência de aprendizagem\@.

\subsection{Limitações e Desafios}

A implementação do modelo requer infraestrutura tecnológica e capacitação docente, representando investimentos para instituições educacionais\@. A necessidade de integração com múltiplos sistemas existentes pode apresentar desafios técnicos, especialmente em instituições com infraestrutura tecnológica limitada ou desatualizada\@.

Questões de privacidade e proteção de dados estudantis devem ser consideradas, especialmente em implementações que envolvem monitoramento contínuo de atividades\@. A necessidade de conformidade com regulamentações de proteção de dados (como LGPD no Brasil) requer implementação de salvaguardas técnicas e procedimentais\@.

A generalização dos resultados para diferentes contextos educacionais requer validação adicional, considerando variações culturais, disciplinares e institucionais\@. A dependência de ferramentas tecnológicas pode limitar a aplicabilidade em contextos com recursos limitados, criando potencial para aumento de desigualdades educacionais\@.

Do ponto de vista pedagógico, existe o risco de que o monitoramento excessivo possa inibir a criatividade e experimentação dos estudantes, elementos importantes para aprendizagem em PBL\@. A necessidade de equilibrar transparência com autonomia estudantil representa desafio na implementação prática do modelo\@.

\section{Conclusões}

Este artigo apresentou um modelo conceitual de gêmeo digital híbrido baseado em agentes LLM adversariais para avaliação multidimensional em Project-Based Learning\@. O modelo propõe uma arquitetura distribuída com potencial para superar métodos tradicionais em dimensões: identificação precoce de dificuldades de aprendizagem, redução da subjetividade na avaliação, melhoria na qualidade do feedback pedagógico e aumento do engajamento estudantil\@.

A integração das três visões arquiteturais -- estrutural, comportamental e de processo -- em um framework unificado com agentes adversariais oferece potencial para compreensão dos processos de aprendizagem, possibilitando intervenções pedagógicas precisas\@. A validação conceitual confirma a viabilidade técnica e pedagógica da abordagem proposta para implementação em contextos de PBL em engenharia de software\@.

As contribuições científicas incluem a extensão da taxonomia de gêmeos digitais para aplicações educacionais híbridas com agentes LLM adversariais, o desenvolvimento de um framework conceitual para implementação de gêmeos digitais educacionais, e a proposta de arquitetura para avaliação multidimensional em PBL\@. O modelo estabelece precedente para aplicações de tecnologias em metodologias pedagógicas ativas, contribuindo para o avanço do campo de tecnologias educacionais\@.

Do ponto de vista prático, o modelo conceitual demonstra potencial para a experiência educacional em PBL quando implementado no futuro, oferecendo framework para ferramentas de melhoria da qualidade do ensino e da aprendizagem\@. A capacidade proposta de monitoramento contínuo e análise adversarial pode contribuir para redução de taxas de evasão e melhoria na preparação profissional dos graduados\@.

\subsection{Trabalhos Futuros}

Futuras pesquisas devem focar na implementação prática do modelo proposto, explorando sua aplicabilidade em diferentes disciplinas e níveis educacionais além da engenharia de software\@. A integração com tecnologias como realidade aumentada, explicação automatizada de decisões de IA e blockchain para certificação de competências representa oportunidades para expansão das capacidades da arquitetura adversarial proposta\@.

O desenvolvimento de métricas para competências do século XXI, incluindo pensamento crítico, criatividade, colaboração e comunicação, pode ampliar o escopo de aplicação do gêmeo digital híbrido\@. A incorporação de técnicas de aprendizado de máquina pode melhorar a precisão das análises preditivas e personalização das recomendações pedagógicas\@.

A expansão para contextos internacionais permitirá validação da robustez cultural do modelo, enquanto estudos longitudinais poderão avaliar o impacto na formação profissional e sucesso na carreira dos graduados\@. A investigação de aspectos éticos e sociais do monitoramento contínuo em ambientes educacionais representa área importante para pesquisa futura\@.

O desenvolvimento de versões simplificadas do modelo para instituições com recursos limitados representa oportunidade para democratização dos benefícios desta abordagem, contribuindo para a melhoria da qualidade da educação em engenharia\@. A criação de padrões abertos e protocolos de interoperabilidade pode facilitar a adoção da tecnologia de gêmeos digitais em contextos educacionais diversos\@.

\printbibliography{}

\end{document}
