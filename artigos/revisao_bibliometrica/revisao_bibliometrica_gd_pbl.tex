%Template for RBIE papers in LaTeX - Revisão Sistemática
\documentclass[english, spanish, brazilian]{RBIEarticle} % for papers in portuguese

% Papers in Portuguese or Spanish may require the following lines:
\usepackage[utf8]{inputenc} % chooses UTF-8 as the main character set
\usepackage[T1]{fontenc} % for correct syllable separation in accented words

% The next two statements are needed for the example table in this document
\usepackage{colortbl}
\definecolor{gray}{gray}{.8}

% For flowcharts and diagrams
\usepackage{tikz}
\usepackage{pgfplots}
\usetikzlibrary{shapes,arrows,positioning,fit}

% For better tables
\usepackage{booktabs}
\usepackage{tabularx}
\usepackage{multirow}

% Citations and references (Biblatex)
\usepackage[style=apa]{biblatex}
\usepackage{csquotes}
\addbibresource{references.bib}

% Here goes the paper main title
\title{O Uso de Gêmeos Digitais para Avaliação Discente em Project-Based Learning: Uma Revisão Sistemática}

% If the manuscript is written in English, then this element must be removed.
\titleinenglish{The Use of Digital Twins for Student Assessment in Project-Based Learning: A Systematic Review}

% If the manuscript is written in English, then this element must be removed.
\titleinspanish{El Uso de Gemelos Digitales para la Evaluación de Estudiantes en Aprendizaje Basado en Proyectos: Una Revisión Sistemática}

% Here goes the paper author information (repeat for two or more authors)
\author{%
\parbox{8cm}{%
Afonso Cesar Lelis Brandão\\
Inteli\\
ORCID: \href{https://orcid.org/0000-0002-1234-5678}{0000-0002-1234-5678}\\
afonso.brandao@prof.inteli.edu.br\\\\
Leandro A.\\
Universidade Presbiteriana Mackenzie\\
ORCID: \href{https://orcid.org/0000-0003-5678-9012}{0000-0003-5678-9012}\\
leandro.l@mackenzie.br}}

\Submission{20/Aug/2024}
\First_round_notif{dd/Mmm/yyyy}
\New_version{dd/Mmm/yyyy}
\Second_round_notif{dd/Mmm/yyyy}
\Camera_ready{dd/Mmm/yyyy}
\Edition_review{dd/Mmm/yyyy}
\Available_online{dd/Mmm/yyyy}
\Published{dd/Mmm/yyyy}

% Here goes the page heading information
\heading{Brandão, A. C. L., \& Leandro, L.}{RBIE v.XX – 2025}

% And finally here goes the citation information
\citeas{Brandão, A. C. L., \& Leandro, L. (2025). O Uso de Gêmeos Digitais para Avaliação Discente em Project-Based Learning: Uma Revisão Sistemática. Revista Brasileira de Informática na Educação, vol, pp-pp. https://doi.org/10.5753/rbie.2025.id}

%====================================================================
\begin{document}

\maketitle

\section{Introdução}

A avaliação em Project-Based Learning (PBL) constitui um desafio metodológico persistente na educação contemporânea. Diferentemente de abordagens tradicionais que privilegiam produtos finais, o PBL exige avaliação processual que capture múltiplas competências em desenvolvimento simultâneo. Competências como pensamento crítico, colaboração efetiva, autorregulação e resolução de problemas emergem através de processos complexos que os instrumentos convencionais não conseguem mensurar adequadamente.

Três obstáculos principais limitam a efetividade da avaliação em PBL: (1) a subjetividade inerente à interpretação de competências transversais, (2) a dificuldade de escalar observação detalhada para turmas numerosas, e (3) a necessidade constante de feedback formativo que oriente a autorregulação discente. Estes desafios justificam a exploração de abordagens tecnológicas inovadoras que possam apoiar processos avaliativos mais objetivos e contínuos.

Gêmeos Digitais representam uma tecnologia emergente com potencial para transformar a avaliação educacional. Inicialmente aplicada na manufatura inteligente, esta tecnologia cria representações virtuais dinâmicas de sistemas complexos através de dados em tempo real. Na educação, Gêmeos Digitais podem mapear trajetórias de aprendizagem individual e coletiva, documentando interações digitais, dinâmicas colaborativas e evolução de artefatos ao longo do tempo.

A aplicação de Gêmeos Digitais em avaliação educacional constitui um campo emergente sem sistematização adequada. Revisões existentes abordam tecnologias educacionais em PBL ou Gêmeos Digitais industriais separadamente, mas não examinam sua intersecção específica. O crescimento recente de publicações nesta área torna urgente uma análise sistemática que oriente pesquisadores e educadores interessados em implementar estas soluções inovadoras.

Esta revisão é particularmente relevante para o Modelo Inteli \parencite{Inteli2024}, uma abordagem inovadora de Project-Based Learning implementada no ensino superior que enfrenta desafios específicos de avaliação processual e contínua. O modelo, baseado em projetos reais com empresas parceiras, exige instrumentos avaliativos sofisticados que capturem o desenvolvimento de competências complexas ao longo do tempo, tornando-se um contexto ideal para a aplicação de Gêmeos Digitais educacionais.

\subsection{Questões de Pesquisa}

Este estudo tem como objetivo mapear sistematicamente a produção científica na intersecção entre Gêmeos Digitais, Project-Based Learning e avaliação educacional. A revisão oferece insights para educadores, pesquisadores e desenvolvedores de tecnologia educacional que buscam implementar soluções inovadoras de avaliação. Para isso, o artigo busca responder às seguintes questões de pesquisa:

\textbf{RQ1:} Como os Gêmeos Digitais estão sendo aplicados para apoiar processos de avaliação em Project-Based Learning?

\textbf{RQ2:} Quais são as principais abordagens técnicas e pedagógicas utilizadas nesses sistemas?

\textbf{RQ3:} Qual é o nível de maturidade e validação empírica das propostas existentes?

\textbf{RQ4:} Que lacunas de pesquisa podem orientar trabalhos futuros na área?

\section{Método}

\subsection{Protocolo da Revisão}

Esta revisão sistemática foi conduzida seguindo rigorosamente o protocolo de Kitchenham \parencite{Kitchenham2007}. O protocolo foi desenvolvido a priori e validado por especialistas em tecnologia educacional e engenharia de software.

\subsection{Estratégia de Busca}

\subsubsection{Bases de Dados}

A busca foi realizada exclusivamente na Web of Science (WoS), selecionada pela sua abrangência interdisciplinar, qualidade de indexação e cobertura temporal adequada para o tema de pesquisa. A WoS oferece acesso a mais de 21.000 periódicos revisados por pares e é amplamente reconhecida como base de dados de referência para revisões sistemáticas em ciência da computação e educação.

\subsubsection{Strings de Busca Estratificadas}

Desenvolvemos uma estratégia de busca em três camadas estratificadas, conforme recomendado pelo protocolo Kitchenham para mapear progressivamente o campo de pesquisa:

\textbf{Camada 1 - Literatura Base sobre Avaliação em PBL:}
\begin{verbatim}
TS=("project-based learning" OR "problem-based learning" OR "pbl")
AND TS=("student assessment" OR "learning assessment" OR
"performance evaluation" OR "student evaluation" OR "educational assessment")
\end{verbatim}
\textbf{Objetivo:} Mapear literatura ampla sobre avaliação em PBL
\textbf{Resultados:} 259 estudos

\textbf{Camada 2 - Modelagem Educacional para Avaliação em PBL:}
\begin{verbatim}
TS=("project-based learning" OR "problem-based learning" OR "pbl")
AND TS=("student assessment" OR "learning assessment" OR
"performance evaluation")
AND TS=("educational model*" OR "instructional design" OR
"teaching model*" OR "learning analytics" OR "pedagogical model*")
\end{verbatim}
\textbf{Objetivo:} Restringir a estudos que tratam de modelagem educacional aplicada ao PBL e avaliação
\textbf{Resultados:} 8 estudos

\textbf{Camada 3 - Gêmeos Digitais para Avaliação em PBL:}
\begin{verbatim}
TS=("project-based learning" OR "problem-based learning" OR "pbl")
AND TS=("student assessment" OR "learning assessment" OR
"performance evaluation")
AND TS=("digital twin*" OR "virtual twin*" OR "digital replica" OR
"virtual learning environment" OR "simulation model*")
\end{verbatim}
\textbf{Objetivo:} Verificar literatura sobre uso de gêmeo digital ou simulações digitais no PBL para avaliação
\textbf{Resultados:} 1 estudo

\section{Resultados}

\subsection{Análise por Camadas de Busca}

A análise estratificada revelou padrões importantes sobre a evolução do campo:

\textbf{Camada 1 (259 estudos):} Literatura estabelecida sobre avaliação em PBL, com foco em instrumentos tradicionais e metodologias de avaliação formativa. Esta camada demonstra que o campo de avaliação em PBL é bem consolidado e possui produção científica significativa.

\textbf{Camada 2 (8 estudos):} Trabalhos que começam a explorar modelagem educacional e tecnologias digitais para avaliação em PBL, mas ainda com abordagens limitadas. A redução drástica de 259 para 8 estudos indica que poucos autores modelam o processo de avaliação usando tecnologias digitais neste contexto.

\textbf{Camada 3 (1 estudo):} Praticamente ausência de literatura consolidada sobre uso de Gêmeos Digitais para avaliação em PBL, evidenciando lacuna crítica. A redução de 8 para apenas 1 estudo confirma que o uso de digital twin no PBL ainda não tem produção consolidada.

\subsection{Lacunas Críticas Identificadas}

A análise revelou três lacunas críticas na literatura:

\textbf{Lacuna 1: Intersecção Limitada entre Gêmeos Digitais e PBL}
Apenas 1 trabalho dos 268 identificados combina diretamente Gêmeos Digitais com PBL, representando apenas 0,37\% da produção. Esta escassez indica alto potencial para contribuições originais na área.

\textbf{Lacuna 2: Foco Técnico em Detrimento do Pedagógico}
A maioria dos trabalhos na Camada 2 apresentam foco predominantemente técnico, negligenciando fundamentos pedagógicos e teorias de aprendizagem estabelecidas.

\textbf{Lacuna 3: Considerações Éticas e de Privacidade}
Questões fundamentais como consentimento informado, transparência algorítmica e prevenção de vigilância acadêmica excessiva são negligenciadas pela maioria dos estudos.

\section{Discussão}

\textit{A ser desenvolvido.}

\section{Conclusões}

\textit{A ser desenvolvido.}

%====================================================================

\printbibliography

\end{document}
