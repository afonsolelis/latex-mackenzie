\documentclass{patente}

\begin{document}

\tituloinvencao{Sistema e Método para Monitoramento e Avaliação Automatizada de Projetos de Project Based Learning através de Análise de Repositórios de Versionamento}

\secaopatente{1}{Campo Técnico}
A presente invenção refere-se ao campo de tecnologias educacionais aplicadas ao ensino de programação e desenvolvimento de software, especificamente sistemas automatizados para captura, análise e avaliação de dados provenientes de repositórios de versionamento em ambientes educacionais de programação.

\secaopatente{2}{Estado da Técnica}
Atualmente, a avaliação de projetos de programação em ambientes educacionais é realizada predominantemente através de métodos manuais ou ferramentas de análise estática limitadas. Sistemas existentes de plataformas educacionais especializadas não oferecem análise temporal profunda do processo de desenvolvimento, focando apenas em resultados finais. Não existe uma solução integrada que capture e analise automaticamente padrões de versionamento, evolução temporal do código, colaboração entre estudantes, e métricas de qualidade de software de forma contínua durante o desenvolvimento de projetos educacionais.

\secaopatente{3}{Problema Técnico}
O problema técnico resolvido pela presente invenção é a \problemtecnico{falta de um sistema automatizado que permita o monitoramento contínuo e análise multidimensional de projetos de programação educacionais} através da extração e processamento inteligente de dados de repositórios de versionamento. Problemas específicos incluem: \problemtecnico{ausência de feedback em tempo real sobre práticas de desenvolvimento}, dificuldade em identificar padrões de colaboração e contribuição individual, impossibilidade de detectar automaticamente problemas de qualidade de código durante o desenvolvimento, e limitações na avaliação objetiva do processo de aprendizagem em programação.

\secaopatente{4}{Solução Proposta}
A invenção propõe um sistema automatizado de captura e análise de dados de versionamento que monitora repositórios educacionais em tempo real, extraindo métricas multidimensionais sobre o processo de desenvolvimento. O sistema utiliza \destaque{filtros baseados em palavras-chave de padrões estruturados} para análise de mensagens de versionamento, métricas de engenharia de software para avaliação de qualidade, análise temporal para identificação de padrões de trabalho, e algoritmos de classificação para identificação automática de tipos de atividade de desenvolvimento.

\figurapatenteauto{assets/image1.png}{Arquitetura geral do sistema de captura e análise de versionamento educacional}{fig:arquitetura}

A arquitetura apresentada na Figura \ref{fig:arquitetura} demonstra a
integração entre os diferentes módulos do sistema, evidenciando o fluxo de
dados desde a captura dos repositórios até a geração de relatórios
educacionais.

\secaopatente{5}{Breve Descrição dos Desenhos}
\listafiguras{
  \itemfigura{1}{Arquitetura geral do sistema de captura e análise de versionamento educacional para projetos de Project Based Learning}
}

\secaopatente{6}{Descrição Detalhada}

\subsection*{6.1 Arquitetura do Sistema}
O sistema compreende uma arquitetura modular composta pelos seguintes
componentes principais:

\subsubsection*{6.1.1 Módulo de Captura de Dados de Versionamento}
Responsável pela extração sob demanda de dados de repositórios de versionamento mediante solicitação do usuário, incluindo: histórico completo de alterações com metadados temporais, alterações em arquivos com análise diferencial, informações de autoria e colaboração, estrutura de ramificações e junções, marcações e versões, itens de acompanhamento quando disponíveis, e \textbf{sistema de preservação de estados históricos} que, quando acionado, realiza capturas do estado completo do repositório para proteção contra perda de informações por operações destrutivas de reescrita de histórico.

\subsubsection*{6.1.2 Módulo de Processamento e Análise}
Processa os dados capturados aplicando algoritmos específicos para: análise de mensagens de versionamento usando filtros baseados em palavras-chave de padrões estruturados, cálculo de métricas de engenharia de software como complexidade ciclomática, linhas de código, cobertura de testes, identificação de padrões temporais de desenvolvimento e detecção de anomalias, classificação automática de tipos de atividade de desenvolvimento através de filtros padronizados, e análise de colaboração e contribuição individual em projetos de equipe.

\subsubsection*{6.1.3 Módulo de Avaliação Educacional}
Transforma métricas técnicas em indicadores educacionais relevantes: mapeamento de atividades de desenvolvimento para objetivos de aprendizagem, geração de scores de qualidade de código adaptados ao nível educacional, identificação de dificuldades de aprendizagem através de padrões atípicos, e recomendações personalizadas para melhoria do processo de desenvolvimento.

\subsubsection*{6.1.4 Interface de Visualização e Relatórios}
Apresenta informações através de dashboards interativos com: linha temporal de desenvolvimento com marcos importantes, métricas de qualidade com evolução temporal, comparação entre estudantes e equipes mantendo privacidade, alertas automáticos para professores sobre situações que requerem intervenção, e relatórios personalizáveis para diferentes stakeholders educacionais.

\subsection*{6.2 Algoritmos e Técnicas Inovadoras}

\subsubsection*{6.2.1 Análise de Mensagens de Versionamento por Filtros de Palavras-Chave}
O sistema utiliza \destaque{filtros baseados em palavras-chave derivadas de padrões estruturados} para classificar automaticamente mensagens de versionamento em categorias educacionalmente relevantes. Os filtros identificam \destaque{prefixos padronizados} como indicações de implementação de funcionalidades, correção de defeitos, refatoração de código, adição de documentação, implementação de testes, e outros tipos definidos em padrões reconhecidos. Esta classificação por filtros permite identificar automaticamente o tipo de atividade de aprendizagem sendo realizada sem necessidade de processamento complexo de linguagem natural.

\subsubsection*{6.2.2 Detecção de Padrões Temporais}
Algoritmos de análise temporal identificam padrões comportamentais de desenvolvimento: concentração de atividade próxima a deadlines, distribuição de esforço ao longo do projeto, identificação de sessões de desenvolvimento intensivo, detecção de períodos de inatividade anômalos, e correlação entre padrões temporais e qualidade do código produzido.

\subsubsection*{6.2.3 Métricas de Qualidade Adaptativas}
O sistema calcula métricas de qualidade de software adaptadas ao contexto educacional: complexidade de código ajustada ao nível de experiência dos estudantes, detecção de code smells relevantes para o aprendizado, análise de evolução da qualidade ao longo do tempo, e identificação de melhorias ou degradações no código.

\subsection*{6.3 Funcionalidades Específicas}

\subsubsection*{6.3.1 Sistema Reativo de Coleta e Preservação de Estados}
O sistema opera de forma reativa, coletando e processando dados de repositórios de versionamento sob demanda, mediante solicitação do usuário. Diferentemente de sistemas baseados em eventos, o sistema permanece passivo até ser acionado, momento em que acessa os repositórios configurados para extração e análise dos dados disponíveis. \textbf{Implementa um mecanismo de preservação de estados históricos} que, quando solicitado, captura snapshots completos do repositório, incluindo todas as ramificações, histórico de alterações e metadados associados. Esta funcionalidade é essencial para proteger contra perda de dados educacionais causada por operações de reescrita forçada do histórico de versionamento, garantindo que o processo de aprendizagem seja preservado mesmo em situações de erro ou ações inadvertidas dos estudantes.

\subsubsection*{6.3.2 Análise Comparativa}
Funcionalidade de comparação entre estudantes ou equipes, mantendo anonimização quando necessário. Permite identificação de outliers positivos e negativos, análise de distribuição de performance na turma, e identificação de estudantes que podem se beneficiar de suporte adicional.

\subsubsection*{6.3.3 Sistema de Análise sob Demanda}
O sistema fornece análises detalhadas mediante solicitação do usuário, identificando: períodos de inatividade prolongada, padrões de desenvolvimento que indicam possíveis dificuldades, degradação na qualidade do código, e comportamentos atípicos que podem indicar necessidade de intervenção pedagógica. Todas as informações são apresentadas de forma organizada quando o sistema é consultado, permitindo aos educadores tomar decisões informadas baseadas em dados atualizados.


\secaopatente{7}{Reivindicações}
\begin{reivindicacoes}
  \item Sistema automatizado para monitoramento e avaliação de projetos de programação
  educacionais caracterizado por capturar e analisar continuamente dados de
  repositórios de versionamento, compreendendo módulos de extração de dados,
  processamento através de algoritmos de análise semântica e temporal, geração
  de métricas educacionais específicas, e mecanismo de preservação automática
  de estados históricos do repositório para proteção contra perda de informações
  por operações destrutivas de reescrita.

  \item Método de análise multidimensional de repositórios de versionamento
  educacionais que compreende: extração automatizada de metadados de alterações,
  análise de mensagens usando filtros baseados em palavras-chave de padrões
  estruturados, cálculo de métricas de qualidade de software adaptadas ao
  contexto educacional, identificação de padrões temporais de desenvolvimento, e
  geração de indicadores de aprendizagem baseados em atividades de programação.

  \item Sistema de classificação automática de atividades de desenvolvimento em
  contextos educacionais caracterizado por utilizar filtros baseados em
  palavras-chave de padrões estruturados para categorizar alterações em tipos de
  atividade pedagogicamente relevantes, incluindo implementação, depuração,
  refatoração, documentação e testes.

  \item Interface de visualização para dados de versionamento educacionais
  caracterizada por apresentar métricas de desenvolvimento através de painéis
  adaptativos, linha temporal interativa de evolução de projetos, comparações
  anônimas entre estudantes, e sistema de alertas automáticos para identificação
  de situações que requerem intervenção pedagógica.

  \item Método de detecção de padrões comportamentais em desenvolvimento de software
  educacional através de análise temporal de repositórios de versionamento,
  identificando concentração de atividade, distribuição de esforço, sessões de
  trabalho intensivo, e correlações entre padrões temporais e qualidade de
  código.


  \item Método de avaliação de colaboração em projetos de programação em equipe através
  de análise automatizada de contribuições individuais em repositórios de
  versionamento compartilhados, identificando padrões de participação,
  distribuição de responsabilidades, e efetividade de colaboração.

  \item Sistema de geração de feedback reativo para estudantes de programação
  baseado em análise sob demanda de repositórios de versionamento, fornecendo
  recomendações personalizadas mediante solicitação para melhoria de práticas de desenvolvimento,
  qualidade de código, e gestão de projetos.
\end{reivindicacoes}

\secaopatente{8}{Resumo}
A presente invenção descreve um sistema automatizado para monitoramento e avaliação de projetos de programação educacionais através da captura e análise inteligente de dados de repositórios de versionamento. O sistema utiliza filtros baseados em palavras-chave de padrões estruturados, análise temporal, e métricas de engenharia de software para extrair indicadores educacionais relevantes do processo de desenvolvimento. Características inovadoras incluem classificação automática de atividades de programação através de filtros padronizados, detecção de padrões comportamentais mediante consulta, sistema de análise reativa que responde a solicitações do usuário, interface de visualização adaptada ao contexto educacional, e mecanismo de preservação de estados históricos que protege contra perda de dados por operações destrutivas. A arquitetura modular oferece capacidades de coleta sob demanda, análise comparativa mediante solicitação, e geração de relatórios quando requisitados por educadores e estudantes.

\end{document}
