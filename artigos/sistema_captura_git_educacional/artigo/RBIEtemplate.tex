%
% Template for RBIE papers in LaTeX
%

% The above language combination is for this template document only.
% You should use one of the following:
\documentclass[english, spanish, brazilian]{RBIEarticle} % for papers in portuguese
%\documentclass[brazilian, spanish, english]{RBIEarticle} % for papers in english
%\documentclass[brazilian, english, spanish]{RBIEarticle} % for papers in spanish

% Papers in Portuguese or Spanish may require the following lines:
\usepackage[utf8]{inputenc} % chooses UTF-8 as the main character set
\usepackage[T1]{fontenc} % for correct syllable separation in accented words

% The next two statements are needed for the example table in this document
% (i.e. you don't necessarily need them in your own paper)
\usepackage{colortbl}
\definecolor{gray}{gray}{.8}

% Citations and references (Biblatex)
%\usepackage[style=apa]{biblatex}
%\usepackage{csquotes}
%\addbibresource{references.bib}

% Here goes the paper main title
\title{Um Sistema de Apoio à Avaliação em PBL}

% If the manuscript is written in English, then this element must be removed.
\titleinenglish{A Support System for Assessment in PBL}

% If the manuscript is written in English, then this element must be removed.
\titleinspanish{Un Sistema de Apoyo a la Evaluación en PBL}

% Here goes the paper author information (repeat for two or more authors)
\author{%
	\parbox{8cm}{%
		<Full name Author 1>\\
		<Affiliation>\\
		ORCID: \href{https://orcid.org/0000-0000-0000-0000}{0000-0000-0000-0000}\\
		<author1@my-email>
	}
	\parbox{8cm}{%
		<Full name Author 2>\\
		<Affiliation>\\
		ORCID: \href{https://orcid.org/0000-0000-0000-0000}{0000-0000-0000-0000}\\
		<author2@my-email>
	}
}

\Submission{dd/Mmm/yyyy}
\First_round_notif{dd/Mmm/yyyy}
\New_version{dd/Mmm/yyyy}
\Second_round_notif{dd/Mmm/yyyy}
\Camera_ready{dd/Mmm/yyyy}
\Edition_review{dd/Mmm/yyyy}
\Available_online{dd/Mmm/yyyy}
\Published{dd/Mmm/yyyy}

% Here goes the page heading information
\heading{Surname, initials Author 1, Surname, initials Author 2 (for 1 to 2 Authors)	\\
Last author's surname et al. (for more than 2 authors)
}{RBIE v.VV – yyyy}

% And finally here goes the citation information
\citeas{Last name, Initials., \ldots \& Last name, Initials.  (Year). Article title in the original language. Revista Brasileira de Informática na Educação, vol, pp-pp. https://doi.org/10.5753/rbie.yyyy.id}

%====================================================================
%\hyphenpenalty=10000
%\setcounter{page}{01}

\begin{document}

% Fix babel captions
\addto\captionsbrazilian{%
  \renewcommand{\abstractname}{Resumo}%
}
\addto\captionsspanish{%
  \renewcommand{\abstractname}{Resumen}%
}

\maketitle

% If the manuscript is written in English, then this element must be removed.
\begin{otherlanguage}{brazilian}
\begin{abstract}
Em metodologias de Aprendizagem Baseada em Projetos (PBL), a avaliação da participação individual de estudantes em projetos colaborativos representa um desafio crítico para orientadores. A dificuldade de identificar contribuições individuais em entregas coletivas compromete a qualidade da avaliação e o acompanhamento do desenvolvimento de cada estudante. Este trabalho aborda a problemática específica da visibilidade e rastreabilidade do trabalho individual em projetos de desenvolvimento de software, propondo um sistema automatizado que utiliza dados de repositórios Git para identificar e classificar contribuições individuais ao longo do ciclo de vida do projeto.
\keywords Avaliação Individual; Aprendizagem Baseada em Projetos; Contribuições Colaborativas; Rastreamento de Atividades; Análise de Repositórios; Participação Estudantil.
\end{abstract}
\end{otherlanguage}

\begin{otherlanguage}{english}
\begin{abstract}
In Project-Based Learning (PBL) methodologies, assessing individual student participation in collaborative projects represents a critical challenge for advisors. The difficulty of identifying individual contributions in collective deliverables compromises assessment quality and tracking of each student's development. This work addresses the specific problem of visibility and traceability of individual work in software development projects, proposing an automated system that uses Git repository data to identify and classify individual contributions throughout the project lifecycle.
\keywords Individual Assessment; Project-Based Learning; Collaborative Contributions; Activity Tracking; Repository Analysis; Student Participation.
\end{abstract}
\end{otherlanguage}

% If the manuscript is written in English, then this element must be removed.
\begin{otherlanguage}{spanish}
\begin{abstract}
En metodologías de Aprendizaje Basado en Proyectos (PBL), la evaluación de la participación individual de estudiantes en proyectos colaborativos representa un desafío crítico para orientadores. La dificultad de identificar contribuciones individuales en entregas colectivas compromete la calidad de la evaluación y el seguimiento del desarrollo de cada estudiante. Este trabajo aborda la problemática específica de la visibilidad y trazabilidad del trabajo individual en proyectos de desarrollo de software, proponiendo un sistema automatizado que utiliza datos de repositorios Git para identificar y clasificar contribuciones individuales a lo largo del ciclo de vida del proyecto.
\keywords Evaluación Individual; Aprendizaje Basado en Proyectos; Contribuciones Colaborativas; Seguimiento de Actividades; Análisis de Repositorios; Participación Estudiantil.
\end{abstract}
\end{otherlanguage}

\pagebreak

%====================================================================

\section{Introdução}

A Aprendizagem Baseada em Projetos (PBL) estabelece-se como metodologia educacional onde estudantes trabalham colaborativamente no desenvolvimento de soluções para problemas reais. Neste contexto, um dos desafios mais críticos enfrentados por educadores refere-se à avaliação da participação individual de cada estudante em projetos essencialmente colaborativos \cite{Baker2011}.

Em projetos de desenvolvimento de software, que constituem uma parcela significativa dos projetos em cursos de tecnologia, as entregas finais representam o resultado do trabalho coletivo de toda a equipe. Esta característica colaborativa, embora fundamental para o desenvolvimento de competências de trabalho em equipe, cria uma dificuldade inerente: como identificar e avaliar a contribuição individual de cada membro da equipe?

A problemática da "caixa preta" dos projetos colaborativos manifesta-se quando orientadores têm acesso apenas aos resultados finais das entregas, sem visibilidade sobre o processo de construção, a distribuição de tarefas, os padrões de participação individual e a evolução do trabalho ao longo do tempo. Esta limitação compromete não apenas a capacidade de avaliação justa e precisa, mas também o acompanhamento do desenvolvimento individual de cada estudante.

Este trabalho aborda especificamente a dificuldade de avaliação da participação individual em projetos colaborativos de desenvolvimento de software, investigando como dados de repositórios Git podem ser utilizados para prover visibilidade sobre contribuições individuais e apoiar processos avaliativos mais precisos e justos.

\section{A Problemática da Avaliação Individual em Projetos Colaborativos}

\subsection{O Desafio da Invisibilidade das Contribuições Individuais}

Em projetos colaborativos de desenvolvimento de software, o produto final representa a síntese do trabalho coletivo da equipe. Esta característica colaborativa, fundamental para simular ambientes profissionais reais, cria uma zona de invisibilidade sobre as contribuições individuais de cada membro da equipe. Orientadores frequentemente se encontram na situação de avaliar um produto final sem compreender o processo de construção, a distribuição de responsabilidades ou os diferentes níveis de participação dos estudantes.

Esta invisibilidade manifesta-se em diversas dimensões críticas para o processo educacional:

\textbf{Distribuição desigual de trabalho:} Sem visibilidade sobre o processo de desenvolvimento, é difícil identificar situações onde alguns estudantes assumem responsabilidades desproporcionais enquanto outros participam minimamente do projeto.

\textbf{Tipos diferentes de contribuição:} Estudantes podem contribuir de formas distintas - desenvolvimento de código, documentação, testes, design, pesquisa - mas estas diferenças não são visíveis na entrega final.

\textbf{Evolução temporal do envolvimento:} O nível de participação pode variar ao longo do projeto, mas esta dinâmica temporal é perdida quando se avalia apenas o resultado final.

\subsection{Impactos na Qualidade da Avaliação Educacional}

A limitação de visibilidade sobre contribuições individuais gera múltiplos impactos negativos no processo educacional:

\textbf{Avaliação imprecisa:} Sem dados objetivos sobre participação individual, avaliações baseiam-se em percepções subjetivas ou autorrelatrios dos estudantes, comprometendo a precisão e justiça do processo avaliativo.

\textbf{Reforço de comportamentos inadequados:} Estudantes que participam minimamente podem receber notas semelhantes aos que se dedicam intensamente, criando incentivos inadequados para o engajamento.

\textbf{Perda de oportunidades de desenvolvimento:} Orientadores não conseguem identificar estudantes que precisam de apoio adicional ou aqueles que demonstram habilidades excepcionais em áreas específicas.

\textbf{Acompanhamento insuficiente do progresso:} A incapacidade de rastrear contribuições ao longo do tempo limita a capacidade de acompanhar o desenvolvimento de competências individuais.

\section{Características dos Dados de Desenvolvimento de Software}

\subsection{Potencial Informativo dos Repositórios Git}

Repositórios Git, amplamente utilizados em projetos de desenvolvimento de software, geram naturalmente uma grande quantidade de dados estruturados sobre o processo de desenvolvimento. Estes dados, comumente subutilizados para fins educacionais, contêm informações detalhadas sobre contribuições individuais, padrões de trabalho e evolução temporal dos projetos.

\textbf{Historico de commits:} Cada commit representa uma unidade atômica de trabalho, contendo informações sobre autor, timestamp, arquivos modificados, linhas adicionadas/removidas e mensagem descritiva. Esta granularidade permite rastreamento detalhado das contribuições individuais.

\textbf{Padrões temporais:} A distribuição temporal dos commits revela hábitos de trabalho, períodos de maior atividade e possíveis padrões de procrastinação ou sobrecarga.

\textbf{Tipos de atividade:} Análise das modificações em arquivos e mensagens de commit pode revelar diferentes tipos de contribuição (desenvolvimento de funcionalidades, correção de bugs, documentação, refatoração).

\textbf{Colaboração e interação:} Pull requests e merges fornecem informações sobre processos de revisão de código e colaboração entre membros da equipe.

\subsection{Limitações dos Métodos Convencionais de Acompanhamento}

Métodos tradicionais de acompanhamento em projetos colaborativos apresentam limitações significativas que comprometem a qualidade da avaliação:

\textbf{Dependência de autorrelatórios:} Muitas instituições dependem de relatórios onde estudantes auto-descrevem suas contribuições. Este método é suscetível a viés de autorrelato e pode não refletir a realidade das contribuições.

\textbf{Observação limitada:} Orientações presenciais oferecem apenas snapshots momentâneos do progresso, perdendo a continuidade e os padrões de trabalho entre sessões.

\textbf{Escalabilidade reduzida:} Métodos manuais de acompanhamento não escalam adequadamente quando orientadores são responsáveis por múltiplos grupos simultaneamente.

\textbf{Subjetividade avaliativa:} Avaliações baseadas em percepções podem ser influenciadas por fatores como comunicação verbal, presença em reuniões e outras variáveis não diretamente relacionadas à contribuição técnica.

\subsection{Desafios Práticos na Avaliação Individual}

A experiência prática em instituições que implementam currículos intensivos baseados em projetos revela desafios específicos na avaliação individual:

\textbf{Efeito "carona":} Situações onde alguns estudantes se beneficiam do trabalho de colegas mais dedicados sem contribuir proporcionalmente para o projeto.

\textbf{Concentração de responsabilidades:} Cenários onde estudantes mais experientes ou proativos assumem a maior parte do trabalho, limitando oportunidades de aprendizado para outros membros da equipe.

\textbf{Variação temporal da participação:} Estudantes podem ter níveis diferentes de participação em diferentes fases do projeto, mas esta variação não é capturada em avaliações baseadas apenas no resultado final.

\textbf{Tipos diferentes de contribuição:} Dificuldade em valorizar adequadamente diferentes tipos de contribuição (código, documentação, testes, design) quando todas são essenciais para o sucesso do projeto.

\textbf{Identificação tardia de problemas:} Sem acompanhamento contínuo, problemas de participação só são identificados próximo ao final do projeto, quando já é tarde para intervenções corretivas.

\subsection{Necessidade de Abordagens Automatizadas}

Os desafios identificados apontam para a necessidade de abordagens automatizadas que:

\textbf{Forneçam visibilidade contínua:} Permitam acompanhamento do progresso individual ao longo de todo o ciclo de vida do projeto, não apenas em momentos específicos de avaliação.

\textbf{Sejam objetivas e baseadas em dados:} Reduzam a dependência de percepções subjetivas e autorrelatórios, baseando-se em dados objetivos sobre contribuições efetivas.

\textbf{Escalem adequadamente:} Permitam que orientadores acompanhem múltiplos grupos simultaneamente sem comprometer a qualidade do acompanhamento individual.

\textbf{Preservem a colaboração:} Mantenham o foco no trabalho colaborativo sem criar incentivos para competição disfuncional entre membros da equipe.

\section{Proposta de Sistema Automatizado}

\section{Mineração de Dados Educacionais como Abordagem de Solução}

\subsection{Fundamentação da Mineração de Dados Educacionais}

A Mineração de Dados Educacionais (EDM - Educational Data Mining) representa uma área de pesquisa que aplica técnicas de análise de dados para compreender fenômenos educacionais e melhorar processos de ensino-aprendizagem \cite{Baker2011}. No contexto específico de PBL, a EDM oferece oportunidades únicas para transformar dados naturalmente gerados durante o desenvolvimento de projetos em insights pedagógicos acionáveis.

O desenvolvimento de projetos em ambiente digital gera naturalmente grandes volumes de dados sobre padrões de trabalho, colaboração e evolução de competências. Plataformas de versionamento de código, ferramentas de gestão de projetos e sistemas de comunicação criam rastros digitais detalhados das atividades dos estudantes, representando uma fonte rica de informações sobre o processo de aprendizagem.

\subsection{Potencial dos Dados de Desenvolvimento de Software}

Em projetos de desenvolvimento de software, que constituem uma parcela significativa dos projetos em cursos de tecnologia, plataformas como GitHub geram dados estruturados sobre:

\textbf{Atividade de commits:} Informações sobre frequência, timing, mensagens e arquivos modificados em cada commit, oferecendo insights sobre padrões de trabalho, tipos de atividades realizadas e evolução do projeto.

\textbf{Colaboração através de pull requests:} Dados sobre criação, revisão e merge de pull requests, revelando padrões de colaboração, qualidade de código e processos de revisão entre pares.

\textbf{Participação individual:} Métricas de contribuição individual que podem revelar desequilíbrios de participação, identificar estudantes menos ativos e compreender distribuição de responsabilidades.

\textbf{Evolução temporal:} Padrões temporais de atividade que podem indicar hábitos de trabalho, momentos de intensificação de atividades e identificar potenciais problemas como procrastinação ou sobrecarga.

\subsection{Transformação de Dados em Insights Pedagógicos}

O valor dos dados de desenvolvimento para fins pedagógicos reside na capacidade de transformá-los em insights acionáveis que apoiem decisões educacionais. Esta transformação envolve \cite{Seffrin2013}:

\textbf{Classificação automática de atividades:} Algoritmos podem categorizar commits por tipo de atividade (desenvolvimento de funcionalidades, correção de bugs, documentação, refatoração), oferecendo visão sobre a natureza do trabalho realizado por cada estudante.

\textbf{Identificação de padrões anômalos:} Análises estatísticas podem identificar desvios significativos em padrões de atividade que podem indicar dificuldades, desmotivação ou outros problemas que requerem intervenção pedagógica.

\textbf{Métricas de colaboração:} Análise de redes sociais aplicada a dados de colaboração pode revelar dinâmicas de grupo, identificar estudantes isolados ou dominantes, e compreender fluxos de conhecimento dentro das equipes.

\textbf{Indicadores de progresso:} Métricas longitudinais podem rastrear o desenvolvimento de competências ao longo do tempo, identificando acelerações ou desacelerações no aprendizado.

\section{Direções para Soluções Tecnológicas}

\subsection{Características Desejáveis em Sistemas de Apoio}

Com base na análise da problemática apresentada, sistemas tecnológicos de apoio ao acompanhamento pedagógico em PBL devem apresentar características específicas:

\textbf{Automatização de coleta de dados:} Para reduzir a carga cognitiva extrínseca dos orientadores, sistemas devem automatizar a coleta de dados de atividades de desenvolvimento, eliminando necessidade de monitoramento manual.

\textbf{Análise inteligente e contextualizada:} Além de coletar dados, sistemas devem oferecer análises que considerem o contexto educacional, transformando dados brutos em insights pedagógicos relevantes.

\textbf{Dashboards intuitivos e acionáveis:} Interfaces devem apresentar informações de forma que facilite tomada rápida de decisões pedagógicas, priorizando clareza e acionabilidade sobre completude de dados.

\textbf{Alertas proativos:} Sistemas devem identificar situações que requerem atenção pedagógica e alertar orientadores de forma proativa, permitindo intervenções oportunas.

\textbf{Integração não-invasiva:} Soluções devem integrar-se aos workflows existentes sem criar overhead adicional para estudantes ou orientadores.

\subsection{Potencial de Impacto}

A implementação de sistemas de apoio baseados em análise automática de dados de desenvolvimento tem potencial para transformar significativamente a experiência de orientação em PBL:

\textbf{Liberação de capacidade cognitiva:} Automatizando tarefas operacionais, orientadores podem direcionar mais atenção para atividades de alto valor pedagógico como mentoria personalizada e facilitação de reflexões.

\textbf{Acompanhamento escalável:} Tecnologia pode amplificar a capacidade de acompanhamento individual sem comprometer qualidade, permitindo orientação efetiva de grupos maiores.

\textbf{Intervenções mais oportunas:} Identificação automática de padrões problemáticos permite intervenções pedagógicas mais precoces e efetivas.

\textbf{Personalização baseada em evidências:} Dados objetivos sobre padrões de trabalho e progresso individual podem fundamentar abordagens pedagógicas personalizadas mais efetivas.

\section{Considerações Finais}

A análise apresentada revela que os desafios enfrentados por orientadores em ambientes de Aprendizagem Baseada em Projetos são multifacetados e sistemáticos. A sobrecarga técnica, limitações de escala, falta de visibilidade e fragmentação avaliativa não são problemas isolados, mas manifestações de um desalinhamento fundamental entre as demandas cognitivas impostas aos orientadores e os recursos disponíveis para atendê-las.

A experiência prática de instituições como o Inteli confirma que estes desafios são reais e significativos, impactando diretamente a qualidade da experiência educacional. No entanto, a mesma análise revela oportunidades importantes para mitigação através do uso inteligente de tecnologia educacional.

A Mineração de Dados Educacionais aplicada a dados de desenvolvimento de software emerge como uma abordagem promissora para transformar esta realidade. A abundância de dados gerados naturalmente durante o desenvolvimento de projetos, combinada com técnicas modernas de análise, oferece oportunidades únicas para automatizar aspectos operacionais do acompanhamento pedagógico.

O caminho para soluções efetivas passa pela compreensão profunda da problemática pedagógica, seguida pelo desenvolvimento de ferramentas que automatizem tarefas de baixo valor cognitivo e ampliem a capacidade dos orientadores de focar em atividades estratégicas de alto impacto educacional.

Esta transformação não representa apenas uma otimização operacional, mas uma oportunidade de realinhar a prática pedagógica em PBL com seus objetivos fundamentais: facilitar a construção ativa do conhecimento, desenvolver competências para o século XXI e preparar estudantes para desafios profissionais complexos e em constante evolução.

%====================================================================

\bibliography{references}
\bibliographystyle{plain}

\end{document}