\documentclass[12pt,a4paper]{article}
\usepackage[utf8]{inputenc}
\usepackage[brazil]{babel}
\usepackage{graphicx}
\usepackage{hyperref}
\usepackage{abnt-alf}
\usepackage[top=3cm,bottom=2cm,left=3cm,right=2cm]{geometry}
\usepackage{indentfirst}

\begin{document}

% CAPA
\pagestyle{empty}
\begin{center}
\large  \textbf{UNIVERSIDADE PRESBITERIANA MACKENZIE}
\large  \textbf{PROGRAMA DE PÓS-GRADUAÇÃO EM}\\
\large  \textbf{ENGENHARIA ELÉTRICA E COMPUTAÇÃO}\\
\vskip 2.0cm
\textbf{\large Afonso Cesar Lelis Brandão}\\
\vskip 4.0cm
\setlength{\baselineskip}{1.5\baselineskip}
\textbf{\large Modelo de Gêmeo Digital de Processos e Sistemas para Avaliações em Project Based Learning}\\
\vskip 4.5cm
\end{center}
\hfill{\vbox{\hsize=8.5cm\noindent\strut
Projeto de Pesquisa apresentado ao Programa\break
de Pós-Graduação em Engenharia Elétrica e\break
Computação da Universidade Presbiteriana\break
Mackenzie como parte dos requisitos para\break
qualificação no programa de doutorado.}\\
\strut}
\vskip 3.0cm
\textbf{\normalsize Orientador: Dr. Ismar Frango Silveira}\\
\vskip 2.0cm
\begin{center}
São Paulo, 2025\\
\end{center}

% RESUMO
\newpage
\thispagestyle{plain}
\pagenumbering{roman}
\begin{center}
\large
\textbf{RESUMO}
\end{center}
\renewcommand{\baselinestretch}{0.6666666}
Em criação.
\\[0.5cm]
\begin{flushleft}
{\bf Palavras-chave:} {\it palavra 1, 2, 3...}
\end{flushleft}

% GLOSSÁRIO
\newpage
\thispagestyle{plain}
\begin{center}
\large
\textbf{GLOSSÁRIO}
\end{center}
\renewcommand{\baselinestretch}{1.0}
\normalsize

\begin{description}
\item[TAPI] \textit{Termo de Abertura de Projeto Institucional} - Documento formal que define o escopo, objetivos e parâmetros iniciais de projetos institucionais em metodologias ativas de aprendizagem.

\item[COVID-19] \textit{Coronavirus Disease 2019} - Doença causada pelo coronavírus SARS-CoV-2, identificada pela primeira vez em 2019.

\item[IEEE] \textit{Institute of Electrical and Electronics Engineers} - Instituto de Engenheiros Eletricistas e Eletrônicos. Organização profissional dedicada ao avanço da tecnologia.

\item[Inteli] Instituto de Tecnologia e Liderança - Instituição de ensino superior tecnológico que implementa metodologia PBL integrada com parceiros industriais.

\item[IoT] \textit{Internet of Things} - Internet das Coisas. Rede de objetos físicos incorporados com sensores, software e outras tecnologias para conectar e trocar dados.

\item[ISO/IEC/IEEE] \textit{International Organization for Standardization/International Electrotechnical Commission/Institute of Electrical and Electronics Engineers} - Organizações internacionais de padronização que desenvolvem normas técnicas em conjunto.

\item[MLOps] \textit{Machine Learning Operations} - Práticas e ferramentas para operacionalizar e manter modelos de aprendizado de máquina em produção.

\item[PBL] \textit{Project-Based Learning} - Aprendizagem Baseada em Projetos. Metodologia educacional que organiza o processo de ensino-aprendizagem em torno de projetos autênticos e complexos.

\end{description}

% SUMÁRIO
\newpage
\thispagestyle{empty}
\tableofcontents

% DESENVOLVIMENTO
\newpage
\pagestyle{plain}
\pagenumbering{arabic}
\renewcommand{\baselinestretch}{1.5}
\normalsize

% ORDEM CORRETA: Introdução > Justificativa > Referencial Teórico > Modelo Proposto > Case de Estudo > Resultados > Considerações Finais > Metodologia > Cronograma

% SEÇÃO 1: INTRODUÇÃO
\section{INTRODUÇÃO}

% [Manter todo o conteúdo da introdução existente]

% SEÇÃO 2: JUSTIFICATIVA  
\section{JUSTIFICATIVA}

% [Manter todo o conteúdo da justificativa existente]

% SEÇÃO 3: REFERENCIAL TEÓRICO
\section{REFERENCIAL TEÓRICO}

% [Manter todo o conteúdo do referencial teórico existente]

% SEÇÃO 4: MODELO PROPOSTO
\section{MODELO PROPOSTO}

% [Manter todo o conteúdo do modelo proposto existente]

% SEÇÃO 5: CASE DE ESTUDO - IMPLEMENTAÇÃO PRÁTICA DO MODELO
\section{CASE DE ESTUDO: IMPLEMENTAÇÃO PRÁTICA DO MODELO}

Para demonstração e validação empírica do modelo conceitual proposto, será conduzido case de estudo em ambiente real de PBL, implementando o gêmeo digital de processos e sistemas através de pipeline estruturado que integra múltiplas fontes de dados e ferramentas tecnológicas específicas.

% [Todo o conteúdo detalhado do case de estudo que estava misturado]

% SEÇÃO 6: RESULTADOS COLETADOS (PLACEHOLDER)
\section{RESULTADOS COLETADOS}

Esta seção apresentará os resultados empíricos obtidos durante a implementação do case de estudo, incluindo análise das métricas coletadas, validação das hipóteses de pesquisa e comparação com métodos tradicionais de avaliação. Os resultados serão organizados de acordo com as três visões arquiteturais do modelo proposto e incluirão tanto análises quantitativas quanto qualitativas dos dados obtidos.

\subsection{Resultados da Visão Estrutural}

[A ser preenchido com os resultados da análise estrutural dos projetos PBL]

\subsection{Resultados da Visão Comportamental}

[A ser preenchido com os resultados da análise comportamental das equipes]

\subsection{Resultados da Visão de Processo}

[A ser preenchido com os resultados da análise dos processos ágeis]

\subsection{Validação das Hipóteses}

[A ser preenchido com a validação empírica das hipóteses H1-H5]

\subsection{Análise Comparativa}

[A ser preenchido com a comparação entre o modelo proposto e métodos tradicionais]

% SEÇÃO 7: CONSIDERAÇÕES FINAIS
\section{CONSIDERAÇÕES FINAIS}

Esta seção sintetizará as principais contribuições da pesquisa, limitações encontradas durante a implementação e direcionamentos para trabalhos futuros. Serão apresentadas as implicações práticas dos resultados obtidos para a área de educação em engenharia de software e as perspectivas de aplicação do modelo em diferentes contextos educacionais.

\subsection{Principais Contribuições}

[A ser preenchido com as contribuições científicas, metodológicas e práticas]

\subsection{Limitações da Pesquisa}

[A ser preenchido com as limitações identificadas durante a implementação]

\subsection{Trabalhos Futuros}

[A ser preenchido com propostas de extensão e melhoria do modelo]

\subsection{Implicações para a Prática Educacional}

[A ser preenchido com as recomendações para implementação prática]

% SEÇÃO 8: METODOLOGIA
\section{METODOLOGIA}

% [Manter todo o conteúdo da metodologia existente]

% SEÇÃO 9: CRONOGRAMA
\section{CRONOGRAMA}

% [Manter todo o conteúdo do cronograma existente]

\def\refname{REFERÊNCIAS BIBLIOGRÁFICAS}
\bibliography{biblproj}
\addcontentsline{toc}{section}{REFERÊNCIAS BIBLIOGRÁFICAS}
\bibliographystyle{abnt-alf}

\end{document}