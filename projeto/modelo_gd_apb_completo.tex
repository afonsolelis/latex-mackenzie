\documentclass[12pt, a4paper, oneside]{abntex2}

\usepackage[utf8]{inputenc}
\usepackage{lmodern}
\usepackage[T1]{fontenc}
\usepackage{graphicx}
\usepackage[alf, abnt-emphasize=bf]{abntex2cite}
% --- Sumário ---
\tableofcontents
% ---

% --- ELEMENTOS TEXTUAIS ---
\textual
\onehalfspacing

% --- Elementos Textuais ---
\textual

\section{INTRODUÇÃO}

A Aprendizagem Baseada em Projetos (Project-Based Learning - PBL) tem se
consolidado como uma metodologia educacional que apresenta resultados
documentados no desenvolvimento de competências técnicas e transversais em
cursos de engenharia, com foco específico na integração entre teoria e prática
através de projetos autênticos \cite{zhang2023, lavado2024, guo2020}. Esta
abordagem pedagógica, centrada na resolução de problemas reais e complexos,
promove o aprendizado experiencial através da execução de projetos práticos que
integram conhecimentos teóricos e aplicações práticas. O desenvolvimento de
métodos eficazes para avaliação da aprendizagem neste contexto requer a adoção
de critérios estruturados que contemplem múltiplas dimensões do processo
educacional, incluindo aspectos de utilidade, viabilidade, propriedade e
precisão dos instrumentos de avaliação.

No contexto contemporâneo, os Gêmeos Digitais (Digital Twins) emergem como uma
tecnologia com potencial transformador que pode oferecer representações
virtuais dinâmicas e interativas de sistemas, processos ou entidades, não
necessariamente limitados ao domínio físico \cite{grieves2014, tao2018}.
Diferentemente de simulações estáticas tradicionais, os gêmeos digitais mantêm
sincronização contínua com seus equivalentes reais, permitindo monitoramento em
tempo real, análise preditiva e otimização de processos
\cite{silveira2024panorama}. Esta tecnologia tem encontrado aplicações
promissoras na educação em engenharia, especialmente quando integrada à
metodologia PBL, conforme indicado em estudos recentes que apresentam casos
práticos de desenvolvimento de protótipos físicos e virtuais por estudantes de
engenharia \cite{bachmann2023}.

A taxonomia de gêmeos digitais estabelece quatro categorias principais: gêmeos
de componente, que representam elementos individuais; gêmeos de produto/ativo,
que modelam sistemas completos; gêmeos de sistema, que abrangem múltiplos
ativos interconectados; e gêmeos de processo, que simulam fluxos de trabalho e
procedimentos operacionais \cite{barricelli2019}. Silveira
\cite{silveira2024panorama} complementa esta classificação destacando que, no
contexto educacional, os gêmeos digitais de processo são particularmente
adequados para modelagem de metodologias pedagógicas ativas, diferenciando-se
de simulações estáticas por sua capacidade de atualização contínua e
interatividade em tempo real. Para o contexto educacional de PBL, esta
categoria apresenta relevância específica, pois possibilita a modelagem e
avaliação sistemática dos processos de aprendizagem, incluindo a estrutura
pedagógica, o comportamento dos estudantes durante a execução dos projetos e a
assimilação progressiva dos conceitos do meta-projeto. A aplicação efetiva
desta categoria de gêmeo digital requer a definição de métricas e indicadores
que atendam aos critérios de confiabilidade, validade, usabilidade,
objetividade e normalização estabelecidos para instrumentos de avaliação
educacional.

No âmbito da engenharia de software, a aplicação de visões arquiteturais
tradicionais como estrutural, comportamental e de processo pode ser adaptada
para caracterizar meta-projetos PBL. A visão estrutural define a organização
dos componentes do projeto, incluindo recursos, ferramentas e artefatos. A
visão comportamental modela as interações dinâmicas entre estudantes,
orientadores e sistema durante a execução do projeto. A visão de processo
mapeia os fluxos de atividades, marcos de entrega e critérios de avaliação.
Esta abordagem arquitetural permite uma compreensão sistemática dos elementos
que compõem um projeto PBL e suas inter-relações, fundamentando a criação de
instrumentos de avaliação que contemplem a responsividade às necessidades dos
stakeholders, a justificação de conclusões baseadas em evidências e a
utilização efetiva de recursos.

Apesar dos avanços na aplicação de gêmeos digitais em contextos educacionais,
observa-se uma lacuna significativa na literatura quanto à aplicação específica
desta tecnologia para avaliação sistemática de projetos em PBL na área de
engenharia de software. As iniciativas existentes concentram-se
predominantemente na criação de protótipos físicos e virtuais
\cite{bachmann2023}, não abordando adequadamente a avaliação contínua e
multidimensional dos processos de aprendizagem. Esta lacuna torna-se
particularmente relevante quando se considera a necessidade de implementar
sistemas de avaliação que atendam aos padrões de excelência educacional,
contemplando critérios de efetividade, eficiência, impacto e sustentabilidade
no contexto de programas de engenharia.

\subsection{Objetivo Geral}

Desenvolver um modelo de gêmeo digital de processos e sistemas para avaliação
contínua e multidimensional de projetos em Project-Based Learning na área de
engenharia de software, integrando as visões arquiteturais estrutural,
comportamental e de processo para proporcionar feedback em tempo real sobre a
assimilação de conceitos, evolução do aprendizado e qualidade dos artefatos
produzidos.

\subsection{Objetivos Específicos}

\begin{itemize}
  \item Conceituar um modelo que integre gêmeos digitais de processo com metodologias
        PBL para engenharia de software;
  \item Definir métricas e indicadores de desempenho para avaliação multidimensional de
        projetos PBL utilizando as três visões arquiteturais (estrutural,
        comportamental e de processo);
  \item Modelar a arquitetura do gêmeo digital contemplando a captura, processamento e
        análise de dados provenientes da execução de projetos PBL;
  \item Implementar um protótipo funcional do modelo proposto integrando ferramentas de
        desenvolvimento, controle de versão e plataformas de colaboração;
  \item Validar o modelo através de estudos de caso em disciplinas de engenharia de
        software, comparando os resultados com métodos tradicionais de avaliação;
  \item Estabelecer diretrizes para implantação e escalabilidade do modelo em
        diferentes contextos educacionais.
\end{itemize}

\subsection{Hipótese}

\subsubsection{Hipótese Principal}

A implementação de um modelo de gêmeo digital de processos e sistemas,
integrado às visões arquiteturais estrutural, comportamental e de processo,
proporciona avaliação mais eficaz, objetiva e contínua de projetos em
Project-Based Learning na área de engenharia de software quando comparado aos
métodos tradicionais de avaliação pontual, resultando em:

\begin{itemize}
  \item \textbf{Maior precisão na identificação de dificuldades de aprendizagem}: O monitoramento contínuo permite detectar padrões de comportamento e desempenho que indicam obstáculos no processo de assimilação de conceitos antes que se manifestem como deficiências nos produtos finais;

  \item \textbf{Redução da subjetividade na avaliação}: A coleta automatizada de métricas objetivas sobre o processo de desenvolvimento (commits, refatorações, testes, colaboração) complementa e fundamenta a avaliação docente com dados quantitativos e qualitativos consistentes;

  \item \textbf{Melhoria na qualidade do feedback pedagógico}: A disponibilidade de dados em tempo real sobre múltiplas dimensões do processo de aprendizagem permite intervenções pedagógicas mais oportunas, específicas e personalizadas;

  \item \textbf{Aumento do engajamento e autorregulação dos estudantes}: A visibilidade contínua sobre o próprio progresso e comparação com marcos de referência promove maior consciência metacognitiva e motivação para melhoria contínua.
\end{itemize}

\subsubsection{Hipóteses Secundárias}

Para sustentar e detalhar a hipótese principal, estabelecem-se as seguintes
hipóteses secundárias:

\textbf{H1 - Eficácia da Visão Estrutural}: A modelagem da estrutura estática dos projetos PBL através de gêmeos digitais (recursos, ferramentas, artefatos, organização de equipes) permite identificação mais precisa de lacunas de recursos e inadequações organizacionais que impactam o desempenho das equipes, em conformidade com os viewpoints estruturais definidos pela norma ISO/IEC/IEEE 42010:2022, quando comparada à avaliação baseada exclusivamente em observação docente e autorelatos de estudantes.

\textbf{H2 - Eficácia da Visão Comportamental}: O monitoramento das interações dinâmicas entre estudantes, orientadores e sistemas tecnológicos através de gêmeos digitais revela padrões de colaboração, comunicação e resolução de problemas que não são capturados por métodos de avaliação convencionais, seguindo os princípios de architecture aspects comportamentais estabelecidos pela norma internacional, proporcionando insights valiosos sobre competências transversais e dinâmicas de equipe.

\textbf{H3 - Eficácia da Visão de Processo}: A representação dos fluxos de atividades, marcos de entrega e progressão temporal através de gêmeos digitais pode oferecer compreensão mais detalhada sobre a aderência às metodologias de desenvolvimento de software e a qualidade dos processos adotados pelas equipes, aplicando conceitos de viewpoints de processo conforme especificado na norma ISO/IEC/IEEE 42010:2022, com potencial para superar limitações dos métodos tradicionais de acompanhamento de cronogramas.

\textbf{H4 - Integração Sinérgica das Visões}: A combinação das três visões arquiteturais em um modelo unificado de gêmeo digital produz compreensão holística dos processos de aprendizagem que é superior à soma das contribuições individuais de cada visão, permitindo identificação de inter-relações complexas entre estrutura, comportamento e processo que influenciam o sucesso dos projetos PBL.

\textbf{H5 - Escalabilidade e Transferibilidade}: O modelo de gêmeo digital proposto pode manter sua eficácia quando aplicado a diferentes contextos de projetos PBL em engenharia de software (variando em complexidade, duração, tamanho de equipe e tecnologias utilizadas), sugerindo robustez e aplicabilidade geral da abordagem.

\subsubsection{Critérios de Validação}

A validação das hipóteses será realizada através de critérios quantitativos e
qualitativos específicos:

\textbf{Critérios Quantitativos}:

A definição de métricas quantitativas específicas para validação das hipóteses
apresenta desafios metodológicos significativos, uma vez que não existe
consenso na literatura sobre benchmarks estabelecidos para avaliação de
instrumentos educacionais em contextos de PBL. Reconhecendo esta limitação,
espera-se que o modelo proposto demonstre melhorias mensuráveis em relação aos
métodos tradicionais, sem estabelecer percentuais específicos a priori. Os
critérios quantitativos incluem:

\begin{itemize}
  \item \textbf{Tempo de identificação de dificuldades}: Medição do intervalo temporal entre o surgimento de obstáculos de aprendizagem e sua identificação pelos instrumentos de avaliação, comparando o modelo proposto com métodos convencionais de acompanhamento. Espera-se redução significativa neste intervalo, embora os valores específicos dependam da definição operacional de "dificuldade de aprendizagem" a ser estabelecida durante a implementação;

  \item \textbf{Confiabilidade inter-avaliadores}: Avaliação da correlação entre avaliações realizadas por diferentes docentes utilizando os dados fornecidos pelo gêmeo digital versus avaliações tradicionais. Espera-se aumento na consistência das avaliações, com a magnitude específica dependendo da variabilidade baseline observada no contexto de aplicação;

  \item \textbf{Qualidade dos produtos finais}: Comparação dos indicadores de qualidade dos artefatos produzidos pelos estudantes em projetos utilizando o modelo proposto versus projetos com avaliação tradicional. Os indicadores específicos (funcionalidade, usabilidade, qualidade de código, documentação) serão definidos com base nas características de cada projeto;

  \item \textbf{Engajamento e satisfação dos estudantes}: Medição através de questionários validados e métricas comportamentais (frequência de participação, tempo de dedicação, interações colaborativas). Espera-se melhoria nos índices, com a magnitude dependente dos instrumentos de medição selecionados e do contexto específico de aplicação.
\end{itemize}

A ausência de valores percentuais específicos reflete a natureza exploratória
desta pesquisa e a necessidade de estabelecer baselines empíricos durante a
fase de implementação. A validação focará na demonstração de diferenças
estatisticamente significativas e na magnitude do efeito observado, utilizando
testes apropriados para cada tipo de variável medida.

\textbf{Critérios Qualitativos}:
\begin{itemize}
  \item Evidências de maior especificidade e oportunidade das intervenções pedagógicas;
  \item Demonstração de insights sobre processos de aprendizagem não capturados por
        métodos convencionais;
  \item Confirmação de maior consciência metacognitiva dos estudantes sobre seu próprio
        aprendizado;
  \item Validação da aplicabilidade do modelo em diferentes contextos de projetos PBL.
\end{itemize}

A refutação das hipóteses ocorrerá caso os estudos empíricos não demonstrem
diferenças estatisticamente significativas nos critérios estabelecidos, ou caso
sejam identificadas limitações técnicas ou pedagógicas que impeçam a
implementação prática do modelo proposto em condições reais de ensino.

\section{JUSTIFICATIVA}

A educação em engenharia de software enfrenta desafios significativos
relacionados à necessidade de formar profissionais capazes de lidar com a
complexidade crescente dos sistemas de software contemporâneos. Neste contexto,
a Aprendizagem Baseada em Projetos (PBL) emerge como uma metodologia
educacional essencial, pois proporciona experiências autênticas que simulam as
condições reais do exercício profissional. Entretanto, a implementação eficaz
do PBL em cursos de engenharia de software requer instrumentos de avaliação que
transcendam os métodos tradicionais, contemplando a natureza multidimensional e
processual da aprendizagem experiencial.

A literatura educacional identifica desafios específicos nos métodos de
avaliação aplicados em contextos de PBL, particularmente na área de engenharia
de software. Os instrumentos convencionais de avaliação, que tradicionalmente
focam em produtos finais e momentos pontuais de verificação, apresentam
características distintas dos requisitos de avaliação processual e contínua
demandados pelo PBL \cite{hmelo2004}. Esta limitação torna-se especialmente
crítica quando consideramos que o desenvolvimento de software é, por natureza,
um processo iterativo e colaborativo que demanda competências técnicas,
metodológicas e transversais que se desenvolvem de forma gradual e integrada.

A necessidade de métodos de avaliação contínua e multidimensional em PBL é
corroborada por estudos que sugerem a importância do feedback em tempo real
para o aprendizado efetivo \cite{thomas2000}. No contexto da engenharia de
software, onde projetos tipicamente envolvem ciclos de desenvolvimento,
iterações de design, refatoração de código e integração contínua, a avaliação
deve acompanhar dinamicamente estas transformações, oferecendo insights sobre o
progresso da aprendizagem e identificando oportunidades de intervenção
pedagógica.

A aplicação de tecnologias de Gêmeos Digitais no contexto educacional
representa uma oportunidade inovadora para endereçar estas limitações.
Diferentemente de simulações estáticas ou sistemas de monitoramento pontuais,
os gêmeos digitais podem oferecer capacidades de sincronização contínua com
processos reais, análise preditiva e otimização baseada em dados
\cite{grieves2014, tao2018}. No contexto educacional, estas características
traduzem-se em possibilidades de monitoramento contínuo dos processos de
aprendizagem, análise comportamental de estudantes e equipes, e geração de
insights para otimização das experiências educacionais.

A relevância desta proposta de pesquisa manifesta-se em múltiplas dimensões. Do
ponto de vista \textbf{científico}, o trabalho contribui para o avanço do
conhecimento na interseção entre tecnologias emergentes e educação em
engenharia, área que tem recebido crescente atenção da comunidade acadêmica
internacional. A aplicação sistemática de gêmeos digitais para avaliação
educacional representa uma abordagem inovadora que pode estabelecer novos
paradigmas para instrumentos de avaliação em contextos de aprendizagem ativa.

Do ponto de vista \textbf{metodológico}, a proposta pode oferecer contribuições
para o desenvolvimento de métodos de avaliação que atendam aos critérios de
confiabilidade, validade, usabilidade e objetividade exigidos para instrumentos
educacionais de qualidade. A integração das visões arquiteturais estrutural,
comportamental e de processo em um modelo unificado de gêmeo digital de
processos e sistemas representa uma abordagem sistemática para compreensão
holística dos processos de aprendizagem em PBL.

Do ponto de vista \textbf{tecnológico}, o trabalho contribui para a expansão
das aplicações de gêmeos digitais além dos domínios industriais tradicionais,
indicando sua viabilidade e eficácia em contextos educacionais. Esta
contribuição é particularmente relevante considerando o panorama apresentado
por Silveira e Martins \cite{silveira2024panorama} sobre a necessidade de
desenvolvimento de aplicações educacionais de gêmeos digitais na América
Latina.

Do ponto de vista \textbf{pedagógico}, a pesquisa oferece subsídios para
aprimoramento da qualidade da educação em engenharia de software através do
desenvolvimento de instrumentos de avaliação mais precisos, objetivos e
informativos. A capacidade de fornecer feedback em tempo real sobre múltiplas
dimensões do processo de aprendizagem pode transformar a experiência
educacional, promovendo maior engajamento, motivação e efetividade do
aprendizado.

A \textbf{originalidade} da proposta reside na aplicação específica de gêmeos
digitais de processo para avaliação educacional em PBL, uma abordagem que não
foi encontrada na literatura consultada. Enquanto trabalhos anteriores focam na
utilização de gêmeos digitais como produtos de aprendizagem \cite{bachmann2023}
ou na aplicação de arquiteturas de gêmeos digitais em contextos industriais
\cite{arakaki2022}, a presente proposta posiciona os gêmeos digitais como
instrumentos de avaliação pedagógica, oferecendo uma perspectiva inédita sobre
suas potencialidades educacionais.

A \textbf{viabilidade} técnica da proposta é respaldada pelo amadurecimento das
tecnologias necessárias para sua implementação, incluindo plataformas de
desenvolvimento colaborativo, sistemas de controle de versão, ferramentas de
análise de dados em tempo real e ambientes de desenvolvimento integrados que
oferecem APIs para coleta de métricas de uso. A convergência destas tecnologias
cria condições favoráveis para a implementação prática do modelo proposto.

A \textbf{relevância social} da pesquisa manifesta-se na contribuição para a
melhoria da qualidade da educação superior em engenharia de software, área
estratégica para o desenvolvimento tecnológico e econômico do país. A formação
de profissionais mais qualificados e melhor preparados para os desafios da
indústria de software contribui diretamente para a competitividade nacional no
setor de tecnologia da informação.

Finalmente, a pesquisa alinha-se com tendências internacionais de digitalização
da educação e aplicação de tecnologias emergentes para melhoria dos processos
educacionais. A experiência acumulada durante a pandemia de COVID-19 evidenciou
a importância de tecnologias educacionais robustas e a necessidade de métodos
de avaliação adaptados aos ambientes digitais de aprendizagem. Neste contexto,
o desenvolvimento de instrumentos de avaliação baseados em gêmeos digitais de
processos e sistemas representa uma contribuição oportuna e estratégica para o
futuro da educação em engenharia.

\section{REFERENCIAL TEÓRICO}

\subsection{Aprendizagem Baseada em Projetos (PBL)}

A Aprendizagem Baseada em Projetos (Project-Based Learning - PBL) constitui uma
metodologia educacional com características específicas que a distinguem de
outras abordagens pedagógicas, organizando o processo educacional em torno de
projetos autênticos e complexos onde os estudantes desenvolvem competências
através da execução de atividades práticas e significativas \cite{thomas2000,
  savery2015}. Esta metodologia fundamenta-se nos princípios da aprendizagem
experiencial de Kolb \cite{kolb1984}, que estabelece um ciclo de aprendizado
composto por quatro estágios: experiência concreta, observação reflexiva,
conceituação abstrata e experimentação ativa.

O conceito de PBL caracteriza-se por sua abordagem específica de organizar o
processo educacional em torno de projetos complexos, autênticos e com propósito
definido, complementando outras metodologias pedagógicas através de seu foco na
aplicação prática de conhecimentos \cite{duch2001}. Segundo Hmelo-Silver
\cite{hmelo2004}, a eficácia do PBL reside na integração sistemática de
conhecimentos declarativos (saber o quê), procedimentais (saber como) e
condicionais (saber quando e onde aplicar), promovendo o desenvolvimento de
competências metacognitivas essenciais para a formação profissional em
engenharia.

A estrutura pedagógica do PBL caracteriza-se por elementos fundamentais que
distinguem esta metodologia de abordagens convencionais de ensino por projetos.
Primeiramente, os projetos devem apresentar questões ou problemas complexos que
não possuem soluções únicas ou predeterminadas, exigindo dos estudantes
investigação aprofundada e tomada de decisões fundamentadas \cite{savery2015}.
Em segundo lugar, a autenticidade dos projetos é crucial, devendo refletir
situações reais do contexto profissional e conectar-se com necessidades
genuínas da sociedade ou da indústria. Terceiro, a natureza colaborativa dos
projetos promove o desenvolvimento de competências interpessoais e de trabalho
em equipe, essenciais na prática profissional contemporânea.

A implementação efetiva do PBL em cursos de engenharia requer a consideração de
dimensões pedagógicas específicas que garantam a qualidade do processo
educacional. A dimensão estrutural compreende a organização curricular, a
definição de objetivos de aprendizagem claros e mensuráveis, e a articulação
entre diferentes componentes curriculares. A dimensão processual envolve a
sequenciação de atividades, a gestão do tempo e recursos, e a facilitação do
processo de aprendizagem pelos docentes. A dimensão avaliativa abrange a
definição de critérios e instrumentos de avaliação que contemplem tanto
produtos quanto processos de aprendizagem \cite{thomas2000}.

No contexto da engenharia de software, o PBL encontra aplicação particularmente
relevante devido à natureza intrínseca da área, que envolve a resolução de
problemas complexos através do desenvolvimento de sistemas de software. Os
projetos típicos incluem análise de requisitos de sistemas reais, modelagem de
arquiteturas de software, implementação de protótipos funcionais, realização de
testes e validação, e documentação técnica. Esta abordagem permite aos
estudantes vivenciar o ciclo completo de desenvolvimento de software, desde a
concepção até a entrega, proporcionando compreensão profunda das metodologias,
ferramentas e boas práticas da área.

A avaliação em contextos de PBL representa um desafio metodológico específico,
uma vez que deve contemplar múltiplas dimensões do processo de aprendizagem. Os
critérios convencionais de avaliação, tradicionalmente focados na verificação
de conhecimentos declarativos através de provas e testes, apresentam
características distintas dos requisitos de avaliação processual e contínua
típicos do PBL. Torna-se necessário desenvolver instrumentos de avaliação que
considerem a qualidade dos produtos desenvolvidos, a efetividade dos processos
adotados, o desenvolvimento de competências transversais, e a capacidade de
reflexão crítica sobre o próprio aprendizado \cite{hmelo2004}.

Estudos recentes documentam resultados positivos da aplicação de PBL no
desenvolvimento de competências específicas da engenharia de software. Bachmann
et al. \cite{bachmann2023} apresentam uma revisão de literatura sobre
aplicações de gêmeos digitais na educação, identificando oportunidades de
integração desta tecnologia com metodologias pedagógicas ativas. Os resultados
sugerem potencial para melhorias na capacidade de análise de sistemas, no
desenvolvimento de soluções inovadoras, e na integração de conhecimentos
teóricos com aplicações práticas.

\subsection{Gêmeos Digitais (Digital Twins)}

Os Gêmeos Digitais emergem como uma tecnologia com potencial transformador que
pode representar uma evolução dos paradigmas tradicionais de simulação e
modelagem de sistemas. Grieves \cite{grieves2014} estabelece a definição
seminal de gêmeo digital como uma representação virtual dinâmica de um objeto,
sistema, processo ou entidade que mantém sincronização contínua com seu
equivalente real através de dados em tempo real. Esta definição diferencia
fundamentalmente os gêmeos digitais de simulações estáticas convencionais,
estabelecendo três componentes essenciais: a entidade real (física ou
conceitual), sua representação virtual e a conexão bidirecional de dados que
permite a sincronização contínua.

A evolução conceitual dos gêmeos digitais reflete o amadurecimento das
tecnologias de Internet das Coisas (IoT), computação em nuvem, inteligência
artificial e análise de dados em tempo real. Tao et al. \cite{tao2018} expandem
a conceituação original ao integrar aspectos de big data e aprendizado de
máquina, propondo uma arquitetura que engloba não apenas a representação
virtual, mas também capacidades preditivas e de otimização. Esta evolução
posiciona os gêmeos digitais como sistemas inteligentes capazes de antecipar
comportamentos, identificar anomalias e sugerir melhorias operacionais.

Barricelli et al. \cite{barricelli2019} apresentam uma taxonomia abrangente que
classifica os gêmeos digitais em quatro categorias distintas, cada uma adequada
a diferentes níveis de complexidade e propósitos de aplicação. Os
\textbf{gêmeos de componente} representam elementos individuais de um sistema,
como sensores, atuadores ou dispositivos específicos, fornecendo monitoramento
detalhado e diagnóstico de componentes críticos. Os \textbf{gêmeos de produto
  ou ativo} modelam sistemas completos formados pela integração de múltiplos
componentes, possibilitando análise holística de performance e comportamento.
Os \textbf{gêmeos de sistema} abrangem conjuntos complexos de ativos
interconectados, permitindo compreensão das interdependências e otimização
sistêmica. Por fim, os \textbf{gêmeos de processo} modelam fluxos de trabalho,
procedimentos operacionais e sequências de atividades, oferecendo oportunidades
de otimização de processos e identificação de gargalos operacionais.

A aplicação de gêmeos digitais no contexto educacional representa uma fronteira
emergente com potencial transformador. Silveira e Martins
\cite{silveira2024panorama} analisam o panorama de aplicação de gêmeos digitais
na América Latina, destacando iniciativas pioneiras em educação superior que
utilizam esta tecnologia para simulações realistas e interativas. Os autores
enfatizam a distinção fundamental entre simulações estáticas tradicionais e
gêmeos digitais dinâmicos, ressaltando como a capacidade de atualização em
tempo real e interatividade contínua transforma as possibilidades pedagógicas.

No âmbito específico da educação em engenharia, Bachmann et al.
\cite{bachmann2023} apresentam uma revisão abrangente sobre aplicações de
gêmeos digitais na educação, analisando como esta tecnologia pode ser integrada
a metodologias pedagógicas ativas. A revisão identifica oportunidades para que
estudantes compreendam não apenas os aspectos de implementação, mas também as
complexidades de modelagem virtual, sincronização de dados e análise
comportamental. Os resultados da literatura sugerem potencial para melhorias na
compreensão de conceitos de sistemas complexos, modelagem matemática e
integração de tecnologias emergentes.

A arquitetura de gêmeos digitais para aplicações educacionais requer
considerações específicas que diferem de implementações industriais
tradicionais. Arakaki et al. \cite{arakaki2022} apresentam um modelo
arquitetural de gêmeo digital aplicado com técnicas MLOps que oferece insights
relevantes para contextos educacionais. O modelo proposto integra coleta de
dados em tempo real através de sensores IoT, processamento inteligente
utilizando algoritmos de aprendizado de máquina, e interfaces de visualização
que facilitam a compreensão de comportamentos complexos. Esta arquitetura
sugere como gêmeos digitais podem transcender a mera representação visual,
oferecendo capacidades analíticas e preditivas que podem enriquecer o processo
educacional.

A personalização representa um aspecto particularmente relevante dos gêmeos
digitais em contextos educacionais. Diferentemente de simulações genéricas, os
gêmeos digitais podem adaptar-se às características específicas de cada
projeto, equipe ou contexto de aprendizagem, oferecendo experiências
educacionais customizadas e relevantes. Esta capacidade de personalização
alinha-se com princípios pedagógicos contemporâneos que enfatizam a importância
de atender às necessidades individuais de aprendizagem e promover engajamento
através de experiências significativas e contextualizadas.

\subsection{Integração de PBL e Gêmeos Digitais}

A convergência entre Aprendizagem Baseada em Projetos e tecnologia de Gêmeos
Digitais pode representar uma oportunidade para transformar a educação em
engenharia, particularmente na área de engenharia de software. Esta integração
fundamenta-se na complementaridade natural entre a necessidade de projetos
autênticos e complexos exigidos pelo PBL e as capacidades de modelagem,
simulação e análise oferecidas pelos gêmeos digitais.

Segundo a classificação estabelecida por Silveira \cite{silveira2024panorama} e
discutida na literatura por Bachmann et al. \cite{bachmann2023}, a proposta de
modelo de gêmeo digital para avaliação de projetos PBL enquadra-se em uma
categoria híbrida que combina características de \textbf{gêmeo de processo} e
\textbf{gêmeo de sistema}. Esta classificação justifica-se pela natureza
abrangente do sistema proposto, que visa modelar tanto os processos de
aprendizagem (atividades dos estudantes, evolução dos projetos, interações
colaborativas e assimilação progressiva de conceitos) quanto os sistemas que
suportam esses processos (infraestrutura tecnológica, ambientes de
desenvolvimento, ferramentas colaborativas e plataformas integradas).

O gêmeo híbrido de processos e sistemas proposto diferencia-se de
implementações industriais convencionais por incorporar dimensões pedagógicas
específicas que refletem tanto a complexidade dos processos educacionais quanto
a interação com sistemas tecnológicos de suporte. Enquanto gêmeos industriais
focam tipicamente em otimização de eficiência e redução de custos, o modelo
educacional deve contemplar objetivos de aprendizagem multidimensionais,
incluindo desenvolvimento de competências técnicas, transversais e
metacognitivas, bem como a integração efetiva entre processos pedagógicos e
sistemas tecnológicos.

A arquitetura do gêmeo digital educacional baseia-se na integração das três
visões arquiteturais fundamentais adaptadas do contexto de engenharia de
software: estrutural, comportamental e de processo. A \textbf{visão estrutural}
mapeia tanto os componentes estáticos do projeto PBL quanto os sistemas que os
suportam, incluindo recursos disponíveis, ferramentas utilizadas, artefatos
produzidos, estrutura organizacional das equipes e infraestrutura tecnológica
subjacente. A \textbf{visão comportamental} modela as interações dinâmicas
entre estudantes, orientadores e sistemas tecnológicos, capturando padrões de
colaboração, comunicação, resolução de problemas e a interoperabilidade entre
diferentes sistemas. A \textbf{visão de processo} representa os fluxos de
atividades, marcos de entrega, critérios de avaliação e progressão temporal dos
projetos, bem como os processos de integração e sincronização entre os sistemas
envolvidos.

A definição das visões arquiteturais segue os princípios estabelecidos pela
norma ISO/IEC/IEEE 42010:2022 \cite{iso42010}, que especifica um framework para
descrição de arquitetura baseado em viewpoints (pontos de vista) e views
(visões) arquiteturais. Esta norma internacional fornece uma base metodológica
sólida para a estruturação do gêmeo digital educacional, estabelecendo que cada
preocupação identificada pelos stakeholders deve ser enquadrada por pelo menos
um viewpoint específico. As três visões adotadas - estrutural, comportamental e
de processo - oferecem perspectivas complementares para a compreensão holística
dos processos de aprendizagem em PBL, abrangendo desde a organização estrutural
dos recursos até a dinâmica temporal das atividades educacionais, em
conformidade com os conceitos de architecture aspects definidos pela norma.

Esta abordagem arquitetural permite uma compreensão holística e sistemática dos
elementos que compõem um projeto PBL e suas inter-relações complexas.
Diferentemente de métodos de avaliação tradicionais que capturam apenas
produtos finais ou momentos pontuais do processo educacional, o gêmeo digital
pode oferecer visibilidade contínua sobre a evolução do aprendizado,
possibilitando intervenções pedagógicas oportunas e personalizadas.

A implementação prática desta integração requer a definição de métricas e
indicadores específicos que atendam aos critérios de qualidade estabelecidos
para instrumentos de avaliação educacional. Estes indicadores devem contemplar
aspectos como confiabilidade (consistência temporal das medições), validade
(correspondência entre o que é medido e os objetivos de aprendizagem),
usabilidade (facilidade de interpretação por docentes e estudantes),
objetividade (redução de subjetividade na avaliação) e normalização
(comparabilidade entre diferentes contextos e projetos).

A contribuição inovadora desta proposta reside na aplicação sistemática da
tecnologia de gêmeos digitais especificamente para avaliação contínua e
multidimensional tanto dos processos de aprendizagem quanto dos sistemas que os
suportam em PBL. Enquanto a literatura existente \cite{bachmann2023}
concentra-se na análise de aplicações educacionais de gêmeos digitais de forma
geral, a presente proposta posiciona o gêmeo digital como instrumento de
avaliação pedagógica abrangente, oferecendo capacidades analíticas para
compreensão profunda dos processos de aprendizagem, análise de performance dos
sistemas tecnológicos utilizados e fornecimento de feedback em tempo real para
otimização tanto da experiência educacional quanto da infraestrutura de
suporte.

\subsubsection{Modelo de PBL de Referência: Inteli e a Aplicação Prática de PBL}

O modelo educacional desenvolvido pelo Instituto de Tecnologia e Liderança
(Inteli) \cite{inteli2024} apresenta um exemplo prático de implementação de PBL
em educação superior tecnológica, servindo como base de referência para a
compreensão das dimensões arquiteturais necessárias para modelagem através de
gêmeos digitais de processos e sistemas.

O modelo Inteli fundamenta-se em uma abordagem tridimensional de competências
que integra aspectos técnicos (computação), empresariais (negócios) e de
liderança (soft skills), organizando o processo educacional em torno de
meta-projetos que abordam desafios reais propostos por parceiros industriais e
organizações sociais. A estrutura pedagógica do Inteli implementa o princípio
de "aprendizagem just-in-time", onde conceitos teóricos são apresentados no
momento preciso em que são necessários para o avanço dos projetos.

Esta abordagem oferece um framework concreto para análise das visões
arquiteturais em contextos de PBL:

\textbf{Visão Estrutural no Modelo Inteli}: A organização estrutural compreende a arquitetura de parceiros educacionais (business drivers), onde empresas e organizações sociais atuam como provedores de desafios autênticos, criando um ecossistema de inovação que conecta o ambiente acadêmico às demandas reais do mercado. Os recursos tecnológicos incluem laboratórios de última geração, plataformas de desenvolvimento colaborativo e ambientes integrados que suportam o desenvolvimento de projetos complexos. A estrutura organizacional das equipes segue modelos de gestão ágil, com rotação de papéis e responsabilidades que simulam ambientes profissionais reais.

\textbf{Visão Comportamental no Modelo Inteli}: As interações dinâmicas caracterizam-se pela colaboração intensiva entre estudantes, orientadores e parceiros externos, criando redes de aprendizagem que transcendem o ambiente acadêmico tradicional. As metodologias ativas promovem engajamento dos estudantes através de desafios autênticos que exigem solução criativa e aplicação prática de conhecimentos teóricos, com instrumentos de formalização que estruturam a comunicação entre diferentes stakeholders do processo educacional, permitindo feedback contínuo e ajustes pedagógicos baseados em evidências de aprendizagem.

\textbf{Visão de Processo no Modelo Inteli}: A gestão de meta-projetos segue ciclos iterativos que incluem definição de problemas, desenvolvimento de soluções, implementação e reflexão metacognitiva. Cada meta-projeto é estruturado em marcos de entrega que permitem acompanhamento contínuo do progresso e identificação de oportunidades de intervenção pedagógica. Os processos de avaliação integram múltiplas dimensões, incluindo competências técnicas, colaboração efetiva e capacidade de comunicação, através de critérios que contemplam tanto produtos finais quanto processos de desenvolvimento.

O modelo Inteli também evidencia a importância dos \textbf{requisitos não
  funcionais} em implementações de PBL, incluindo escalabilidade para atender
crescente demanda por profissionais qualificados, flexibilidade para adaptação
a mudanças tecnológicas e requisitos de mercado, e sustentabilidade a longo
prazo do ecossistema de parceiros e recursos educacionais.

Particularmente relevante para o desenvolvimento de gêmeos digitais de
processos e sistemas é a compreensão de que a \textbf{tecnologia representa
  sempre a última camada a ser definida} no modelo Inteli. A escolha de
ferramentas e plataformas tecnológicas é guiada pelos objetivos pedagógicos,
requisitos de projetos específicos e necessidades dos parceiros educacionais,
não por limitações ou preferências tecnológicas pré-definidas. Esta abordagem
alinha-se com os princípios de engenharia de software e design centrado no
usuário, onde soluções tecnológicas devem atender requisitos funcionais
claramente definidos, considerando tanto os processos pedagógicos quanto os
sistemas que os viabilizam.

A aplicação do modelo Inteli como referência para desenvolvimento de gêmeos
digitais de processos e sistemas pode oferecer insights valiosos sobre como
estruturar sistemas de monitoramento e avaliação que contemplem a complexidade
multidimensional da aprendizagem em PBL. A integração das três visões
arquiteturais em um framework coerente pode permitir capturar tanto os aspectos
estáticos (estrutura organizacional, recursos disponíveis, infraestrutura de
sistemas) quanto dinâmicos (interações, progressão temporal, performance
sistêmica) dos processos de aprendizagem e dos sistemas que os suportam, com
potencial para fornecer base sólida para o desenvolvimento de instrumentos de
avaliação eficazes e informativos.

\section{METODOLOGIA}

Esta pesquisa fundamenta-se nos princípios de Design Science Research aplicado
ao desenvolvimento de tecnologias educacionais \cite{gil91}, caracterizando-se
pela criação de um artefato tecnológico inovador (modelo de gêmeo digital de
processos e sistemas) para resolver um problema prático específico (avaliação
contínua e multidimensional em PBL). A abordagem metodológica segue as
diretrizes estabelecidas para pesquisas em educação em engenharia
\cite{andrade99}, integrando desenvolvimento conceitual, implementação prática
e validação empírica em contexto real de aprendizagem.

\subsection{Abordagem Metodológica Geral}

\subsubsection{Design Science Research como Framework Principal}

A pesquisa adota Design Science Research como paradigma metodológico principal,
adequado para investigações que visam criar e avaliar artefatos tecnológicos
destinados a resolver problemas identificados na prática educacional. Esta
escolha metodológica justifica-se pela natureza construtiva da pesquisa, que
busca desenvolver um modelo conceitual original e validá-lo através de
implementação prática, seguindo os princípios estabelecidos para projetos de
pesquisa aplicada \cite{gil91}.

O framework de Design Science Research estrutura-se em cinco etapas sequenciais
e iterativas: (1) identificação do problema e motivação; (2) definição dos
objetivos da solução; (3) projeto e desenvolvimento do artefato; (4)
demonstração da aplicabilidade; e (5) avaliação da eficácia. Esta estrutura
alinha-se com os objetivos específicos estabelecidos para a pesquisa e com as
características do problema investigado, proporcionando base metodológica
sólida para o desenvolvimento e validação do modelo proposto.

\subsubsection{Natureza da Pesquisa}

A pesquisa caracteriza-se como aplicada quanto à sua natureza, pois visa gerar
conhecimentos para aplicação prática na resolução de problemas específicos da
avaliação educacional em PBL \cite{gil91}. Quanto aos objetivos, apresenta
caráter exploratório na fase inicial (revisão da literatura e identificação de
lacunas) e experimental na fase de validação (comparação entre métodos de
avaliação). Do ponto de vista dos procedimentos, combina pesquisa
bibliográfica, desenvolvimento tecnológico e estudo de caso para validação
empírica.

\subsection{Fases da Pesquisa}

\subsubsection{Fase 1: Pesquisa Exploratória e Fundamentação Teórica}

\textbf{Revisão Sistemática da Literatura}: A fundamentação teórica baseia-se em revisão abrangente da literatura nas áreas de Aprendizagem Baseada em Projetos \cite{thomas2000, savery2015, hmelo2004}, tecnologias de Gêmeos Digitais \cite{grieves2014, tao2018, barricelli2019} e suas aplicações educacionais \cite{bachmann2023, silveira2024panorama}. A revisão sistemática seguiu protocolo estruturado para identificação, seleção e análise crítica das fontes primárias, visando mapear o estado da arte e identificar lacunas específicas na intersecção entre estas áreas de conhecimento.

\textbf{Análise de Lacunas}: A análise crítica da literatura permitiu identificar lacuna específica na aplicação de gêmeos digitais para avaliação contínua de projetos PBL em engenharia de software. Esta lacuna fundamenta a originalidade da pesquisa e justifica o desenvolvimento do modelo proposto como contribuição inovadora para as áreas de educação em engenharia e tecnologias educacionais emergentes.

\textbf{Framework Conceitual}: Com base na revisão da literatura e na análise das teorias de aprendizagem experiencial \cite{kolb1984}, foi desenvolvido framework conceitual que integra os princípios pedagógicos do PBL \cite{duch2001} com as capacidades tecnológicas dos gêmeos digitais, estabelecendo base teórica para a modelagem das três visões arquiteturais (estrutural, comportamental e de processo).

\subsubsection{Fase 2: Design e Desenvolvimento do Modelo}

\textbf{Especificação de Requisitos}: O desenvolvimento do modelo iniciou-se com especificação detalhada de requisitos funcionais e não funcionais, baseada na análise das necessidades identificadas na literatura sobre avaliação em PBL \cite{guo2020, lavado2024} e nas capacidades técnicas dos gêmeos digitais aplicados em contextos educacionais. A especificação seguiu os princípios estabelecidos pela norma ISO/IEC/IEEE 42010:2022 \cite{iso42010} para descrição de arquiteturas de sistemas.

\textbf{Arquitetura do Gêmeo Digital}: A modelagem conceitual do gêmeo digital de processos e sistemas foi desenvolvida integrando as três visões arquiteturais fundamentais, conforme framework estabelecido pela norma internacional \cite{iso42010}. A arquitetura contempla tanto os aspectos estáticos (visão estrutural) quanto dinâmicos (visões comportamental e de processo) dos projetos PBL, permitindo monitoramento contínuo e análise multidimensional dos processos de aprendizagem.

\textbf{Implementação das Visões Arquiteturais}: Cada visão arquitetural foi detalhada com especificação de componentes, interfaces, métricas associadas e mecanismos de coleta e processamento de dados. A implementação seguiu princípios de modularidade e extensibilidade, permitindo adaptação a diferentes contextos de PBL e escalabilidade para múltiplos projetos simultâneos.

\textbf{Desenvolvimento Iterativo}: O desenvolvimento seguiu metodologia ágil com ciclos iterativos de prototipagem, teste e refinamento, buscando garantir alinhamento contínuo entre requisitos pedagógicos e capacidades tecnológicas do modelo. Esta abordagem iterativa permitiu incorporação de feedback de especialistas e ajustes baseados em avaliações parciais da arquitetura proposta.

\subsubsection{Fase 3: Case de Estudo}

\textbf{Seleção e Caracterização do Contexto}: O case de estudo será conduzido em disciplina de engenharia de software que adote metodologia PBL, selecionada com base em critérios específicos que incluem: complexidade adequada dos projetos para demonstração das capacidades do gêmeo digital, disponibilidade de infraestrutura tecnológica necessária, e acessibilidade para coleta de dados sobre o processo de aprendizagem. A seleção seguirá o modelo de referência do Inteli \cite{inteli2024}, que demonstra implementação prática bem-sucedida de PBL em educação tecnológica.

\textbf{Participantes e Critérios}: Os participantes incluirão estudantes organizados em equipes de desenvolvimento, docentes orientadores e, quando aplicável, parceiros externos que fornecem desafios autênticos para os projetos. A seleção dos participantes seguirá critérios de representatividade e diversidade, buscando garantir validade externa dos resultados obtidos.

\textbf{Instrumentos de Coleta de Dados}: A coleta de dados será realizada através de múltiplos instrumentos que contemplam as três visões arquiteturais do modelo: (a) métricas quantitativas extraídas automaticamente de repositórios de código, ferramentas de gestão de projetos e plataformas de colaboração; (b) observações qualitativas estruturadas sobre comportamentos colaborativos, comunicação e resolução de problemas; (c) questionários de percepção aplicados a estudantes e docentes sobre a eficácia do processo de avaliação.

\textbf{Procedimentos de Implementação}: A implementação seguirá protocolo estruturado que inclui: configuração do ambiente tecnológico, treinamento dos participantes, período de adaptação ao sistema, coleta de dados durante ciclo completo de desenvolvimento de projeto, e avaliação comparativa com métodos tradicionais de avaliação utilizados na mesma disciplina.

\subsubsection{Fase 4: Avaliação e Validação}

\textbf{Metodologia de Análise de Dados}: A análise dos dados coletados utilizará abordagem de métodos mistos, combinando análise estatística descritiva e inferencial para dados quantitativos, e análise de conteúdo para dados qualitativos. A triangulação de dados de múltiplas fontes buscará garantir robustez e confiabilidade dos resultados obtidos.

\textbf{Validação das Hipóteses}: Cada hipótese de pesquisa (H1 a H5) será validada através de critérios específicos e testes estatísticos apropriados. A validação incluirá análise comparativa entre o modelo proposto e métodos tradicionais de avaliação, medição de correlações entre diferentes dimensões do processo de aprendizagem, e avaliação da eficácia preditiva do modelo em relação aos resultados de aprendizagem.

\textbf{Critérios de Qualidade}: A pesquisa adotará critérios estabelecidos para avaliação de instrumentos educacionais, incluindo confiabilidade (consistência temporal das medições), validade (correspondência entre medições e objetivos de aprendizagem), usabilidade (facilidade de interpretação e uso prático), e objetividade (redução de subjetividade na avaliação). Estes critérios orientarão tanto o desenvolvimento quanto a avaliação do modelo proposto.

\subsection{Considerações Éticas}

A pesquisa seguirá rigorosamente as diretrizes éticas para pesquisas envolvendo
seres humanos, incluindo obtenção de consentimento livre e esclarecido de todos
os participantes, garantia de anonimato e confidencialidade dos dados
coletados, e direito de retirada da pesquisa a qualquer momento sem prejuízos.
O uso de dados de estudantes será restrito aos objetivos da pesquisa, com
implementação de medidas de segurança para proteção das informações coletadas.

\subsection{Limitações Metodológicas}

As principais limitações metodológicas incluem: (a) generalização limitada a
contextos similares de PBL em engenharia de software; (b) dependência da
disponibilidade de infraestrutura tecnológica adequada; (c) possível efeito da
novidade tecnológica sobre o comportamento dos participantes; (d) complexidade
na definição de métricas objetivas para competências transversais. Estas
limitações serão consideradas na interpretação dos resultados e na proposição
de trabalhos futuros.

\section{CRONOGRAMA}

O desenvolvimento desta pesquisa de doutorado está estruturado em fases
sequenciais e sobrepostas, distribuídas ao longo de um período de 9 meses,
culminando com a qualificação prevista para março de 2026. O cronograma
apresentado na Tabela \ref{tab:cronograma} detalha as principais atividades e
seus respectivos períodos de execução.

\begin{table}[htbp]
  \centering
  \caption{Cronograma de Execução da Pesquisa}
  \label{tab:cronograma}
  \begin{tabular}{|l|c|c|c|c|c|c|c|c|c|}
    \hline
    \textbf{Atividade}      & \textbf{Jul} & \textbf{Ago} & \textbf{Set} & \textbf{Out} & \textbf{Nov} & \textbf{Dez} & \textbf{Jan} & \textbf{Fev} & \textbf{Mar} \\
    \hline
    Pesquisa Bibliográfica  & X            & X            & X            &              &              &              &              &              &              \\
    \hline
    Proposição do Modelo    &              & X            & X            &              &              &              &              &              &              \\
    \hline
    Case de Estudo          &              &              &              & X            & X            & X            &              &              &              \\
    \hline
    Análise de Resultados   &              &              &              &              &              & X            & X            & X            &              \\
    \hline
    Preparação Qualificação &              &              &              &              &              &              &              & X            & X            \\
    \hline
    \textbf{Qualificação}   &              &              &              &              &              &              &              &              & \textbf{X}   \\
    \hline
  \end{tabular}
\end{table}

\textbf{Fase 1 - Pesquisa Bibliográfica (Julho-Setembro 2025)}: Revisão sistemática da literatura sobre gêmeos digitais aplicados à educação, metodologias PBL em engenharia de software, e frameworks de avaliação educacional. Esta fase incluirá a análise crítica de trabalhos correlatos e a identificação de lacunas de pesquisa que justifiquem a originalidade da proposta.

\textbf{Fase 2 - Proposição do Modelo (Agosto-Setembro 2025)}: Desenvolvimento do modelo conceitual de gêmeo digital de processos e sistemas para avaliação de projetos PBL, incluindo a especificação detalhada das três visões arquiteturais (estrutural, comportamental e de processo), definição de métricas e indicadores de avaliação, e elaboração dos requisitos funcionais e não funcionais do sistema.

\textbf{Fase 3 - Case de Estudo (Outubro-Dezembro 2025)}: Implementação prática do modelo proposto em contexto real de PBL, incluindo seleção do ambiente educacional, especificação de requisitos específicos do contexto, escolha criteriosa de tecnologias adequadas, desenvolvimento do protótipo funcional, e coleta de dados empíricos para validação.

\textbf{Fase 4 - Análise de Resultados (Dezembro 2025 - Fevereiro 2026)}: Processamento e análise dos dados coletados durante o case de estudo, validação das hipóteses de pesquisa através de testes estatísticos apropriados, avaliação da eficácia do modelo proposto, e documentação dos resultados obtidos.

\textbf{Fase 5 - Preparação para Qualificação (Fevereiro-Março 2026)}: Consolidação de todos os resultados em documento de qualificação, preparação da apresentação oral, e ajustes finais baseados em revisões do orientador e colaboradores.

A \textbf{Qualificação} está prevista para \textbf{março de 2026}, quando serão
apresentados todos os resultados obtidos nas fases anteriores, incluindo o
modelo conceitual validado, os resultados empíricos do case de estudo, e as
contribuições científicas da pesquisa para as áreas de educação em engenharia e
tecnologias educacionais emergentes.

\section{MODELO PROPOSTO}

Com base no referencial teórico apresentado e na análise das lacunas
identificadas na literatura, este capítulo apresenta o modelo conceitual de
gêmeo digital de processos e sistemas desenvolvido especificamente para
avaliação de projetos em Project-Based Learning na área de engenharia de
software. O modelo integra as três visões arquiteturais fundamentais -
estrutural, comportamental e de processo - em conformidade com a norma
ISO/IEC/IEEE 42010:2022, proporcionando uma abordagem sistemática e padronizada
para monitoramento e avaliação contínua dos processos de aprendizagem e dos
sistemas que os suportam.

\subsection{Arquitetura Conceitual do Modelo}

O modelo proposto fundamenta-se na premissa de que os processos de aprendizagem
em PBL e os sistemas que os suportam podem ser efetivamente modelados e
monitorados através de um gêmeo digital híbrido de processos e sistemas que
mantém sincronização contínua com as atividades educacionais reais e a
infraestrutura tecnológica subjacente. A arquitetura conceitual do sistema está
representada na Figura \ref{fig:modelo_proposto}, que ilustra a integração
entre os componentes físicos (estudantes, projetos, recursos) e virtuais
(representação digital, algoritmos de análise, interfaces de feedback) do
modelo.

\begin{figure}[htbp]
  \centering
  \includegraphics[width=0.8\textwidth]{assets/f1.png}
  \caption{Arquitetura Conceitual do Modelo de Gêmeo Digital para Avaliação em PBL}
  \label{fig:modelo_proposto}
\end{figure}

A arquitetura apresentada na Figura \ref{fig:modelo_proposto} ilustra como o
modelo integra dados provenientes de múltiplas fontes (sistemas de controle de
versão, plataformas de colaboração, ferramentas de desenvolvimento, interações
presenciais) para construir uma representação virtual abrangente dos processos
de aprendizagem. Esta integração permite a captura de métricas quantitativas e
qualitativas sobre o progresso dos projetos, a dinâmica das equipes e a
assimilação de conceitos pelos estudantes.

\subsection{Componentes Conceituais do Modelo}

O modelo proposto fundamenta-se na integração conceitual de múltiplas dimensões
de dados que capturam diferentes aspectos do processo de aprendizagem em PBL.
Esta abordagem holística permite construção de representação abrangente e
precisa das atividades educacionais, conforme princípios estabelecidos na
literatura de Educational Data Mining \cite{romero2010, romero2020}.

\subsubsection{Dimensão Pedagógica}

A dimensão pedagógica engloba todos os aspectos diretamente relacionados ao
processo de ensino-aprendizagem, incluindo dados sobre frequência e
participação dos estudantes, perfil e metodologias dos orientadores, estrutura
curricular dos projetos e objetivos de aprendizagem estabelecidos. Esta
dimensão permite monitoramento da evolução das competências técnicas e
transversais ao longo do período de desenvolvimento dos projetos.

\subsubsection{Dimensão de Parceria Externa}

Representa a integração com organizações externas através de documentos
estruturados que especificam requisitos, contexto organizacional e critérios de
avaliação externos. Esta dimensão assegura que o gêmeo digital considere não
apenas aspectos acadêmicos, mas também demandas reais do mercado e da
sociedade, característica fundamental da metodologia PBL.

\subsubsection{Dimensão Técnica de Desenvolvimento}

Compreende aspectos relacionados ao processo de desenvolvimento de software,
incluindo versionamento de código, práticas de colaboração técnica e evolução
temporal dos produtos desenvolvidos. Esta dimensão permite avaliação objetiva
da qualidade técnica e da maturidade das práticas de engenharia de software
adotadas pelas equipes.

\subsubsection{Dimensão de Gestão de Processos}

Engloba métricas e indicadores relacionados à aplicação de metodologias ágeis,
incluindo controle de fluxo de trabalho, produtividade das equipes e evolução
da maturidade na aplicação de processos estruturados de desenvolvimento. Esta
dimensão revela aspectos da organização e eficiência dos processos adotados.

\subsection{Framework de Avaliação Multidimensional}

O sistema de avaliação proposto integra múltiplas perspectivas de análise do
processo de aprendizagem, distribuindo pesos específicos para diferentes
aspectos do desenvolvimento educacional \cite{inteli2024}:

\subsubsection{Avaliação Processual Contínua}

Componente que acompanha o desenvolvimento gradual das competências através de
atividades distribuídas ao longo do período letivo, permitindo identificação
precoce de dificuldades e ajustes pedagógicos oportunos.

\subsubsection{Avaliação de Produto}

Análise do resultado final desenvolvido pelos estudantes, considerando aspectos
de funcionalidade, qualidade técnica, inovação e aderência aos requisitos
estabelecidos pelos parceiros externos.

\subsubsection{Avaliação Conceitual}

Verificação individual da assimilação de conceitos teóricos fundamentais,
garantindo que o aprendizado experiencial seja complementado por sólida base
conceitual.

\subsubsection{Avaliação Colaborativa}

Processo de avaliação entre pares que revela aspectos das dinâmicas de equipe e
do desenvolvimento de competências transversais essenciais para a prática
profissional.

\subsection{Gêmeo Digital de Processos e Sistemas}

O modelo proposto implementa uma abordagem híbrida que combina gêmeo digital de
processos para avaliação do processo educacional com gêmeo digital de sistemas
para avaliação do MVP desenvolvido pelos estudantes em 10 semanas
\cite{barricelli2019}:

\subsubsection{Gêmeo Digital de Processos Educacionais}

Modela e monitora o processo de aprendizagem em si, incluindo progressão
temporal das competências, dinâmicas de equipe, eficácia das intervenções
pedagógicas e aderência aos objetivos de aprendizagem estabelecidos. Este
componente permite identificação precoce de dificuldades de aprendizagem e
personalização das estratégias pedagógicas.

\subsubsection{Gêmeo Digital de Sistemas (MVP)}

Representa virtualmente o produto de software desenvolvido pelos estudantes,
incluindo arquitetura técnica, qualidade de código, funcionalidades
implementadas e evolução temporal do sistema. Este componente permite avaliação
objetiva da qualidade técnica do produto e sua aderência às especificações do
TAPI.

\subsection{Arquitetura Integrada do Modelo}

A integração das múltiplas fontes de dados e sistemas de avaliação é realizada
através de arquitetura em camadas que garante escalabilidade, modularidade e
extensibilidade:

\textbf{Camada de Coleta de Dados}: Implementa conectores específicos para cada fonte de dados, incluindo integração manual com Adalove, API REST para repositórios Git, e processamento via regex/NLP para documentos TAPI, garantindo padronização e integridade das informações coletadas.

\textbf{Camada de Datalake}: Armazena centralizadamente todos os dados coletados em formatos padronizados JSON e Parquet, servindo como repositório único e estruturado para análise posterior por algoritmos de NLP, LLM e Educational Data Mining.

\textbf{Camada de Processamento e Análise}: Aplica algoritmos de análise específicos sobre os dados do datalake, incluindo análise estatística de métricas quantitativas, processamento via LLM para dados qualitativos e algoritmos de correlação para identificação de padrões inter-dimensionais.

\textbf{Camada de Representação Virtual}: Mantém modelos digitais atualizados tanto dos processos educacionais quanto dos sistemas desenvolvidos, permitindo simulação de cenários, análise preditiva e identificação de tendências.

\textbf{Camada de Interface e Feedback}: Oferece dashboards personalizados para diferentes stakeholders, incluindo visões específicas para estudantes (automonitoramento), docentes (acompanhamento pedagógico) e coordenadores (gestão institucional).

\subsection{Visões Arquiteturais Integradas}

Conforme estabelecido pela norma ISO/IEC/IEEE 42010:2022 \cite{iso42010}, o
modelo implementa três viewpoints específicos para capturar diferentes aspectos
dos processos de aprendizagem:

\textbf{Viewpoint Estrutural}: Mapeia a organização estática dos projetos PBL, incluindo estrutura das equipes, distribuição de recursos, arquitetura dos artefatos produzidos e configuração do ambiente de desenvolvimento. Este viewpoint permite identificação de lacunas organizacionais e inadequações na alocação de recursos que possam impactar o desempenho das equipes.

\textbf{Viewpoint Comportamental}: Modela as interações dinâmicas entre os participantes do processo educacional, capturando padrões de colaboração, frequência e qualidade da comunicação, dinâmicas de resolução de problemas e desenvolvimento de competências transversais. Este viewpoint revela aspectos do processo de aprendizagem que são tradicionalmente difíceis de capturar através de métodos convencionais de avaliação.

\textbf{Viewpoint de Processo}: Representa os fluxos temporais de atividades, marcos de entrega, aderência a metodologias de desenvolvimento de software e evolução qualitativa dos produtos desenvolvidos. Este viewpoint pode oferecer visibilidade sobre a qualidade dos processos adotados pelas equipes e sua conformidade com as melhores práticas da engenharia de software.

\subsection{Considerações de Implementação}

A implementação prática do modelo requer consideração de aspectos técnicos,
pedagógicos e éticos específicos do contexto educacional. Do ponto de vista
técnico, o sistema deve buscar garantir escalabilidade para suportar múltiplos
projetos simultâneos, interoperabilidade com ferramentas existentes no ambiente
educacional e performance adequada para análise em tempo real. Do ponto de
vista pedagógico, o modelo deve respeitar a natureza construtivista da
aprendizagem em PBL, evitando interferências excessivas no processo natural de
descoberta e experimentação dos estudantes. Do ponto de vista ético, a
implementação deve buscar garantir privacidade dos dados dos estudantes,
transparência nos critérios de avaliação e uso responsável das informações
coletadas.

\subsubsection{Abordagem Metodológica: Case de Estudo e Escolha Tecnológica}

Em conformidade com os princípios de design centrado no usuário e arquitetura
orientada a requisitos, a validação do modelo proposto será conduzida através
de um \textbf{case de estudo} específico que permitirá implementação prática e
avaliação empírica da eficácia do gêmeo digital de processos e sistemas em
contexto real de PBL.

O case de estudo seguirá a filosofia de que \textbf{a tecnologia representa
  sempre a última camada a ser definida} no processo de design, priorizando a
compreensão profunda dos requisitos pedagógicos, necessidades dos usuários
(estudantes e docentes) e objetivos educacionais antes da seleção de
ferramentas e plataformas tecnológicas específicas. Esta abordagem busca
garantir que as soluções tecnológicas sejam escolhidas com base em critérios de
adequação funcional e pedagógica, não por limitações ou preferências
tecnológicas pré-concebidas.

A estrutura do case de estudo contemplará:

\textbf{Definição do Contexto}: Seleção de uma disciplina de engenharia de software que utilize metodologia PBL, preferencialmente em nível de graduação ou pós-graduação, com projetos de complexidade adequada para demonstração das capacidades do gêmeo digital de processos e sistemas.

\textbf{Identificação de Stakeholders}: Mapeamento dos parceiros educacionais (análogo aos business drivers do modelo Inteli), incluindo docentes orientadores, estudantes participantes, coordenação acadêmica e, quando aplicável, parceiros externos que fornecem desafios autênticos para os projetos.

\textbf{Especificação de Requisitos}: Definição detalhada dos requisitos funcionais e não funcionais do sistema, considerando as três visões arquiteturais (estrutural, comportamental e de processo) e as necessidades específicas do contexto educacional escolhido.

\textbf{Seleção Tecnológica Criteriosa}: Somente após a completa especificação de requisitos, será realizada a seleção de tecnologias, ferramentas e plataformas que melhor atendam aos objetivos pedagógicos identificados. Esta seleção considerará critérios como facilidade de integração com ambientes educacionais existentes, capacidade de coleta de dados em tempo real, escalabilidade, usabilidade para docentes e estudantes, e custos de implementação e manutenção.

\textbf{Implementação Iterativa}: O desenvolvimento do gêmeo digital seguirá metodologias ágeis de desenvolvimento de software, permitindo refinamentos contínuos baseados em feedback dos usuários e análise dos resultados parciais obtidos durante o case de estudo.

Esta abordagem metodológica busca assegurar que o foco do doutorado permaneça
na \textbf{criação do modelo conceitual de gêmeo digital de processos e
  sistemas} para avaliação educacional, utilizando o case de estudo como meio de
validação prática sem que a escolha de tecnologias específicas limite ou
comprometa a generalidade e aplicabilidade do modelo proposto.

\section{CASE DE ESTUDO: IMPLEMENTAÇÃO PRÁTICA DO MODELO}

Para demonstração e validação empírica do modelo conceitual proposto, será
conduzido case de estudo em ambiente real de PBL, implementando o gêmeo digital
de processos e sistemas através de pipeline estruturado que integra múltiplas
fontes de dados e ferramentas tecnológicas específicas.

\subsection{Contexto do Case de Estudo}

O case de estudo será realizado no contexto educacional do Instituto de
Tecnologia e Liderança (Inteli), utilizando como base uma disciplina de
engenharia de software que adote metodologia PBL com duração de 10 semanas.
Esta escolha justifica-se pela maturidade do modelo pedagógico implementado
pela instituição \cite{inteli2024} e pela disponibilidade de infraestrutura
tecnológica adequada para coleta e análise de dados em tempo real.

\subsubsection{Características do Projeto}

O projeto selecionado para implementação do case de estudo seguirá as
características típicas dos meta-projetos do Inteli: desenvolvimento de MVP
(Minimum Viable Product) em 10 semanas, equipes de 4-6 estudantes, parceiro
educacional real que fornece desafio autêntico através de TAPI, e aplicação de
metodologias ágeis para gestão do desenvolvimento.

\subsection{Pipeline de Implementação do Gêmeo Digital}

A implementação do gêmeo digital de processos e sistemas seguirá pipeline
estruturado em cinco etapas principais, garantindo coleta abrangente de dados e
sincronização contínua entre o ambiente real e a representação virtual.

\subsubsection{Etapa 1: Configuração da Infraestrutura de Coleta}

\textbf{Datalake Centralizado}: Implementação de arquitetura de datalake para armazenamento padronizado de todos os dados coletados em formatos JSON e Parquet, garantindo consistência estrutural e facilidade de processamento por algoritmos de NLP e LLM. O datalake servirá como repositório central para todas as fontes de dados do gêmeo digital.

\textbf{Integração Manual com Sistema Adalove}: Coleta manual inicial de dados sobre frequência dos estudantes, perfil dos orientadores, estrutura curricular dos meta-projetos e cronograma de atividades, com posterior armazenamento padronizado no datalake em formato JSON estruturado para análise automatizada.

\textbf{Processamento do TAPI via NLP}: Armazenamento dos documentos TAPI no datalake seguido de processamento através de regex e algoritmos de NLP para extração estruturada de informações, incluindo requisitos funcionais e não funcionais, critérios de aceitação, contexto organizacional do parceiro e métricas de sucesso. Os dados extraídos são convertidos para formato JSON padronizado.

\textbf{Monitoramento de Repositórios Git via API}: Integração automatizada com repositórios GitHub através de API REST HTTP \cite{kalliamvakou2014} para captura de métricas de desenvolvimento, incluindo frequência de commits, tamanho das mudanças, estratégias de branching, processos de merge e práticas de code review \cite{perezriverol2016}. Todos os dados coletados são armazenados em formato JSON/Parquet no datalake.

\textbf{Coleta de Métricas dos Orientadores}: Armazenamento, processamento e padronização em formato JSON das métricas dos orientadores, incluindo frequência de intervenções, feedback fornecido, e avaliações realizadas, garantindo ambiente totalmente padronizado para análise posterior.

\subsubsection{Etapa 2: Processamento e Análise de Dados}

\textbf{Pipeline de Processamento do Datalake}: Implementação de pipeline automatizado para processamento dos dados armazenados no datalake em formatos JSON e Parquet, garantindo normalização, limpeza e estruturação adequada para análise por algoritmos de NLP e LLM. O pipeline utilizará tecnologias de Big Data para processamento eficiente de grandes volumes de dados \cite{romero2020}.

\textbf{Análise via LLM e NLP}: Aplicação de Large Language Models e algoritmos de NLP para análise automatizada dos dados textuais armazenados no datalake, incluindo processamento de documentos TAPI, comentários em código, retrospectivas de sprint, feedback de orientadores e documentação técnica produzida pelos estudantes. Os LLMs permitirão extração de insights qualitativos e quantitativos dos dados não estruturados.

\textbf{Educational Data Mining}: Aplicação de algoritmos de mineração de dados educacionais \cite{romero2010} sobre os dados estruturados em JSON para identificação de padrões comportamentais, correlações entre diferentes dimensões do processo de aprendizagem e detecção de anomalias que possam indicar dificuldades de aprendizagem.

\subsubsection{Etapa 3: Construção do Modelo Virtual}

\textbf{Gêmeo Digital de Processos Educacionais}: Desenvolvimento de representação virtual dos processos de aprendizagem, incluindo modelagem temporal da evolução das competências, dinâmicas de colaboração em equipe e eficácia das intervenções pedagógicas realizadas pelos orientadores.

\textbf{Gêmeo Digital de Sistemas (MVP)}: Criação de modelo virtual do produto de software desenvolvido pelos estudantes, incluindo arquitetura técnica, qualidade de código, funcionalidades implementadas e aderência aos requisitos estabelecidos no TAPI.

\textbf{Sincronização Temporal}: Implementação de mecanismos de sincronização que garantam atualização contínua dos modelos virtuais conforme evolução dos processos reais, mantendo latência mínima entre eventos reais e sua representação digital.

\subsubsection{Etapa 4: Interface e Visualização}

\textbf{Dashboards para Estudantes}: Desenvolvimento de interfaces personalizadas que permitam aos estudantes monitoramento em tempo real de seu progresso individual e da equipe, incluindo métricas de contribuição técnica, evolução das competências e comparação com marcos de referência estabelecidos.

\textbf{Painéis de Controle para Orientadores}: Criação de dashboards específicos para orientadores, oferecendo visibilidade sobre o progresso das equipes, identificação automática de situações que requerem intervenção pedagógica e sugestões baseadas em análise preditiva.

\textbf{Visão Gerencial para Coordenadores}: Implementação de interfaces de alto nível para coordenadores acadêmicos, permitindo análise comparativa entre diferentes projetos, identificação de tendências institucionais e avaliação da eficácia do modelo pedagógico adotado.

\subsubsection{Etapa 5: Validação e Refinamento}

\textbf{Coleta de Feedback Contínuo}: Implementação de mecanismos para coleta sistemática de feedback de todos os stakeholders (estudantes, orientadores, parceiros educacionais) sobre a eficácia e usabilidade do sistema desenvolvido.

\textbf{Análise de Eficácia}: Condução de análise comparativa entre métodos tradicionais de avaliação e o modelo proposto, utilizando métricas quantitativas e qualitativas para validação das hipóteses de pesquisa estabelecidas.

\textbf{Refinamento Iterativo}: Implementação de ciclos de melhoria contínua baseados nos resultados obtidos e feedback coletado, garantindo evolução constante do modelo e sua adaptação às necessidades específicas do contexto educacional.

\subsection{Métricas e Indicadores de Medição}

A implementação prática do modelo requer definição precisa de como cada métrica
proposta será coletada, processada e analisada, garantindo objetividade e
reprodutibilidade dos resultados obtidos.

\subsubsection{Métricas da Dimensão Pedagógica}

\textbf{Frequência e Engajamento}:
\begin{itemize}
  \item \textit{Taxa de Presença}: Percentual de presença física e virtual em atividades síncronas, coletada automaticamente via sistema Adalove
  \item \textit{Tempo de Engajamento}: Duração efetiva de participação em atividades online, medida através de logs de acesso e interação
  \item \textit{Qualidade da Participação}: Análise de contribuições em discussões, frequência de perguntas e nível de interação com colegas e orientadores
\end{itemize}

\textbf{Evolução das Competências}:
\begin{itemize}
  \item \textit{Progressão Temporal}: Análise longitudinal do desenvolvimento de competências técnicas e transversais através de avaliações periódicas
  \item \textit{Autoavaliação vs. Avaliação Externa}: Comparação entre autopercepção dos estudantes e avaliação objetiva de competências
  \item \textit{Transferência de Aprendizado}: Medição da aplicação de conceitos aprendidos em contextos diferentes dentro do mesmo projeto
\end{itemize}

\subsubsection{Métricas da Dimensão Técnica}

\textbf{Análise de Repositórios Git}:
\begin{itemize}
  \item \textit{Frequência de Commits}: Número de commits por estudante por período de tempo, revelando padrões de desenvolvimento
  \item \textit{Tamanho e Complexidade}: Análise de linhas de código modificadas, complexidade ciclomática e refatorações realizadas
  \item \textit{Qualidade de Mensagens}: Avaliação da clareza e descritividade das mensagens de commit através de NLP
  \item \textit{Colaboração}: Análise de pull requests, code reviews e resolução colaborativa de conflitos
\end{itemize}

\textbf{Qualidade de Código}:
\begin{itemize}
  \item \textit{Métricas Estáticas}: Análise automatizada de complexidade, duplicação, cobertura de testes e aderência a padrões
  \item \textit{Dívida Técnica}: Monitoramento da evolução da qualidade técnica ao longo do desenvolvimento
  \item \textit{Arquitetura}: Avaliação da estrutura arquitetural e aderência a princípios de design de software
\end{itemize}

\subsubsection{Métricas de Processos Ágeis}

\textbf{Fluxo de Trabalho}:
\begin{itemize}
  \item \textit{Work in Progress (WIP)}: Contagem automatizada de tarefas em andamento, coletada via integração com ferramentas de gestão
  \item \textit{Throughput}: Taxa de conclusão de user stories ou tarefas por sprint, calculada automaticamente
  \item \textit{Lead Time}: Tempo total desde criação até conclusão de uma funcionalidade, medido through sistema de tickets
  \item \textit{Cycle Time}: Tempo efetivo de desenvolvimento, excluindo períodos de espera ou bloqueio
\end{itemize}

\textbf{Qualidade do Processo}:
\begin{itemize}
  \item \textit{Estabilidade do Throughput}: Análise da variabilidade na velocidade de entrega ao longo do tempo
  \item \textit{Eficácia das Retrospectivas}: Análise qualitativa através de NLP dos pontos levantados e ações implementadas
  \item \textit{Maturidade Ágil}: Avaliação da evolução da aplicação de práticas ágeis pela equipe
\end{itemize}

\subsubsection{Métricas de Resultado}

\textbf{Sistema de Avaliação Integrado}:
\begin{itemize}
  \item \textit{Atividades Ponderadas (35 pontos)}: Coleta automatizada de notas de entregas parciais via sistema Adalove
  \item \textit{Desenvolvimento do Projeto (40 pontos)}: Avaliação automatizada de aspectos técnicos combinada com avaliação manual de inovação
  \item \textit{Prova do Trimestre (20 pontos)}: Integração direta com sistema de avaliação institucional
  \item \textit{Avaliação de Pares (5 pontos)}: Coleta e análise de avaliações mútuas através de formulários estruturados
\end{itemize}

\textbf{Correlações Inter-dimensionais}:
\begin{itemize}
  \item \textit{Frequência vs. Desempenho}: Análise de correlação entre presença e resultados académicos
  \item \textit{Métricas Técnicas vs. Notas}: Correlação entre qualidade de código e avaliações formais
  \item \textit{Processos vs. Produtos}: Análise da relação entre qualidade dos processos ágeis e qualidade dos produtos finais
\end{itemize}

\subsection{Validação das Hipóteses através do Case}

O case de estudo permitirá validação empírica das cinco hipóteses de pesquisa
através de análise comparativa detalhada entre o modelo proposto e métodos
tradicionais de avaliação, utilizando tanto métricas quantitativas quanto
análise qualitativa de percepções dos stakeholders envolvidos no processo
educacional.

\section{RESULTADOS COLETADOS}

Esta seção apresentará os resultados empíricos obtidos durante a implementação
do case de estudo, incluindo análise das métricas coletadas, validação das
hipóteses de pesquisa e comparação com métodos tradicionais de avaliação. Os
resultados serão organizados de acordo com as três visões arquiteturais do
modelo proposto e incluirão tanto análises quantitativas quanto qualitativas
dos dados obtidos.

\subsection{Resultados da Visão Estrutural}

[A ser preenchido com os resultados da análise estrutural dos projetos PBL]

\subsection{Resultados da Visão Comportamental}

[A ser preenchido com os resultados da análise comportamental das equipes]

\subsection{Resultados da Visão de Processo}

[A ser preenchido com os resultados da análise dos processos ágeis]

\subsection{Validação das Hipóteses}

[A ser preenchido com a validação empírica das hipóteses H1-H5]

\subsection{Análise Comparativa}

[A ser preenchido com a comparação entre o modelo proposto e métodos tradicionais]

\section{CONSIDERAÇÕES FINAIS}

Esta seção sintetizará as principais contribuições da pesquisa, limitações
encontradas durante a implementação e direcionamentos para trabalhos futuros.
Serão apresentadas as implicações práticas dos resultados obtidos para a área
de educação em engenharia de software e as perspectivas de aplicação do modelo
em diferentes contextos educacionais.

\subsection{Principais Contribuições}

[A ser preenchido com as contribuições científicas, metodológicas e práticas]

\subsection{Limitações da Pesquisa}

[A ser preenchido com as limitações identificadas durante a implementação]

\subsection{Trabalhos Futuros}

[A ser preenchido com propostas de extensão e melhoria do modelo]

\subsection{Implicações para a Prática Educacional}

[A ser preenchido com as recomendações para implementação prática]

% --- Elementos Pós-Textuais ---
\postextual
\bibliography{biblproj}

\end{document}
